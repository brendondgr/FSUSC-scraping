\documentclass[12pt,a4paper]{article}

% Packages %
\usepackage{titling} % Gives Page Title
\usepackage{setspace} % Gives Ability to use \onehalfspacing and \doublespacing
\usepackage{times} % Time New Roman Font
\usepackage{indentfirst} % Makes the indentation after section
\usepackage{graphicx} % Graphics (Images)
\graphicspath{../images/}

% Bibliography %
\usepackage{natbib}
\usepackage{hyperref}
\bibliographystyle{plain} % Style Here

% Following Are From TablesGenerator.com %
\usepackage[table,xcdraw]{xcolor}
\usepackage{colortbl}

% Title, author, and date %
\title{Title}
\author{Brendon Gutierrez \\ \textit{Class Code: Class name}}
\date{\today}

\begin{document}

\maketitle

\onehalfspacing


\section{What is Computational Science?}
\textit{Source: \url{https://www.sc.fsu.edu/}}

Computational science is an exciting, ever-evolving field that uses computers, networks, software, and algorithms to solve problems, do simulations, build things, and create new knowledge.

\section{What is Data Science?}
\textit{Source: \url{https://www.sc.fsu.edu/}}

We are excited to offer a new 30-credit MS degree in Interdisciplinary Data Science (IDS) with a major in Scientific Computing beginning in Fall 2021!

\section{The Uniqueness of the Department of Scientific Computing at FSU}
\textit{Source: \url{https://www.sc.fsu.edu/}}

Our department offers innovative undergraduate and graduate programs in computational science that impart a synergy among multi-disciplines, thus providing extensive, interdisciplinary hands-on training. We are a self-contained department, unusual among universities, whose faculty members consist of biologists, computer scientists, engineers, geneticists, geophysicists, material scientists, hydrologists, mathematicians, and physicists. Such a varied representation among disciplines opens doors for an even broader spectrum of research interests to be represented in the future.

\section{Facilities}
\textit{Source: \url{https://www.sc.fsu.edu/}}

We maintain a large \textit{computing infrastructure} in support of \textit{research} and \textit{education}. Our computing resources include \textit{large super-computers}, \textit{specialized high-performance computational servers}, a laboratory for \textit{computational intelligence} and \textit{scientific visualization}, a bio-informatics server, a \textit{morphometrics lab}, and much more. We also house a state of the art computer classroom as well as a large multi-purpose seminar room with an 8-foot by 16-foot rear laser projection screen.

\section{Job Placements}
\textit{Source: \url{https://www.sc.fsu.edu/}}

Click here to see a sample of the \textit{excellent jobs} our students are landing with a degree in Computational Science from Florida State University.

\section{Students}
\textit{Source: \url{https://www.sc.fsu.edu/}}

Apply Now to FSU

If you have a question which is not in this section, then please contact us \href{mailto:webmaster@sc.fsu.edu}{here, webmaster@sc.fsu.edu}.

\section{Room Reservations}
\url{https://www.sc.fsu.edu/faq}
\subsection{How do I reserve the Scientific Computing Interactive Classroom (422 DSL)?}
Please Email \texttt{admin@sc.fsu.edu}, Attn: Cecelia Farmer.

\subsection{Who do I contact to reserve the large seminar room (499) or the small conference room (416)?}
Please Email \texttt{admin@sc.fsu.edu}, Attn: Cecelia Farmer.

\subsection{How do I reserve the Dirac Computer Classroom (152)?}
Please Email \texttt{admin@sc.fsu.edu}, Attn: Cecelia Farmer.

\section{Facilities}
\url{https://www.sc.fsu.edu/faq}
\subsection{To whom should I report facility problems with the elevator, lights, bathroom, leaks, etc.?}
Please Email \texttt{admin@sc.fsu.edu}, Attn: Cecelia Farmer.

\subsection{If the coffee supplies run out, what should I do?}
Please Email \texttt{admin@sc.fsu.edu}, Attn: Cecelia Farmer.

\section{For Students}
\url{https://www.sc.fsu.edu/faq}
\subsection{Where can I find forms for graduate students?}
See Links to DocuSign PowerForms below.

\subsection{What information is needed for the prospectus defense?}
\begin{enumerate}
    \item Complete \textit{Ph.D. Committee Form External Link}.
    \item Complete \textit{Ph.D. Result of Prospectus Defense Form External Link} \textit{after} defending.
\end{enumerate}

\subsection{Where can students find career opportunites (internships, fellowships, postdocs, etc.)?}
We offer many links/resources for career seeking students here.

\subsection{Where do I go to apply for the scientific computing graduate program?}
Read the application overview and apply now here!

\subsection{Where can students find info on Graduation checks and clearance?}
Click here for the College of Arts and Sciences Graduation checks and clearance FAQ.

\subsection{What is the recommended minimum score for the TOEFL and GRE?}
For TOEFL a score >80, and for GRE a score >160 makes an application competitive.

\subsection{How can I get help with improving my grammar?}
Try this \textit{online resource External Link}.

\subsection{Looing for research opportunities (and funding) for undergraduates?}
\begin{itemize}
    \item Read about Undergraduate Research Opportunity Program (UROP)
    \item Read about IDEA Grants (formerly URCAA/MRCE)
\end{itemize}

\subsection{Where can graduate students find info on Thesis, Treatise, Dissertation Clearance Procedures and Deadlines?}
Click here for the College of Arts and Sciences Thesis, Treatise, Dissertation Clearance Procedures and Deadlines.

\subsection{Starting today (May 23, 2016), and until further notice, we will be contributing funds to students going to conferences to give presentations or present posters.}
Click here for the Travel Award Policy details.

\subsection{How can I get help with Calculus?}
Try this \textit{Khan Academy online resource External Link}.

\subsection{Where can undergraduate students find more details on the Center for Undergraduate Research \& Academic Engagement (CRE)?}
Click here for \href{http://cre.fsu.edu}{cre.fsu.edu}.
\vspace{1em}

\noindent\textbf{Attachments:}
\begin{itemize}
    \item \texttt{CRE-Resources.pdf}: Presentation (860 kB)
\end{itemize}

\subsection{Is there a Database to search for Funding and Grant opportunities?}
Yes. COS/PIVOT (formerly Community of Science). The Community of Science funding search engine, commonly referred to as COS, has recently changed its name to Pivot.

\subsection{Where can undergraduate students find more details on Financial Information, Tuition, Fees, Aid, Scholarships, and Employment?}
It's all in the UNDERGRADUATE BULLETIN here.

\subsection{Is there a central website where all federal agencies post open procurement opportunities?}
Yes. Federal Business Opportunities (\url{fbo.gov}).

\subsection{Where can students find more details on Scholarships, Grants, and Fellowships that Accept Applications from Non-US Citizens?}
Click here for \textit{Non-US Citizen Opportunties External Link}.

\section{Employment, Personnel, and Visa Issues}
\url{https://www.sc.fsu.edu/faq}
\subsection{Who is the FACULTY liaison for appointments, payroll, benefits and any other questions regarding employment at FSU?}
Please Email \texttt{admin@sc.fsu.edu}.

\subsection{Who processes appointment papers for students, post docs and staff, to include new appointments and reappointments (any non-faculty)?}
Please email \texttt{admin@sc.fsu.edu}.

\subsection{Who can help me with a Visa Problem?}
Please Email \texttt{admin@sc.fsu.edu}, Attn: Cecelia Farmer OR Paul Keyser.

\subsection{Who processes appointment papers for anyone who has a visa?}
Please email \texttt{admin@sc.fsu.edu}.

\subsection{Who should receive my timesheet when it is due?}
Please email \texttt{admin@sc.fsu.edu}.

\subsection{Who is the Personnel Representative for the Department of Scientific Computing?}
Please email \texttt{admin@sc.fsu.edu}.

\section{General}
\url{https://www.sc.fsu.edu/faq}
\subsection{How do I make an appointment to see the Chair?}
Please Email \texttt{admin@sc.fsu.edu}.

\subsection{Looking for Info Regarding Employment Outside the University?}
For any secondary employment by an employer other than the University, a faculty member should fill out what form?

\subsection{Looking for Info Regarding Teaching \& Student/Faculty Interactions?}
Refer to the FSU Faculty Handbook here.

\subsection{Who prepares the department Newsletter and handles public relations?}
Please Email \texttt{admin@sc.fsu.edu}, Attn: DSC Public Relations.

\subsection{How do I ask for computer support?}
Please Email \texttt{ops@sc.fsu.edu}.

\subsection{Who are the current (2017-2019) department faculty senator and alternate?}
Dennis Slice \texttt{dslice@fsu.edu} (Faculty Senator). Peter Beerli \texttt{pbeerli@fsu.edu} (Alternate).

\subsection{Need info about the Family Educational Rights and Privacy Act (FERPA)?}
Then go \href{http://registrar.fsu.edu/ferpa/}{here}.

\subsection{Where can I get information on Title IX?}
\url{hr.fsu.edu}
\vspace{1em}

\noindent\textbf{Attachments:}
\begin{itemize}
    \item \texttt{2016 - 2017 Title IX Reporting training.pptx}: Title IX Reporting (539 kB)
\end{itemize}

\subsection{I want to post a news article to the website. Who can help me with this?}
Please Email \texttt{admin@sc.fsu.edu}, Attn: DSC Public Relations.

\subsection{Is there a Database to search for Funding and Grant opportunities?}
Yes. COS/PIVOT (formerly Community of Science). The Community of Science funding search engine, commonly referred to as COS, has recently changed its name to Pivot.

\section{Grants, Proposals, Budget, and Start-ups}
\url{https://www.sc.fsu.edu/faq}
\subsection{Who do I talk with about New Faculty Start-Up Procedures?}
Please Email \texttt{admin@sc.fsu.edu}, Attn: TBA.

\subsection{Who do you see about funding issues for specific budgets?}
Please Email \texttt{admin@sc.fsu.edu}, Attn: TBA.

\subsection{Who do I contact for help with grants and proposals?}
Please Email \texttt{admin@sc.fsu.edu}, Attn: TBA.

\subsection{Is there a central website where faculty can search all federal agencies for open procurement opportunities?}
Yes. Federal Business Opportunities (\url{fbo.gov}).

\section{Inventory}
\url{https://www.sc.fsu.edu/faq}
\subsection{Who can answer surplus questions or assist with recycling equipment?}
Paul Keyser.

\subsection{Who should I tell about missing or stolen property?}
Please Email \texttt{admin@sc.fsu.edu}, Attn: Paul Keyser.

\subsection{Who can answer inventory questions or assist with paperwork?}
Please Email \texttt{admin@sc.fsu.edu}, Attn: Paul Keyser.

\section{Office Supplies}
\url{https://www.sc.fsu.edu/faq}
\subsection{Who is in charge of ordering office supplies?}
Please Email \texttt{admin@sc.fsu.edu}.

\section{Purchasing}
\url{https://www.sc.fsu.edu/faq}
\subsection{Who do I speak with about purchasing questions via the FSU system or the FSU computer store?}
Please email \texttt{admin@sc.fsu.edu}.

\subsection{Who do I see to order equipment through my grant?}
Please Email \texttt{admin@sc.fsu.edu}, Attn: TBA.

\subsection{What is required for use of department petty cash?}
Please \textit{submit} this \textit{Petty Cash Request Form External Link}. If you are approved, then follow up details will be provided by Paul Keyser.

\section{Seminars and Event Planning}
\url{https://www.sc.fsu.edu/faq}
\subsection{If I need help planning an event, whom do I see?}
Please Email \texttt{admin@sc.fsu.edu}, Attn: Cecelia Farmer.

\subsection{Who will help me advertise a department seminar/workshop via e-mails and/or flyers?}
Please Email \texttt{admin@sc.fsu.edu}.

\section{Students}
\url{https://www.sc.fsu.edu/faq}
\subsection{Who supervises the students?}
Administration.

\subsection{Is it OK for me to ask the students to make copies?}
Yes.

\section{Travel}
\url{https://www.sc.fsu.edu/faq}
\subsection{Who do I see for help with travel arrangements?}
Please email \texttt{admin@sc.fsu.edu} for help with travel arrangements for you, or for visitors or seminar speakers, including airline, hotel, and rental car. We will help to complete the necessary forms as required by FSU.

\section{Visitors}
\url{https://www.sc.fsu.edu/faq}
\subsection{How do I get temporary parking (not more than 1 week) for a visitor?}
Please Email \texttt{admin@sc.fsu.edu}, Attn: Cecelia Farmer.

\subsection{How do I request a user account for a visitor?}
It can be requested by filling out the \textit{Visitor Request Form} if it is for a short-term visitor (less than 2 weeks). For a post-doc or a long term visiting scientist (more than 2 weeks) Contact Administration AND request an account here, \textit{Getting an Account}.

\subsection{What do I do if I have a new post doc or a long term visiting scientist (more than 2 weeks)?}
Contact Administration and they will provide office space, keys, etc. The Host is responsible for computer equipment. The visitor is responsible for requesting an FSUID here, \textit{Getting an Account}.

```latex
\section{Contact Information}
\texttt{https://www.sc.fsu.edu/contact-us}

\subsection{General Contacts}
\begin{tabular}{@{}ll}
    General Contacts & (850) 644-1010 \\
    Dept. Chair      & (850) 644-7054 \\
    Facilities       & (850) 645-0304 \\
    Human Resources (HR) & \\
\end{tabular}

\subsection{Department Address}
\textit{Please use this address for direct shipping (DHL, FedEx, UPS, etc.)}
\vspace{1em}

\noindent
[Name of department member] \\
Florida State University \\
Department of Scientific Computing \\
400 Dirac Science Library \\
Tallahassee, FL 32306-4120

\subsection{Department Chair}
Peter Beerli

\subsection{Directions, Parking, Maps}
\noindent Driving Directions

\subsection{Support and Academic Inquiries}
\begin{itemize}
    \item \textbf{Website Related Questions:} Email Website Support
    \item \textbf{Technical Support Questions:} Email Technical Support
    \item \textbf{Graduate Studies Questions:} Email Our Graduate Advisors
    \item \textbf{Undergraduate Studies Questions:} Email Our Undergraduate Advisors
\end{itemize}
```

\section{Education Overview}
\url{https://www.sc.fsu.edu/education}

\begin{itemize}
    \item Doctor of Philosophy (Ph.D.) in Computational Science
    \item Master of Science (M.S.) in Computational Science
    \item Bachelor of Science (B.S.) in Computational Science
    \item Minor in Computational Science
    \item Doctor of Philosophy (Ph.D.) in Fire Dynamics
    \item Doctor of Philosophy (Ph.D.) in Geophysical Fluid Dynamics
    \item Master of Science (M.S.) in Data Science
\end{itemize}

The educational mission of the Department of Scientific Computing (DSC) is to provide innovative, interdisciplinary undergraduate and graduate training programs in computational science and its applications. The graduate and undergraduate degree programs in the DSC are designed to provide students with a broad training in the design, implementation, and use of algorithms for solving science and engineering problems on computers.

\subsection{The Bachelor of Science (B.S.) degree}
The Department of Scientific Computing offers an innovative \textbf{Bachelor of Science (B.S.) degree program in Computational Science}.

This degree program should be of interest to and is well suited for those who like working on computers and who ordinarily would also be interested in any of the mathematical sciences (mathematics, computer science, statistics), or any of the natural sciences (biology, chemistry, physics, geological sciences, ...), or any engineering discipline. It should be of special interest to students interested in two or more of these areas. Students majoring in Computational Science will learn how to develop and apply new computational tools to solve science and engineering problems.

Please note that \textit{computational science} is \textit{different} from \textit{computer science}. At the risk of oversimplifying things, one can say that computer science is about the science of computers whereas computational science is about the use of computers to solve science and engineering problems.

\subsection{The Minor degree}
The Minor in Computational Science offers a substantive programming and algorithmic knowledge base to non-Computational Science students.

Students who minor in the discipline will develop critical computing and modeling skills that are marketable and attractive to potential employers.

The minor requires at least 14 hours of coursework, and students must make a C- or above in each class for the course to be accepted for minor credit.

\subsection{The Master of Science (M.S.) degrees}
\begin{itemize}
    \item M.S. in Computational Science
    \item M.S. in Data Science
\end{itemize}

\subsection{The Doctor of Philosophy (Ph.D.) degree}
The Doctor of Philosophy (Ph.D.) degree consists of several majors:
\begin{itemize}
    \item Ph.D. in Computational Science
    \begin{itemize}
        \item with a Specialization in Atmospheric Science
        \item with a Specialization in Biochemistry
        \item with a Specialization in Biological Science
        \item with a Specialization in Geological Science
        \item with a Specialization in Materials Science
        \item with a Specialization in Physics
    \end{itemize}
\end{itemize}
Students who choose the first major specialize in the more mathematical or computer science aspects of computational science.

Because computational science lies at the intersection of applied mathematics, applied science, engineering, and computer science, the DSC has the unique ability to offer coursework and research opportunities at all levels in topic areas that cut across disciplines. The focus of the research and training activities of the DSC is on the invention, analysis, and implementation of computational algorithms that transcend disciplines and the application of such algorithms to various applications, including, among others, astrophysics, bioinformatics, climate and weather modeling, computational fluid mechanics, computational geometry, computer game design, evolutionary biology, data mining, GPU computing, high-energy density physics, high-performance computing, hydrology, machine learning, material science, medical imaging, morphometrics, nano-materials, numerical analysis, partial differential equations, phylogenetics, polymers, population genetics, scientific visualization, subsurface environmental modeling, superconductivity, systems biology, and uncertainty quantification.

Unlike the faculty of a typical department whose training and research interests lie solely within a single discipline, the faculty of the DSC includes members trained in a variety of disciplines - biology, aeronautical engineering, chemical engineering, geological sciences, electrical engineering, material sciences, mathematics, nuclear engineering, and physics. In addition, the DSC has several affiliated faculty members in other departments at FSU and at other universities, companies, and laboratories that further enhance the educational and research experiences of students in the DSC's degree programs.

In addition to offering degree programs, the DSC contributes to FSU's educational mission by offering courses of interest to students in other departments, by offering tutorials and short courses, and by managing high-performance computational resources available to the campus community.

\section{Undergraduate Program Overview}
\url{https://www.sc.fsu.edu/education}
\begin{itemize}
    \item B.S. in Computational Science
    \item Undergraduate courses
\end{itemize}

\section{Graduate Program Overview}
\url{https://www.sc.fsu.edu/education}
\begin{itemize}
    \item Ph.D. in Computational Science
    \item M.S. in Computational Science
    \item M.S. in Data Science
    \item Graduate courses
    \item Application Instructions
    \item Graduate Handbook
    \item Financial Aid
\end{itemize}

\section{Courses by Semester}
% url: https://www.sc.fsu.edu/courses

\subsection{Fall 2021 Courses}
% url: https://www.sc.fsu.edu/courses/2021-fall-courses
\subsubsection*{ISC 1057 Computational Thinking (3 Units)}
\textbf{Schedule:} ONLINE \\
\textbf{Instructor:} TBD \\
This introductory course considers the question of how computers have come to imitate many kinds of human intelligence. The answer seems to involve our detecting patterns in nature, but also in being able to detect patterns in the very way we think. We will look at some popular computational methods that shape our lives, and try to explain the ideas that make them work. This course has been approved to satisfy the Liberal Studies Quantitative/Logical Thinking requirement.

\subsubsection*{ISC 2310 Introduction to Computational Thinking in Data Science with Python (3 Units)}
\textbf{Schedule:} ONLINE \\
\textbf{Instructor:} Janet Peterson \\
This course investigates strategies behind popular computational methods used in data science. In addition, many of the algorithms are implemented using the programming language Python. No prior programming experience is required so the course presents the basics of the Python language as well as how to leverage Python’s libraries to solve real-world problems in data science. \\
\textit{Prerequisite: MAC 1105 or equivalent.}

\subsubsection*{ISC 3222 Symbolic and Numerical Computations (3 Units)}
\textbf{Schedule:} M W F 8:00-8:55, 152 DSL \\
\textbf{Instructor:} Alan Lemmon \\
Introduces state-of-the-art software environments for solving scientific and engineering problems. Topics include solving simple problems in algebra and calculus; 2-D and 3-D graphics; non-linear function fitting and root finding; basic procedural programming; methods for finding numerical solutions to DE's with applications to chemistry, biology, physics, and engineering. \\
\textit{Prerequisite: MAC 2311.}

\subsubsection*{ISC 3313 Introduction to Scientific Computing with C++ (3 Units)}
\textbf{Schedule:} M W F 12:00-12:50, 152 DSL \\
\textbf{Instructor:} Ashley Gannon \\
This course introduces the student to the science of computations. Topics cover algorithms for standard problems in computational science, as well as the basics of an object-oriented programming language, to facilitate the student’s implementation of algorithms. The computer language will be C++. \\
\textit{Prerequisite: MAC 2311.}

\subsubsection*{ISC 4221C Discrete Algorithms for Science Applications (4 Units)}
\textbf{Schedule:} M W F 9:20-10:10, 152 DSL, M 3:05-5:35 (Lab) 152 DSL \\
\textbf{Instructor:} Bryan Quaife \\
This course offers stochastic algorithms, linear programming, optimization techniques, clustering and feature extraction presented in the context of science problems. The laboratory component includes algorithm implementation for simple problems in the sciences and applying visualization software for interpretation of results. \\
\textit{Prerequisite: MAC 2311.}

\subsubsection*{ISC 4223C Computational Methods for Discrete Problems (4 Units)}
\textbf{Schedule:} M W F 1:20-2:10, 152 DSL, F 3:05-5:35 (Lab) 152 DSL \\
\textbf{Instructor:} Anke Meyer-Baese \\
This course describes several discrete problems arising in science applications, a survey of methods and tools for solving the problems on computers, and detailed studies of methods and their use in science and engineering. The laboratory component illustrates the concepts learned in the context of science problems. \\
\textit{Prerequisites: MAS 3105, ISC 4304C.}

\subsubsection*{ISC 4232C Computational Methods for Continuous Problems (4 Units)}
\textbf{Schedule:} T R 9:45-11:00, 152 DSL, T 3:05-5:35 (Lab) 152 DSL \\
\textbf{Instructor:} Bryan Quaife \\
This course provides numerical discretization of differential equations and implementation for case studies drawn from several science areas. Finite difference, finite element, and spectral methods are introduced and standard software packages used. The lab component illustrates the concepts learned on a variety of application problems. \\
\textit{Prerequisites: MAS 3105, ISC 4304C.}

\subsubsection*{ISC 5228 / ISC 4933 Markov Chain Monte Carlo Simulations (3 Units)}
\textbf{Schedule:} T R 9:45-11:00, 422 DSL \\
\textbf{Instructor:} Sachin Shanbhag \\
Covered are statistical foundations of Monte Carlo (MC) and Markov Chain Monte Carlo (MCMC) simulations, applications of MC and MCMC simulations, which may range from social sciences to statistical physics models, statistical analysis of autocorrelated MCMC data, and parallel computing for MCMC simulations.

\subsubsection*{ISC 5317 / ISC 4933 Computational Evolutionary Biology (3 Units)}
\textbf{Schedule:} T R 1:20-2:35, 152 DSL \\
\textbf{Instructor:} Peter Beerli \\
This course presents computational methods for evolutionary inferences. Presentation includes the underlying models, the algorithms that analyze models, and the creation of software to carry out the analysis.

\subsubsection*{ISC 5305 Scientific Programming (3 Units)}
\textbf{Schedule:} T R 1:20-2:35, 499 DSL \\
\textbf{Instructor:} Gordon Erlebacher \\
This course uses the C language to present object-oriented coding, data structures, and parallel computing for scientific programming. Discussion of class hierarchies, pointers, function and operator overloading, and portability. Examples include computational grids and multidimensional arrays.

\subsubsection*{ISC 5315 Applied Computational Science I (4 Units)}
\textbf{Schedule:} T R 11:35-12:50, 152 DSL, R 3:05-5:35 (Lab) 152 DSL \\
\textbf{Instructor:} Chen Huang \\
Course provides students with high-performance computational tools necessary to investigate problems arising in science and engineering, with an emphasis on combining them to accomplish more complex tasks. A combination of course work and lab work provides the proper blend of theory and practice with problems culled from the applied sciences. Topics include numerical solutions to ODEs and PDEs, data handling, interpolation and approximation and visualization. \\
\textit{Prerequisites: ISC 5305, MAP 2302.}

\begin{center}
\textbf{Attachments}
\begin{tabular}{l l l}
\hline
\textbf{File} & \textbf{Description} & \textbf{File size} \\ \hline
2021fall-all.pdf & Advertisement & 2693 kB \\
DSC-Courses-2021Fall.xlsx & Source File & 13 kB \\ \hline
\end{tabular}
\end{center}
\newpage

\subsection{Spring 2021 Courses}
% url: https://www.sc.fsu.edu/courses/2021-spring-courses
\subsubsection*{ISC 1057 Computational Thinking (3 Units)}
\textbf{Schedule:} Online @ canvas.fsu.edu \\
\textbf{Instructor:} Sachin Shanbhag \\
This introductory course considers the question of how computers have come to imitate many kinds of human intelligence. The answer seems to involve our detecting patterns in nature, but also in being able to detect patterns in the very way we think. We will look at some popular computational methods that shape our lives, and try to explain the ideas that make them work. This course has been approved to satisfy the Liberal Studies Quantitative/Logical Thinking requirement.

\subsubsection*{ISC 2310 Introduction to Computational Thinking in Data Science with Python (3 Units)}
\textbf{Schedule:} Online @ canvas.fsu.edu \\
\textbf{Instructor:} Janet Peterson \\
This course investigates strategies behind popular computational methods used in data science. In addition, many of the algorithms are implemented using the programming language Python. No prior programming experience is required so the course presents the basics of the Python language as well as how to leverage Python’s libraries to solve real-world problems in data science. \\
\textit{Prerequisite: MAC 1105 or equivalent.}

\subsubsection*{ISC 4220C Continuous Algorithms for Science Applications (4 Units)}
\textbf{Schedule:} M W F 9:20-10:10, T 3:05-5:35 (Lab), REMOTE \\
\textbf{Instructor:} Sachin Shanbhag \\
Basic computational algorithms including interpolation, approximation, integration, differentiation, and linear systems solution presented in the context of science problems. The lab component includes algorithm implementation for simple problems in the sciences and applying visualization software for interpretation of results. \\
\textit{Corequisite: ISC 3222; Prerequisite: MAC 2312.}

\subsubsection*{ISC 4304C Programming for Science Applications (4 Units)}
\textbf{Schedule:} T R 9:45-11:00, M 3:05-5:35 (Lab), REMOTE \\
\textbf{Instructor:} Peter Beerli \\
Provides knowledge of a scripting language that serves as a front end to popular packages and frameworks, along with a compiled language such as C++. Topics include the practical use of an object-oriented scripting and compiled language for scientific programming applications. There is a laboratory component for the course; concepts learned are illustrated in several science applications. \\
\textit{Prerequisites: MAC 2311, COP 3014 or ISC 3313.}

\subsubsection*{ISC 4933/5227 Survey of Numerical Partial Differential Equations (3 Units)}
\textbf{Schedule:} T R 11:35-12:50, REMOTE \\
\textbf{Instructor:} Tomasz Plewa \\
This course provides an overview of the most common methods used for numerical partial differential equations. These include techniques such as finite differences, finite volumes, finite elements, discontinuous Galerkin, boundary integral methods, and pseudo-spectral methods.

\subsubsection*{ISC 4933/5238C Scientific Computing for Integral Equation Methods (3 Units)}
\textbf{Schedule:} M W F 1:20-2:10, REMOTE \\
\textbf{Instructor:} Bryan Quaife \\
This course covers key algorithms that are required when solving integral equations. \\
\textit{Prerequisites: MAD 3703 and MAP 4341; ISC 4232; or instructor permission.}

\subsubsection*{ISC 4933/5318 High-Performance Computing (3 Units)}
\textbf{Schedule:} M W F 10:40-11:30, REMOTE \\
\textbf{Instructor:} Xiaoqiang Wang \\
Introduces high-performance computing, which refers to the use of parallel supercomputers, computer clusters, as well as software and hardware to speed up computations. Students learn to write faster code that is highly optimized for modern multi-core processors and clusters, using modern software development tools and performance analyzers, specialized algorithms, parallelization strategies, and advanced parallel programming constructs. \\
\textit{Prerequisite: ISC 5305 or equivalent or instructor permission.}

\subsubsection*{ISC 4943 Practicum in Computational Science (3 Units)}
\textbf{Schedule:} T R 1:20-2:35, REMOTE \\
\textbf{Instructor:} Anke Meyer-Baese \\
This practicum allows students to work individually with a faculty member throughout the semester and meet with the course instructor periodically to provide progress reports. Written reports and an oral presentation of work are required. May be repeated to a maximum of six semester hours, with a maximum of only three semester hour credits allowed to be applied to the Computational Science degree.

\subsubsection*{ISC 5316 Applied Computational Science II (4 Units)}
\textbf{Schedule:} T R 9:30-10:45, R 3:30-6:00 (Lab), REMOTE \\
\textbf{Instructor:} Tomasz Plewa \\
Provides students with high performance computational tools to investigate problems in science and engineering with an emphasis on combining them to accomplish more complex tasks. Topics include numerical methods for partial differential equations, optimization, statistics, and Markov chain Monte Carlo methods. \\
\textit{Prerequisite: ISC 5315.}

\subsubsection*{ISC 5473 Introduction to Density Functional Theory (3 Units)}
\textbf{Schedule:} T R 1:20-2:35, REMOTE \\
\textbf{Instructor:} Chen Huang \\
For materials scientists, chemists, physicists, and applied mathematicians who want to know both the basic concept and certain advanced topics in density functional theory. Density functional theory is widely used in both industry and academia to simulate various properties of materials and molecules, such as electronic properties, crystal structures, and chemical reactions. We will learn how to solve realistic materials problems using density functional theory and the underlying theories.

\begin{center}
\textbf{Attachments}
\begin{tabular}{l l l}
\hline
\textbf{File} & \textbf{Description} & \textbf{File size} \\ \hline
2021spring-all.pdf & Advertisement & 2924 kB \\ \hline
\end{tabular}
\end{center}
\newpage

\subsection{Fall 2022 Courses}
% url: https://www.sc.fsu.edu/courses/2022-fall-courses
\subsubsection*{ISC 1057 Computational Thinking (3 Units)}
\textbf{Schedule:} ONLINE \\
\textbf{Instructor:} Young Hwan Kim \\
This introductory course considers the question of how computers have come to imitate many kinds of human intelligence. The answer seems to involve our detecting patterns in nature, but also in being able to detect patterns in the very way we think. We will look at some popular computational methods that shape our lives, and try to explain the ideas that make them work. This course has been approved to satisfy the Liberal Studies Quantitative/Logical Thinking requirement.

\subsubsection*{ISC 2310 Introduction to Computational Thinking in Data Science with Python (3 Units)}
\textbf{Schedule:} ONLINE \\
\textbf{Instructor:} TBD \\
This course investigates strategies behind popular computational methods used in data science. In addition, many of the algorithms are implemented using the programming language Python. No prior programming experience is required so the course presents the basics of the Python language as well as how to leverage Python’s libraries to solve real-world problems in data science. \\
\textit{Prerequisite: MAC 1105 or equivalent.}

\subsubsection*{ISC 3222 Symbolic and Numerical Computations (3 Units)}
\textbf{Schedule:} M W F 8:00-8:50, 152 DSL \\
\textbf{Instructor:} TBD \\
Introduces state-of-the-art software environments for solving scientific and engineering problems. Topics include solving simple problems in algebra and calculus; 2-D and 3-D graphics; non-linear function fitting and root finding; basic procedural programming; methods for finding numerical solutions to DE's with applications to chemistry, biology, physics, and engineering. \\
\textit{Prerequisite: MAC 2311.}

\subsubsection*{ISC 3313 Introduction to Scientific Computing with C++ (3 Units)}
\textbf{Schedule:} M W F 12:00-12:50, 152 DSL \\
\textbf{Instructor:} TBD \\
This course introduces the student to the science of computations. Topics cover algorithms for standard problems in computational science, as well as the basics of an object-oriented programming language, to facilitate the student’s implementation of algorithms. The computer language will be C++. \\
\textit{Prerequisite: MAC 2311.}

\subsubsection*{ISC 4221C Discrete Algorithms for Science Applications (4 Units)}
\textbf{Schedule:} T R 9:45-11:00, 152 DSL - T 3:05-5:35 (Lab), 152 DSL \\
\textbf{Instructor:} Chen Huang \\
This course offers stochastic algorithms, linear programming, optimization techniques, clustering and feature extraction presented in the context of science problems. The laboratory component includes algorithm implementation for simple problems in the sciences and applying visualization software for interpretation of results. \\
\textit{Prerequisite: MAC 2311.}

\subsubsection*{ISC 4223C Computational Methods for Discrete Problems (4 Units)}
\textbf{Schedule:} M W F 1:20-2:10, 152 DSL - F 3:05-5:35 (Lab), 152 DSL \\
\textbf{Instructor:} Tomasz Plewa \\
This course describes several discrete problems arising in science applications, a survey of methods and tools for solving the problems on computers, and detailed studies of methods and their use in science and engineering. The laboratory component illustrates the concepts learned in the context of science problems. \\
\textit{Prerequisites: MAS 3105, ISC 4304C.}

\subsubsection*{ISC 4232C Computational Methods for Continuous Problems (4 Units)}
\textbf{Schedule:} M W F 9:20-10:10, 152 DSL - M 3:05-5:35 (Lab), 152 DSL \\
\textbf{Instructor:} Bryan Quaife \\
This course provides numerical discretization of differential equations and implementation for case studies drawn from several science areas. Finite difference, finite element, and spectral methods are introduced and standard software packages used. The lab component illustrates the concepts learned on a variety of application problems. \\
\textit{Prerequisites: MAS 3105, ISC 4220, ISC 4304C.}

\subsubsection*{ISC 5305 Scientific Programming (3 Units)}
\textbf{Schedule:} T R 1:20-2:35, 499 DSL \\
\textbf{Instructor:} Gordon Erlebacher \\
This course uses the C language to present object-oriented coding, data structures, and parallel computing for scientific programming. Discussion of class hierarchies, pointers, function and operator overloading, and portability. Examples include computational grids and multidimensional arrays.

\subsubsection*{ISC 5315 Applied Computational Science I (4 Units)}
\textbf{Schedule:} T R 11:35-12:50, 152 DSL - R 3:05-5:35 (Lab), 152 DSL \\
\textbf{Instructor:} Chen Huang \\
Course provides students with high-performance computational tools necessary to investigate problems arising in science and engineering, with an emphasis on combining them to accomplish more complex tasks. A combination of course work and lab work provides the proper blend of theory and practice with problems culled from the applied sciences. Topics include numerical solutions to ODEs and PDEs, data handling, interpolation and approximation and visualization. \\
\textit{Prerequisites: ISC 5305, MAP 2302.}

\subsubsection*{ISC 5228 / ISC 4933 Markov Chain Monte Carlo Simulations (3 Units)}
\textbf{Schedule:} T R 9:45-11:00, 422 DSL \\
\textbf{Instructor:} Sachin Shanbhag \\
Covered are statistical foundations of Monte Carlo (MC) and Markov Chain Monte Carlo (MCMC) simulations, applications of MC and MCMC simulations, which may range from social sciences to statistical physics models, statistical analysis of autocorrelated MCMC data, and parallel computing for MCMC simulations.

\subsubsection*{ISC 4933 / ISC 5935 Understanding Covid (3 Units)}
\textbf{Schedule:} M W F 10:40-11:30, DSL 499 \\
\textbf{Instructor:} Alan Lemmon \\
In this course, students will explore the different elements that determine the outcome of infectious disease in humans, using COVID-19 as a case study. Starting with a very basic model that includes susceptible, infected, and recovered individuals, students will gradually incorporate factors that may affect the outcome of an outbreak, such as masking/quarantining, gain and loss of natural and vaccine-based immunity, and changing virulence/strains. After summarizing data already collected during the recent pandemic, students will inform the model to determine the conditions under which different outcomes may occur. Students will summarize their results graphically and present their findings. No prerequisites or programming experience required. The course is designed to be accessible to all students, regardless of background or major.

\subsubsection*{ISC 4931 Junior Seminar (1 Unit)}
\textbf{Schedule:} F 1:20 - 2:20, 422 DSL \\
\textbf{Instructor:} Alan Lemmon \\
Junior Seminar in Computational Science.

\subsubsection*{ISC 4932 Senior Seminar (1 Unit)}
\textbf{Schedule:} W 4:50 - 5:50, 416 DSL \\
\textbf{Instructor:} Tomasz Plewa \\
Senior Seminar in Computational Science.

\subsubsection*{ISC 5934 Graduate Seminar (1 Unit)}
\textbf{Schedule:} F 3:05-3:55, 499 DSL \\
\textbf{Instructor:} Xiaoqiang Wang \\
A series of lectures given by faculty on the research being conducted.

\subsubsection*{Colloquium (0 Units)}
\textbf{Schedule:} W 3:30 - 4:30, 499 DSL \\
\textbf{Instructor:} Peter Beerli \\
Weekly colloquium given by invited speakers to showcase research.
\newpage

\subsection{Spring 2022 Courses}
% url: https://www.sc.fsu.edu/courses/2022-spring-courses
\subsubsection*{ISC 1057 Computational Thinking (3 Units)}
\textbf{Schedule:} Online @ canvas.fsu.edu \\
\textbf{Instructor:} Student / TBD \\
This introductory course considers the question of how computers have come to imitate many kinds of human intelligence. The answer seems to involve our detecting patterns in nature, but also in being able to detect patterns in the very way we think. We will look at some popular computational methods that shape our lives, and try to explain the ideas that make them work. This course has been approved to satisfy the Liberal Studies Quantitative/Logical Thinking requirement.

\subsubsection*{ISC 2310 Introduction to Computational Thinking in Data Science with Python (3 Units)}
\textbf{Schedule:} Online @ canvas.fsu.edu \\
\textbf{Instructor:} Janet Peterson \\
This course investigates strategies behind popular computational methods used in data science. In addition, many of the algorithms are implemented using the programming language Python. No prior programming experience is required so the course presents the basics of the Python language as well as how to leverage Python’s libraries to solve real-world problems in data science. \\
\textit{Prerequisite: MAC 1105 or equivalent.}

\subsubsection*{ISC 3313 Introduction to Scientific Computing with C++ (3 Units)}
\textbf{Schedule:} M W F 1:20-2:10 (152 DSL) \\
\textbf{Instructor:} TBD \\
This course introduces the student to the science of computations. Topics cover algorithms for standard problems in computational science, as well as the basics of an object-oriented programming language, to facilitate the student’s implementation of algorithms. The computer language will be Java. \\
\textit{Prerequisite: MAC 2311.}

\subsubsection*{ISC 4220C Continuous Algorithms for Science Applications (4 Units)}
\textbf{Schedule:} M W F 9:20-10:10 (152 DSL), T 3:05-5:35 (Lab, 152 DSL) \\
\textbf{Instructor:} Sachin Shanbhag \\
Basic computational algorithms including interpolation, approximation, integration, differentiation, and linear systems solution presented in the context of science problems. The lab component includes algorithm implementation for simple problems in the sciences and applying visualization software for interpretation of results. \\
\textit{Corequisite: ISC 3222; Prerequisite: MAC 2312.}

\subsubsection*{ISC 4245C / CAP 5771 Data Mining (3 Units)}
\textbf{Schedule:} M W 12:00-1:15 (499 DSL) \\
\textbf{Instructor:} Anke Meyer-Baese \\
This course enables students to study concepts and techniques of data mining, including characterization and comparison, association rules mining, classification and prediction, cluster analysis, and mining complex types of data. Students also examine applications and trends in data mining. \\
\textit{Prerequisites: COP 3330, ISC 3222, ISC 3313 or ISC 4304, or instructor permission.}

\subsubsection*{ISC 4302 / ISC 5307 Scientific Visualization (3 Units)}
\textbf{Schedule:} M W F 12:20-1:10, 152 DSL \\
\textbf{Instructor:} Xiaoqiang Wang \\
This course enables students to study concepts and techniques of data mining, including characterization and comparison, association rules mining, classification and prediction, cluster analysis, and mining complex types of data. Students also examine applications and trends in data mining. \\
\textit{Prerequisites: COP 3330, ISC 3222, ISC 3313 or ISC 4304, or instructor permission.}

\subsubsection*{ISC 4304C Programming for Science Applications (4 Units)}
\textbf{Schedule:} T R 9:45-11:00 (152 DSL), F 3:05-5:35 (Lab, 152 DSL) \\
\textbf{Instructor:} Peter Beerli \\
Provides knowledge of a scripting language that serves as a front end to popular packages and frameworks, along with a compiled language such as C++. Topics include the practical use of an object-oriented scripting and compiled language for scientific programming applications. There is a laboratory component for the course; concepts learned are illustrated in several science applications. \\
\textit{Prerequisites: MAC 2312, COP 3014 or ISC 3313.}

\subsubsection*{ISC 4420 / ISC 5425 Introduction to Bioinformatics (3 Units)}
\textbf{Schedule:} M W F 1:20-2:10 (499 DSL) \\
\textbf{Instructor:} Alan Lemmon \\
Bioinformatics provides a quantitative framework for understanding how the genomic sequence and its variations affect the phenotype. Designed for biologists and biochemists seeking to improve quantitative data interpretation skills, and for mathematicians, computer scientists and other quantitative scientists seeking to learn more about computational biology. Lab exercises reinforce the classroom learning.

\subsubsection*{ISC 4933 / ISC 5227 Survey of Numerical Partial Differential Equations (3 Units)}
\textbf{Schedule:} T R 11:35-12:50 (152 DSL) \\
\textbf{Instructor:} Tomasz Plewa \\
This course provides an overview of the most common methods used for numerical partial differential equations. These include techniques such as finite differences, finite volumes, finite elements, discontinuous Galerkin, boundary integral methods, and pseudo-spectral methods.

\subsubsection*{ISC 4933 / ISC 5318 High-Performance Computing (3 Units)}
\textbf{Schedule:} T R 9:45-11:00 (499 DSL) \\
\textbf{Instructor:} Xiaoqiang Wang \\
Introduces high-performance computing, which refers to the use of parallel supercomputers, computer clusters, as well as software and hardware to speed up computations. Students learn to write faster code that is highly optimized for modern multi-core processors and clusters, using modern software development tools and performance analyzers, specialized algorithms, parallelization strategies, and advanced parallel programming constructs. \\
\textit{Prerequisite: ISC 5305 or equivalent or instructor permission.}

\subsubsection*{ISC 4943 Practicum in Computational Science (3 Units)}
\textbf{Schedule:} M W 10:40-11:55 (499 DSL) \\
\textbf{Instructor:} Anke Meyer-Baese \\
This practicum allows students to work individually with a faculty member throughout the semester and meet with the course instructor periodically to provide progress reports. Written reports and an oral presentation of work are required. May be repeated to a maximum of six semester hours, with a maximum of only three semester hour credits allowed to be applied to the Computational Science degree.

\subsubsection*{ISC 5316 Applied Computational Science II (4 Units)}
\textbf{Schedule:} T R 1:20-2:10 (422 DSL), R 3:05-5:35 (Lab, 152 DSL) \\
\textbf{Instructor:} Tomasz Plewa \\
Provides students with high performance computational tools to investigate problems in science and engineering with an emphasis on combining them to accomplish more complex tasks. Topics include numerical methods for partial differential equations, optimization, statistics, and Markov chain Monte Carlo methods. \\
\textit{Prerequisite: ISC 5315.}

\begin{center}
\textbf{Attachments}
\begin{tabular}{l l l}
\hline
\textbf{File} & \textbf{Description} & \textbf{File size} \\ \hline
spring2022.xlsx & & 13 kB \\
2022spring-all.pdf & & 3418 kB \\ \hline
\end{tabular}
\end{center}
\newpage

\subsection{Fall 2023 Courses}
% url: https://www.sc.fsu.edu/courses/2023-fall-courses
\subsubsection*{ISC 1057 Computational Thinking (3 Units)}
\textbf{Schedule:} ONLINE \\
\textbf{Instructor:} Young Hwan Kim \\
This introductory course considers the question of how computers have come to imitate many kinds of human intelligence. The answer seems to involve our detecting patterns in nature, but also in being able to detect patterns in the very way we think. We will look at some popular computational methods that shape our lives, and try to explain the ideas that make them work. This course has been approved to satisfy the Liberal Studies Quantitative/Logical Thinking requirement.

\subsubsection*{ISC 2310 Data Science with Python (3 Units)}
\textbf{Schedule:} ONLINE \\
\textbf{Instructor:} Pankaj Chouhan \\
This course investigates strategies behind popular computational methods used in data science. In addition, many of the algorithms are implemented using the programming language Python. No prior programming experience is required so the course presents the basics of the Python language as well as how to leverage Python’s libraries to solve real-world problems in data science. \\
\textit{Prerequisite: MAC 1105 or equivalent.}

\subsubsection*{ISC 3222 Symbolic and Numerical Computations (4206) (3 Units)}
\textbf{Schedule:} M W F 8:00-8:50, 152 DSL \\
\textbf{Instructor:} Alan Lemmon \\
Introduces state-of-the-art software environments for solving scientific and engineering problems. Topics include solving simple problems in algebra and calculus; 2-D and 3-D graphics; non-linear function fitting and root finding; basic procedural programming; methods for finding numerical solutions to DE's with applications to chemistry, biology, physics, and engineering. \\
\textit{Prerequisite: MAC 2311.}

\subsubsection*{ISC 3313 Introduction to Scientific Computing (C++) (4217) (3 Units)}
\textbf{Schedule:} M W F 12:00-12:50, 152 DSL \\
\textbf{Instructor:} TBA \\
This course introduces the student to the science of computations. Topics cover algorithms for standard problems in computational science, as well as the basics of an object-oriented programming language, to facilitate the student’s implementation of algorithms. The computer language will be C++. \\
\textit{Prerequisite: MAC 2311.}

\subsubsection*{ISC 4221C Discrete Algorithms for Science Applications (4208) (4 Units)}
\textbf{Schedule:} T R 9:45-11:00, 152 DSL - T 3:05-5:35 (Lab), 152 DSL \\
\textbf{Instructor:} Olmo Zavala-Romero \\
This course offers stochastic algorithms, linear programming, optimization techniques, clustering and feature extraction presented in the context of science problems. The laboratory component includes algorithm implementation for simple problems in the sciences and applying visualization software for interpretation of results. \\
\textit{Prerequisite: MAC 2311.}

\subsubsection*{ISC 4223C Computational Methods for Discrete Problems (4 Units)}
\textbf{Schedule:} T R 1:20-2:35, 152 DSL - F 3:05-5:35 (Lab), 152 DSL \\
\textbf{Instructor:} Anke Meyer-Baese \\
This course describes several discrete problems arising in science applications, a survey of methods and tools for solving the problems on computers, and detailed studies of methods and their use in science and engineering. The laboratory component illustrates the concepts learned in the context of science problems. \\
\textit{Prerequisites: MAS 3105, ISC 4304C.}

\subsubsection*{ISC 4232C Computational Methods for Continuous Problems (4210) (4 Units)}
\textbf{Schedule:} M W F 9:20-10:10, 152 DSL - M 3:05-5:35 (Lab), 152 DSL \\
\textbf{Instructor:} Bryan Quaife \\
This course provides numerical discretization of differential equations and implementation for case studies drawn from several science areas. Finite difference, finite element, and spectral methods are introduced and standard software packages used. The lab component illustrates the concepts learned on a variety of application problems. \\
\textit{Prerequisites: MAS 3105, ISC 4220, ISC 4304C.}

\subsubsection*{ISC 5305 Scientific Programming (4203) (3 Units)}
\textbf{Schedule:} T R 1:20-2:35, 499 DSL \\
\textbf{Instructor:} Gordon Erlebacher \\
This course uses the C language to present object-oriented coding, data structures, and parallel computing for scientific programming. Discussion of class hierarchies, pointers, function and operator overloading, and portability. Examples include computational grids and multidimensional arrays.

\subsubsection*{ISC 5315 Applied Computational Science I (4 Units)}
\textbf{Schedule:} T R 11:35-12:50, 152 DSL - R 3:05-5:35 (Lab), 152 DSL \\
\textbf{Instructor:} Chen Huang \\
Course provides students with high-performance computational tools necessary to investigate problems arising in science and engineering, with an emphasis on combining them to accomplish more complex tasks. A combination of course work and lab work provides the proper blend of theory and practice with problems culled from the applied sciences. Topics include numerical solutions to ODEs and PDEs, data handling, interpolation and approximation and visualization. \\
\textit{Prerequisites: ISC 5305, MAP 2302.}

\subsubsection*{ISC 5228 / ISC 4933 Monte Carlo Method (4234) (3 Units)}
\textbf{Schedule:} T R 9:45-11:00, 422 DSL \\
\textbf{Instructor:} Sachin Shanbhag \\
Covered are statistical foundations of Monte Carlo (MC) and Markov Chain Monte Carlo (MCMC) simulations, applications of MC and MCMC simulations, which may range from social sciences to statistical physics models, statistical analysis of autocorrelated MCMC data, and parallel computing for MCMC simulations.

\subsubsection*{ISC 4933 / ISC 5935 Science Professional Development (3 Units)}
\textbf{Schedule:} M W F 9:00-9:50, DSL 499 \\
\textbf{Instructor:} Alan Lemmon \\
The course would cover topics such as improving communication in science (posters, talks, manuscripts, grants, etc), as well as some work on primary literature, and the publication process. Finally there would be a component on applying for grad school or jobs.

\subsubsection*{ISC 4933 / ISC 5935 AI Methods and Applications (3 Units)}
\textbf{Schedule:} T R 9:45-11:00, DSL 499 \\
\textbf{Instructor:} Anke Meyer-Baese \\
AI is a fast-moving discipline which is impacting many areas of our daily life. In this course, students will get an introduction into deep learning and reinforcement learning and will also learn novel concepts such as graph neural networks, encoders and attention networks. Applications from several areas such as medical imaging, weather forecast and finance are introduced.

\subsubsection*{ISC 4931 Junior Seminar (1 Unit)}
\textbf{Schedule:} F 1:20 - 2:20, 422 DSL \\
\textbf{Instructor:} Alan Lemmon \\
Junior Seminar in Computational Science.

\subsubsection*{ISC 4932 Senior Seminar (1 Unit)}
\textbf{Schedule:} W 4:50 - 5:50, 416 DSL \\
\textbf{Instructor:} Tomasz Plewa \\
Senior Seminar in Computational Science.

\subsubsection*{ISC 5934 Graduate Seminar (1 Unit)}
\textbf{Schedule:} F 3:05-3:55, 499 DSL \\
\textbf{Instructor:} Xiaoqiang Wang \\
A series of lectures given by faculty on the research being conducted.

\subsubsection*{Colloquium (0 Units)}
\textbf{Schedule:} W 3:30 - 4:30, 499 DSL \\
\textbf{Instructor:} Olmo Zavala-Romero \\
Weekly colloquium given by invited speakers to showcase research.
\newpage

\subsection{Spring 2023 Courses}
% url: https://www.sc.fsu.edu/courses/2023-spring-courses
\subsubsection*{ISC 1057 Computational Thinking (3 Units)}
\textbf{Schedule:} Online @ canvas.fsu.edu \\
\textbf{Instructor:} Young Hwan Kim \\
This introductory course considers the question of how computers have come to imitate many kinds of human intelligence. The answer seems to involve our detecting patterns in nature, but also in being able to detect patterns in the very way we think. We will look at some popular computational methods that shape our lives, and try to explain the ideas that make them work. This course has been approved to satisfy the Liberal Studies Quantitative/Logical Thinking requirement.

\subsubsection*{ISC 2310 Data Science with Python (3 Units)}
\textbf{Schedule:} Online @ canvas.fsu.edu \\
\textbf{Instructor:} Jingze Zhang \\
This course investigates strategies behind popular computational methods used in data science. In addition, many of the algorithms are implemented using the programming language Python. No prior programming experience is required so the course presents the basics of the Python language as well as how to leverage Python’s libraries to solve real-world problems in data science. \\
\textit{Prerequisite: MAC 1105 or equivalent.}

\subsubsection*{ISC 3313 Introduction to Scientific Computing (with C++) (3 Units)}
\textbf{Schedule:} M W F 1:20-2:10 (152 DSL) \\
\textbf{Instructor:} TBD \\
This course introduces the student to the science of computations. Topics cover algorithms for standard problems in computational science, as well as the basics of an object-oriented programming language, to facilitate the student’s implementation of algorithms. Satisfies FSU Computer Competency requirement. \\
\textit{Prerequisites: MAC 2311 (Calculus I).}

\subsubsection*{ISC 4220C Continuous Algorithms for Science Applications (4 Units)}
\textbf{Schedule:} M W F 9:20-10:10 (152 DSL), T 3:05-5:35 (Lab, 152 DSL) \\
\textbf{Instructor:} Sachin Shanbhag \\
Basic computational algorithms including interpolation, approximation, integration, differentiation, and linear systems solution presented in the context of science problems. The lab component includes algorithm implementation for simple problems in the sciences and applying visualization software for interpretation of results. \\
\textit{Prerequisite: MAC 2312.}

\subsubsection*{ISC 4933 / ISC 5318 High-Performance Computing (3 Units)}
\textbf{Schedule:} T R 9:45-11:00 (499 DSL) \\
\textbf{Instructor:} Xiaoqiang Wang \\
Introduces high-performance computing, the use of parallel supercomputers, computer clusters, and software and hardware, to speed up computations. Students learn to write faster code that is optimized for modern multi-core processors and clusters, using modern software-development tools and performance analyzers, specialized algorithms, parallelization strategies, and advanced parallel programming constructs. \\
\textit{Prerequisite: ISC 5305.}

\subsubsection*{ISC 4304C Programming for Science Applications (4 Units)}
\textbf{Schedule:} T R 9:45-11:00 (152 DSL), F 3:05-5:35 (Lab, 152 DSL) \\
\textbf{Instructor:} Peter Beerli \\
Provides knowledge of a scripting language that serves as a front end to popular packages and frameworks, along with a compiled language such as C++. Topics include the practical use of an object-oriented scripting and compiled language for scientific programming applications. There is a laboratory component for the course; concepts learned are illustrated in several science applications. \\
\textit{Prerequisites: MAC 2312, COP 3014 or ISC 3313.}

\subsubsection*{ISC 4943 Practicum in Computational Science (3 Units)}
\textbf{Schedule:} T R 3:05-4:20 (499 DSL) \\
\textbf{Instructor:} Gordon Erlebacher \\
This practicum allows students to work individually with a faculty member throughout the semester and meet with the course instructor periodically to provide progress reports. Written reports and an oral presentation of work are required. May be repeated to a maximum of six semester hours, with a maximum of only three semester hour credits allowed to be applied to the Computational Science degree.

\subsubsection*{ISC 5314 Verification and Validation in Computational Science (3 Units)}
\textbf{Schedule:} T R 1:20-2:35 (152 DSL) \\
\textbf{Instructor:} Tomasz Plewa \\
Students learn basic terminology, are exposed to procedures and practical methods used in software implementation validation and in solution verification, employ exact and manufactured solutions, and explore elements of software quality assurance. Introduces essential data analysis techniques and reviews software development and maintenance tools. Aspects of code variation, including validation hierarchy, validation benchmarks, uncertainty quantification and simulation code predictive capabilities are illustrated. \\
\textit{Prerequisite: MAC 2312.}

\subsubsection*{ISC 5316 Applied Computational Science II (4 Units)}
\textbf{Schedule:} T R 8:00-9:15 (499 DSL), R 3:05-5:35 (Lab, 152 DSL) \\
\textbf{Instructor:} Tomasz Plewa \\
Provides students with high performance computational tools to investigate problems in science and engineering with an emphasis on combining them to accomplish more complex tasks. Topics include mesh generation, stochastic methods, basic parallel algorithms and programming, numerical optimization, and nonlinear solvers. \\
\textit{Prerequisite: ISC 5315.}

\subsubsection*{ISC 4245C / CAP 5771 Data Mining (3 Units)}
\textbf{Schedule:} T R 11:35-12:50 (499 DSL) \\
\textbf{Instructor:} Gordon Erlebacher \\
This course enables students to study concepts and techniques of data mining, including characterization and comparison, association rules mining, classification and prediction, cluster analysis, and mining complex types of data. Students also examine applications and trends in data mining. \\
\textit{Prerequisites: COP 3330, ISC 3222, ISC 3313 or ISC 4304, or instructor permission.}

\subsubsection*{ISC 4302 / ISC 5307 Scientific Visualization (3 Units)}
\textbf{Schedule:} T R 8:00-9:15 (152 DSL) \\
\textbf{Instructor:} Xiaoqiang Wang \\
This course covers the theory and practice of scientific visualization. Students learn how to use state-of-the-art visualization toolkits, create their own visualization tools, represent both 2-D and 3-D data sets, and evaluate the effectiveness of their visualizations. \\
\textit{Undergrad Prerequisites: COP 3330, ISC 3222, ISC 3313 or ISC 4304, or instructor permission. Graduate Prerequisites: ISC 5305.}

\subsubsection*{ISC 4933 / ISC 5935 Data Science Meets Health Science (3 Units)}
\textbf{Schedule:} M W F 10:40-11:30 (152 DSL) \\
\textbf{Instructor:} Olmo Zavala Romero \\
This course will focus on the applied data science pipeline of data acquisition, data processing and integration, data modeling and analysis, and validation and delivery, commonly used in the Health industry. Topics include data normalization, scientific visualization, multivariate regression, and Artificial Neural Networks (dense, convolutional, recurrent, and adversarial). The examples and projects of this course contain 1D to 4D health data of electrocardiogram sequences, X-ray, Magnetic resonance imaging (MRI), and functional MRI images.

\subsubsection*{ISC 4933 / ISC 5935 Computational Probabilistic Modeling (3 Units)}
\textbf{Schedule:} TR 11:35-12:50 (152 DSL) \\
\textbf{Instructor:} Nick Dexter \\
In this course, students are introduced to probabilistic programming and modeling for modern data science and machine learning applications. Algorithms for predictive inference are covered from a theoretical and practical viewpoint with an emphasis on implementation in Python. Topics include an introduction to probability and learning theory, graph-based methods, machine learning with neural networks, dimensionality reduction, and algorithms for big data.

\subsubsection*{ISC 4931 Junior Seminar (1 Unit)}
\textbf{Schedule:} F 12:00-12:50 (422 DSL) \\
\textbf{Instructor:} Alan Lemmon \\
Junior Seminar in Computational Science.

\subsubsection*{ISC 5934 Graduate Seminar (1 Unit)}
\textbf{Schedule:} F 3:05-3:55 (499 DSL) \\
\textbf{Instructor:} Sachin Shanbhag \\
A series of lectures given by faculty on the research being conducted.

\subsubsection*{ISC 5939 / GFD???? Advance Grad Seminar in Fire Dynamics (1 Unit)}
\textbf{Schedule:} T 12:00 - 3:00 (Keen 118) \\
\textbf{Instructor:} Kevin Speer \\
TBA

\subsubsection*{Course \# TBD Colloquium (0 Units)}
\textbf{Schedule:} W 3:30 - 4:30 (499 DSL) \\
\textbf{Instructor:} \textit{[Email Protected]} \\
Weekly colloquium given by invited speakers to showcase research.

\subsubsection*{GFD 5500L Fire Dynamics Field School (3-4 Units)}
\textbf{Schedule:} (Keen 118) \\
\textbf{Instructor:} Kevin Speer \\
TBA

\subsubsection*{GFD 6935 Fire Dynamics Seminar (1 Unit)}
\textbf{Schedule:} T 12:00 - 3:00 (Keen 118) \\
\textbf{Instructor:} Kevin Speer \\
TBA
\newpage

\subsection{Spring 2024 Courses}
% url: https://www.sc.fsu.edu/courses/2024-spring-courses
\subsubsection*{ISC 1057 Computational Thinking (3 Units)}
\textbf{Schedule:} ONLINE \\
\textbf{Instructor:} Young Hwan Kim \\
ISC 1057. Computational Thinking (3). This course introduces students to the process of creating a representation of a task so that it can be performed by a computer. The course investigates strategies behind popular computational methods that are shaping our daily lives and our future. Students practice logical thinking by applying versions of these computational methods to problems in science and society.

\subsubsection*{ISC 2310 Data Science with Python (3 Units)}
\textbf{Schedule:} ONLINE \\
\textbf{Instructor:} Pankaj Chouhan \\
ISC 2310. Introduction to Computational Thinking in Data Science with Python (3). Prerequisite: MAC 1105 or equivalent. This course investigates strategies behind popular computational methods used in data science. In addition, many of the algorithms are implemented using the programming language Python. No prior programming experience is required so the course presents the basics of the Python language as well as how to leverage Python's libraries to solve real-world problems in data science.

\subsubsection*{ISC 3313 Introduction to Scientific Computing (C++) (4217) (3 Units)}
\textbf{Schedule:} M W F 1:20-2:10, 152 DSL \\
\textbf{Instructor:} Sanjeeb Poudel \\
ISC 3313. Introduction to Scientific Computing (3). Prerequisites: MAC 2311 or instructor permission. This course introduces the student to the science of computation. Topics cover algorithms for standard problems in computational science, as well as the basics of an object-oriented programming language, to facilitate the students' implementation of algorithms.

\subsubsection*{ISC 4220C Continuous Algorithms for Science Applications (4 Units)}
\textbf{Schedule:} M W F 9:20-10:10, 152 DSL - T 3:05-5:35 (lab), 152 DSL \\
\textbf{Instructor:} Sachin Shanbhag \\
ISC 4220C. Continuous Algorithms for Science Applications (4). Prerequisite: MAC 2312. This course provides basic computational algorithms, including interpolation, approximation, integration, differentiation, and linear systems solution presented in the context of science problems. The laboratory component includes algorithm implementation for simple problems in the sciences and applying visualization software for interpretation of results.

\subsubsection*{ISC 4933/5318 High-Performance Computing (3 Units)}
\textbf{Schedule:} T R 9:45-11:00, 499 DSL \\
\textbf{Instructor:} Xiaoqiang Wang \\
ISC 5318. High-Performance Computing (3). Prerequisites: ISC 5305 or equivalent or instructor permission. This course introduces high-performance computing, term which refers to the use of parallel supercomputers, computer clusters, as well as software and hardware in order to speed up computations. Students learn to write faster code that is highly optimized for modern multi-core processors and clusters, using modern software-development tools and performance analyzers, specialized algorithms, parallelization strategies, and advanced parallel programming constructs.

\subsubsection*{ISC 4304C Programming for Science Applications (3 Units)}
\textbf{Schedule:} T R 9:45-11:00, F 3:05-5:35 (Lab), 152 DSL \\
\textbf{Instructor:} Nicholas Dexter \\
ISC 4304C. Programming for Science Applications (4). Prerequisite: MAC 2311. This course provides knowledge of a scripting language that serves as a front-end to popular packages and frameworks, along with a compiled language such as (C++). Students study and practice scientific programming with the scripting language and practice how to interface it with a more traditional programming language to improve the speed of the programs developed in the course. In the laboratory component of this course, students apply the concepts learned in class. Students analyze large data sets by translating from mathematical expressions and algorithms to working computer code that is then used to visualize and summarize the results.

\subsubsection*{ISC 5314 Verification and Validation in Computational Science (3 Units)}
\textbf{Schedule:} T R 1:20-2:35, 152 DSL \\
\textbf{Instructor:} Tomasz Plewa \\
ISC 5314. Verification and Validation in Computational Science (3). Prerequisites: MAC 2312, MAS 3105, or ISC 5315; or instructor permission. This course covers the theory and practice of verification and validation in computational sciences. Students learn basic terminology, are exposed to procedures and practical methods used in software implementation validation and in solution verification, employ exact and manufactured solutions, and explore elements of software quality assurance. The course introduces essential data analysis techniques and reviews software development and maintenance tools. Examples from physical sciences and engineering are used to illustrate aspects of code variation, including validation hierarchy, validation benchmarks, uncertainty quantification and simulation code predictive capabilities. The computational laboratory is an essential part of this course.

\subsubsection*{ISC 5316 Applied Computational Science II (4 Units)}
\textbf{Schedule:} T R 8:00-9:15, 499 DSL - R 3:05-5:35 (Lab), 152 DSL \\
\textbf{Instructor:} Tomasz Plewa \\
ISC 5316. Applied Computational Science II (4). Prerequisite: ISC 5315 or instructor permission. This course provides students with high-performance computational tools necessary to investigate problems arising in science and engineering, with an emphasis on combining them to accomplish more complex tasks. A combination of course work and lab work provides the proper blend of theory and practice with problems culled from the applied sciences. Topics include mesh generation, stochastic methods, basic parallel algorithms and programming, numerical optimization, and nonlinear solvers.

\subsubsection*{ISC 4933/ISC 5935 Computational Probabilistic Modeling (3 Units)}
\textbf{Schedule:} T R 11:35-12:50, 152 DSL \\
\textbf{Instructor:} Nicholas Dexter \\
ISC 4933/ISC 5935. Computational Probabilistic Modeling (3). Prerequisites: MAC 2312 - Calculus II, MAS 3105 – Applied Linear Algebra, and STA 4442/5440: Introduction To Probability or STA 4321/5323: Introduction to Mathematical Statistics, or the permission of the instructor. In this course, students are introduced to probabilistic programming and modeling for modern data science and machine learning applications. Algorithms for predictive inference are covered from a theoretical and practical viewpoint with an emphasis on implementation in Python. Topics include an introduction to probability and learning theory, graph-based methods, machine learning with neural networks, dimensionality reduction, and algorithms for big data.

\subsubsection*{ISC 4302/5307 Scientific Visualization (3 Units)}
\textbf{Schedule:} T R 8:00-9:15, 152 DSL \\
\textbf{Instructor:} Xiaoqiang Wang \\
ISC 4302. Scientific Visualization (3). Prerequisites: MAC 1105 and MAC 2312. This course is an introduction to scientific visualization for large-scale computation and experimental data that covers the visualization methods and techniques, visualization results analysis and evaluation, and visualization practice. It teaches students the techniques for creating compelling visual representations of 2D and 3D scientific data sets. The basic concepts, data structures, and algorithms in scientific visualization are presented and applied using datasets from different disciplines. Classic visualization techniques for scalar, vector, and tensor data such as marching cubes, ray casting, splatting, streamline, and line integral convolution and more, are introduced along with popular visualization software.

\subsubsection*{ISC 4933/5935 Data Science Meets Health Science (3 Units)}
\textbf{Schedule:} M W F 10:40-11:30, 152 DSL \\
\textbf{Instructor:} Olmo Zavala-Romero \\
This course will focus on the applied data science pipeline of data acquisition, data processing and integration, data modeling and analysis, and validation and delivery, commonly used in the Health industry. Topics include data normalization, scientific visualization, multivariate regression, and Artificial Neural Networks (dense, convolutional, recurrent, and adversarial). The examples and projects of this course contain 1D to 4D health data of electrocardiogram sequences, X-ray, Magnetic resonance imaging (MRI), and functional MRI images.

\subsubsection*{ISC 5935 Atomistic Modeling of Molecules and Materials (3 Units)}
\textbf{Schedule:} M W F 12:00-12:50, 152 DSL \\
\textbf{Instructor:} Chen Huang \\
The course is designed for students who are interested in atomistic simulations of molecules and materials. Two popular methods will be introduced: density functional theory (DFT) and molecular dynamics. DFT has become the workhorse in industry and academia for calculating various properties of materials and molecules, such as electronic properties, crystal structures, and chemical reaction energies. We will learn both the theoretical and numerical aspects of DFT. Molecular dynamics are invaluable for understanding the dynamical processes of materials at the atomic scale. We will introduce the theories underlying molecular dynamics simulations and learn how to calculate various properties of materials using molecular dynamics. Popular software in these two fields will be introduced.

\subsubsection*{ISC 4933/5935 Computational Aspects of Data Assimilation (3 Units)}
\textbf{Schedule:} T R 9:45-11:00, 422 DSL \\
\textbf{Instructor:} Hristo Chipilski \\
Data assimilation methods combine numerical models and observations to arrive at the best possible representation of a physical system. This course aims to build a robust theoretical foundation in the subject and explore some of the computational challenges in large scientific and engineering applications. Students will gain hands-on experience by implementing their own algorithms and will complete a final project on a preferred research topic. Prerequisites: Applied Statistics for Engineers and Scientists (STA 3032), Applied Linear Algebra I/II (MAS 3105/MAS 4106) and Programming for Scientific Applications (ISC 4304) or Instructor Permission Required.

\subsubsection*{ISC 4933/5935 Computational Methods in Fire Science (3 Units)}
\textbf{Schedule:} M W F 10:40-11:30, 499 DSL \\
\textbf{Instructor:} Bryan Quaife \\
Have you ever wondered why fires spread and grow in size so quickly or how smoke plumes can travel thousands of kilometers? These behaviors are governed by fuel properties, atmospheric conditions, topography, and more. This course introduces physics-based and data-driven models in fire science, and investigates the sensitivity and uncertainty of these models. We will discuss computational tools including cellular automata, level set methods, and data-driven methods for discovering equations from measured data. Finally, we will explore techniques for analyzing both simulated and measured fire data.

\subsubsection*{CAP 5771/4245C Data Mining (3 Units)}
\textbf{Schedule:} T R 11:35 - 12:50, 499 DSL \\
\textbf{Instructor:} Gordon Erlebacher \\
ISC 4245C. Data Mining (3). Prerequisite: COP 3330, ISC 3222, ISC 3313 or ISC 4304; or instructor permission. In this course, students study concepts and techniques of data mining, including characterization and comparison, association rules mining, classification and prediction, cluster analysis, and mining complex types of data. Students also examine applications and trends in data mining.

\subsubsection*{GFD 4934/5936 Advanced Topics in Fire Dynamics (1 Unit)}
\textbf{Schedule:} 118 KEEN \\
\textbf{Instructor:} Kevin Speer \\
Please contact \textit{[Email Protected]} for details.

\subsubsection*{ISC 5939/GFD 6935 Advance Grad Seminar in Fire Dynamics (1 Unit)}
\textbf{Schedule:} T 12:00 - 3:00, 118 KEEN \\
\textbf{Instructor:} Kevin Speer \\
Please contact \textit{[Email Protected]} for details.

\subsubsection*{GFD 5500L Fire Dynamics Field School Research and Operations in Prescribed Fire (3-4 Units)}
\textbf{Schedule:} 118 KEEN \\
\textbf{Instructor:} Kevin Speer \\
Please contact \textit{[Email Protected]} for details.

\subsubsection*{ISC 4931 Junior Seminar (1 Unit)}
\textbf{Schedule:} F 12:00-12:50, 422 DSL \\
\textbf{Instructor:} Alan Lemmon \\
ISC 4931r. Junior Seminar in Scientific Computing (1–2). (S/U grade only.) Prerequisite: Junior standing (sixty plus hours). This is a special topics course in computational science. May be repeated two times to a maximum of four semester hours, with a maximum of only two semester hours credit allowed to be applied to the Computational Science degree.

\subsubsection*{ISC 5934 Graduate Seminar (1 Unit)}
\textbf{Schedule:} F 3:05-3:55, 499 DSL \\
\textbf{Instructor:} Xiaoqiang Wang \\
A series of lectures given by faculty on the research being conducted.

\subsubsection*{Colloquium (0 Units)}
\textbf{Schedule:} W 3:30 - 4:30, 499 DSL \\
\textbf{Instructor:} Olmo Zavala-Romero \\
Weekly colloquium given by invited speakers to showcase research.

\begin{center}
\textbf{Attachments}
\begin{tabular}{l l l}
\hline
\textbf{File} & \textbf{Description} & \textbf{File size} \\ \hline
Print\_2024\_Spring\_Courses.pdf & Printable Course Listing & 211 kB \\ \hline
\end{tabular}
\end{center}
\newpage

\subsection{Fall 2024 Courses}
% url: https://www.sc.fsu.edu/courses/2024-fall-courses
\textit{No course listings provided.}

\subsection{Spring 2025 Courses}
% url: https://www.sc.fsu.edu/courses/2025-spring-courses
\textit{No course listings provided.}

\subsection{Fall 2025 Courses}
% url: https://www.sc.fsu.edu/courses/2025-fall-courses
\textit{No course listings provided.}

\subsection{Archive}
% url: https://www.sc.fsu.edu/courses/archive
\subsubsection*{Published Course Lists}
\begin{itemize}
    \item Fall 2025 Courses (Published: May 04 2025)
    \item Spring 2025 Courses (Published: September 18 2024)
    \item Fall 2024 Courses (Published: August 26 2024)
    \item Spring 2024 Courses (Published: October 20 2023)
    \item Fall 2023 Courses (Published: May 26 2023)
    \item Spring 2023 Courses (Published: September 20 2022)
    \item Fall 2022 Courses (Published: March 21 2022)
    \item Spring 2022 Courses (Published: October 22 2021)
    \item Fall 2021 Courses (Published: May 05 2021)
    \item Spring 2021 Courses (Published: October 01 2020)
\end{itemize}
\newpage

\section{Undergraduate Course Catalog}
% url: https://www.sc.fsu.edu/undergraduate/courses
We have developed innovative courses to support our degree programs in computational science. These classes generally have an ISC course prefix. In addition, department faculty teach courses in computational science listed with different departments across campus.
\textit{Note: Not all courses are offered each semester. Check the FSU Registrar's Class Search Snapshots for availability.}

\subsection*{Elective Courses for non-majors}
\begin{itemize}
    \item \textbf{ISC 1057 - Computational Thinking (3)} \\ \textit{Note: Satisfies Quantitative and Logical Thinking requirement.}
    \item \textbf{ISC 2310 - Introduction to Computational Thinking in Data Science with Python (3)}
\end{itemize}

\subsection*{Core Courses and Seminar Classes}
\begin{itemize}
    \item \textbf{ISC 3222 - Symbolic and Numerical Computations (3)}
    \item \textbf{ISC 4220 - Continuous Algorithms for Science Applications (4)}
    \item \textbf{ISC 4221 - Discrete Algorithms for Science Applications (4)}
    \item \textbf{ISC 4223 - Computational Methods for Discrete Problems (4)}
    \item \textbf{ISC 4232 - Computational Methods for Continuous Problems (4)}
    \item \textbf{ISC 4304 - Programming for Scientific Applications (4)}
    \item \textbf{ISC 4931 - Junior Seminar in Computational Science (1)}
    \item \textbf{ISC 4932 - Senior Seminar in Computational Science (1)}
    \item \textbf{ISC 4943 - Practicum in Computational Science (varies)}
\end{itemize}

\subsection*{Collateral Courses}
\begin{itemize}
    \item \textbf{ISC 3313 - Introduction to Scientific Computing (3)} \\ \textit{Note: Satisfies Computer Skills Competency requirement.}
\end{itemize}

\subsection*{Elective Courses}
\begin{itemize}
    \item \textbf{DIG 3725 - Introduction to Game and Simulator Design (3)}
    \item \textbf{ISC 4245C - Data Mining (3)}
    \item \textbf{ISC 4302 - Scientific Visualization (3)}
    \item \textbf{ISC 4304C - Programming for Scientific Applications (3)}
    \item \textbf{ISC 4420 - Introduction to Bioinformatics (3)}
    \item \textbf{ISC 4933 - Computational Evolutionary Biology (3)}
    \item \textbf{ISC 4933 - Computational Space Physics (3)}
    \item \textbf{ISC 4933 - Data Science Meets Health Science (3)}
    \item \textbf{ISC 4933 - Genome Sequencing and Analysis (3)}
    \item \textbf{ISC 4933 - Geometric Morphometrics (3)}
    \item \textbf{ISC 4933 - Inferences in Conservation Genetics (3)}
    \item \textbf{ISC 4933 - Integral Equation Methods (3)}
    \item \textbf{ISC 4933 - Selected Topics In Computational Science (varies)}
    \item \textbf{ISC 4933 - Verification and Validation in Computational Science (3)}
    \item \textbf{ISC 4933 - Survey of Numerical Partial Differential Equations (3)}
    \item \textbf{ISC 4971 - Honors in the Major Program (3)}
\end{itemize}

\subsection*{Special Topics}
\begin{itemize}
    \item \textbf{ISC 4933 - Understanding Covid (3)}
\end{itemize}
\newpage

\section{Graduate and Doctoral Course Catalog}
% url: https://www.sc.fsu.edu/graduate/courses
We have developed innovative courses to support our degree programs in computational science. These classes generally have an ISC course prefix. In addition, department faculty teach courses in computational science listed with different departments across campus.
\textit{Note: Not all courses are offered each semester. Check the FSU Registrar's Class Search Snapshots for availability.}

\subsection*{Required Graduate Core Courses}
\begin{itemize}
    \item ISC 5305 - Introduction to Scientific Programming
    \item ISC 5315 - Applied Computational Science I
    \item ISC 5318 - High-Performance Computing
    \item ISC 5934 - Introductory Seminar on Research In Computational Science
    \item ISC 5939 - Advanced Graduate Student Seminar In Computational Science
\end{itemize}

\subsection*{Required Doctoral and Graduate Elective Courses}
\begin{itemize}
    \item ISC 5316 - Applied Computational Science II
\end{itemize}

\subsection*{Elective Courses}
\begin{itemize}
    \item BSC 5936 - Conservation Genetics
    \item CAP 5771 - Data Mining
    \item ISC 5425 - Introduction to Bioinformatics
    \item ISC 5225 - Molecular Dynamics: Algorithms and Applications
    \item ISC 5227 - Survey of Numerical Partial Differential Equations
    \item ISC 5228 - Markov Chain Monte Carlo Simulations
    \item ISC 5229 - Multiscale Modeling of Materials
    \item ISC 5238 - Integral Equation Methods
    \item ISC 5306 - Programming Skills for Computation Biology and Bioinformatics
    \item ISC 5307 - Scientific Visualization
    \item ISC 5314 - Verification and Validation in Computational Science
    \item ISC 5317 - Computational Evolutionary Biology
    \item ISC 5319 - Advanced Topics in High-Performance Computing
    \item ISC 5415 - Computational Space Physics
    \item ISC 5935 - Computational Issues in Optimization
    \item ISC 5935 - Computational Materials Science
    \item ISC 5935 - Data Science Meets Health Science
    \item ISC 5935 - Genome Sequencing and Analysis
    \item ISC 5326 - Introduction to Game and Simulator Design
    \item ISC 5473 - Introduction to Density Functional Theory
    \item ISC 5935 - Machine Learning
    \item ISC 5935 - Mathematical Foundations of Scientific Computing
    \item ISC 5935 - Selected Topics In Computational Science
    \item ISC 5935 - Uncertainty Quantification in the PDE Setting
    \item ISC 5936 - Numerical Methods for Stochastic Differential Equations
    \item MAD 5420 - Numerical Optimization
    \item MAP 5935 - Finite Element Methods
\end{itemize}

\subsection*{Non-coursework Courses}
\begin{itemize}
    \item ISC 5906 - Directed Individual Study In Computational Science
    \item ISC 5907 - Directed Individual Study In Computational Science
    \item ISC 5948 - Graduate Internship In Computational Science
    \item ISC 5975 - Thesis
    \item ISC 6981 - Dissertation
    \item ISC 8964 - Doctoral Qualifying Examination
    \item ISC 8965 - Doctoral Preliminary Examination
    \item ISC 8977 - Master's Thesis Defense
    \item ISC 8982 - Defense of Dissertation
\end{itemize}

\subsection*{Special Topics}
\begin{itemize}
    \item ISC 5935 - Understanding Covid
\end{itemize}
\newpage

\section{M.S. in Data Science Curriculum}
% url: https://www.sc.fsu.edu/graduate/ms/datascience/courses
\textit{At this time, our Interdisciplinary Data Science master’s degree program is only available on FSU’s Tallahassee campus.}

\subsection*{Core Courses (18 Credits Required)}
\begin{tabular}{l l l}
\hline
\textbf{Credit Hours} & \textbf{Course No.} & \textbf{Course Title} \\ \hline
3 & MAP 5196 & Mathematics for Data Science \\
3 & CAP 5768 & Introduction to Data Science \\
3 & STA 5207 & Applied Regression Methods \\
3 & STA 5635 & Machine Learning \\
3 & CAP 5771 & Data Mining \\
2 & PHI 5699 & Data Ethics \\
1 & STA 5910 & Professional Development Seminar \\ \hline
\end{tabular}

\subsection*{Scientific Computing Electives (6+ Credits Required)}
\begin{tabular}{l l l}
\hline
\textbf{Credit Hours} & \textbf{Course No.} & \textbf{Course Title} \\ \hline
3 & ISC 5228 & Monte-Carlo Methods \\
3 & ISC 5307 & Scientific Visualization \\
3 & ISC 5305 & Scientific Programming \\
4 & ISC 5315 & Applied Computational Science I \\
3 & ISC 5318 & High-Performance Computing \\
3 & ISC 5XXX & Cloud Computing \\
3 & ISC 5XXX & Computational Probabilistic Modeling \\
3 & ISC 5XXX & Data Science for Health \\
3 & ISC 5XXX & Neural Differential Equations \\ \hline
\end{tabular}

\subsection*{Free Electives (6+ Credits)}
Free electives are offered by the participating IDS degree programs: Computer Science, Mathematics, Scientific Computing, and Statistics.

\subsection*{Example Course Mapping (Three Semesters)}
\begin{tabular}{|p{4cm}|p{4cm}|p{4cm}|}
\hline
\textbf{Fall Year 1} & \textbf{Spring Year 1} & \textbf{Fall Year 2} \\ \hline
\textbf{[ Core ]} \newline Mathematics for Data Science (3 credits) \newline MAP 5196 & \textbf{[ Core ]} \newline Machine Learning (3 credits) \newline STA 5635 & \textbf{[ Elective ]} \newline Scientific Visualization (3 credits) \newline ISC 5307 \\
\hline
\textbf{[ Core ]} \newline Introduction to Data Science (3 credits) \newline CAP 5768 & \textbf{[ Core ]} \newline Applied Regression Methods (3 credits) \newline STA 5207 & \textbf{[ Elective ]} \newline Applied Computational Science I (4 credits) \newline ISC 5315 \\
\hline
\textbf{[ Core ]} \newline Data Mining (3 credits) \newline CAP 5771 & \textbf{[ Core ]} \newline Data Ethics (2 credits) \newline PHI 5699 & \textbf{[ Elective ]} \newline Monte Carlo Methods (3 credits) \newline ISC 5228 \\
\hline
 & \textbf{[ Core ]} \newline Professional Development (1 credit) \newline STA 5910 & \textbf{[ Elective ]} \newline High-Performance Computing (3 credits) \newline ISC 5318 \\
\hline
 &  & \textbf{Graduation} \\
\hline
\end{tabular}
\newpage

\section{Detailed Course Descriptions}
% Note: The following are detailed descriptions from individual course pages.

\subsection{ISC 1057 - Computational Thinking [3]}
% url: https://www.sc.fsu.edu/index.php?option=com_content&view=article&id=1178&Itemid=288
\textbf{Last Offered:} Fall 2025 \\
ISC 1057. Computational Thinking (3). This course introduces students to the process of creating a representation of a task so that it can be performed by a computer. The course investigates strategies behind popular computational methods that are shaping our daily lives and our future. Students practice logical thinking by applying versions of these computational methods to problems in science and society. \\
\textit{This course has been approved to satisfy the Liberal Studies Quantitative/Logical Thinking requirement.}
\begin{center}
\textbf{Attachments}
\begin{tabular}{l l l}
\hline
\textbf{File} & \textbf{Description} & \textbf{File size} \\ \hline
ISC1057-web.jpg & Advertisement & 1318 kB \\
ISC1057 Poster & Spring 2017 & 1211 kB \\
1057\_syllabus.pdf & Fall 2016 & 71 kB \\ \hline
\end{tabular}
\end{center}

\subsection{ISC 2310 - Data Science with Python [3]}
% url: https://www.sc.fsu.edu/index.php?option=com_content&view=article&id=1556&Itemid=288
\textbf{Last Offered:} Fall 2025 \\
ISC 2310. Introduction to Computational Thinking in Data Science with Python (3). Prerequisite: MAC 1105 or equivalent. This course investigates strategies behind popular computational methods used in data science. In addition, many of the algorithms are implemented using the programming language Python. No prior programming experience is required so the course presents the basics of the Python language as well as how to leverage Python's libraries to solve real-world problems in data science. \\
\textbf{Course Testimonials:}
\begin{itemize}
    \item "I have been wanting to learn Python for a while now and I also wanted to take a class that would expand my knowledge of data analysis/data science. This class was an ideal fit for both of these desires."
    \item "I like the Python portion a lot. Following along to the many examples done in the videos and then doing similar ones on my own in the practice notebooks is very effective. To me, everything here is good."
    \item "I honestly feel as though you guys did a great job with the format of this class. That means everything from the grades, organization of the assignments, videos, and written notes."
    \item "I feel like the data science portion is done well. The reading can be pretty dense but working through examples in text and videos really helped me grasp all the conceptual material."
\end{itemize}
\textit{Formerly: ISC 2310 - Introduction to Computational Thinking in Data Science with Python}

\subsection{ISC 3222 - Symbolic and Numerical Computations [3]}
% url: https://www.sc.fsu.edu/index.php?option=com_content&view=article&id=618&Itemid=288
\textbf{Last Offered:} Fall 2025 \\
ISC 3222. Symbolic and Numerical Computations (3). Prerequisite: MAC 2311. This course introduces state-of-the-art software environments for solving scientific and engineering problems. Topics include solving simple problems in algebra and calculus; 2-D and 3-D graphics; non-linear function fitting and root-finding; basic procedural programming; methods for finding numerical solutions to DE's with applications to chemistry, biology, physics, and engineering.

\subsection{ISC 3313 - Introduction to Scientific Computing [3]}
% url: https://www.sc.fsu.edu/index.php?option=com_content&view=article&id=98&Itemid=288
\textbf{Last Offered:} Fall 2025 \\
\textbf{Programming Language will be C++} \\
ISC 3313. Introduction to Scientific Computing (3). Prerequisites: MAC 2311 or instructor permission. This course introduces the student to the science of computation. Topics cover algorithms for standard problems in computational science, as well as the basics of an object-oriented programming language, to facilitate the students' implementation of algorithms. Satisfies FSU Computer Competency requirement. \\
Each semester, this course is taught with a different programming language, with the following (historical) schedule shown below. The course may also be offered in the summer, in which case the language will be chosen by the instructor.
\begin{itemize}
    \item \textbf{C++:} SP 2025, FA 2024, SP 2024, FA 2023, SP 2023, FA 2022, SP 2022, FA 2021, FA 2020, FA 2019, FA 2018, FA 2016, FA 2015, SP 2014
    \item \textbf{Fortran:} SP 2020, SP 2018, FA 2017, SP 2016, FA 2014
    \item \textbf{Java:} SP 2019, SP 2017, SP 2015, FA 2013
    \item \textbf{Python:} SU 2022, SU 2021, SU 2020
\end{itemize}

\subsection{ISC 4220C - Continuous Algorithms for Science Applications}
% url: https://www.sc.fsu.edu/index.php?option=com_content&view=article&id=619&Itemid=288
\textbf{Last Offered:} Spring 2025 \\
ISC 4220C. Continuous Algorithms for Science Applications (4). Prerequisite: MAC 2312. This course provides basic computational algorithms, including interpolation, approximation, integration, differentiation, and linear systems solution presented in the context of science problems. The laboratory component includes algorithm implementation for simple problems in the sciences and applying visualization software for interpretation of results. \\
\textit{Formerly Algorithms for Science Applications I}

\subsection{ISC 4221C - Discrete Algorithms for Science Applications [4]}
% url: https://www.sc.fsu.edu/index.php?option=com_content&view=article&id=657&Itemid=288
\textbf{Last Offered:} Fall 2025 \\
ISC 4221C. Discrete Algorithms for Science Applications (4). Prerequisites: MAC 2311. This course offers stochastic algorithms, linear programming, optimization techniques, clustering and feature extraction presented in the context of science problems. The laboratory component includes algorithm implementation for simple problems in the sciences and applying visualization software for the interpretation of results. \\
\textit{Formerly named Algorithms for Science Applications II}

\subsection{ISC 4223C - Computational Methods for Discrete Problems [4]}
% url: https://www.sc.fsu.edu/index.php?option=com_content&view=article&id=679&Itemid=288
\textbf{Last Offered:} Fall 2025 \\
ISC 4223C. Computational Methods for Discrete Problems (4). Prerequisites: ISC 4304C and MAS 3105. This course describes several discrete problems arising in science applications, a survey of methods and tools for solving the problems on computers, and detailed studies of algorithms, and their use in science and engineering. The laboratory component illustrates the concepts learned in the context of science problems.

\subsection{ISC 4232C - Computational Methods for Continuous Problems [4]}
% url: https://www.sc.fsu.edu/index.php?option=com_content&view=article&id=680&Itemid=288
\textbf{Last Offered:} Fall 2025 \\
ISC 4232C. Computational Methods for Continuous Problems (4). Prerequisites: ISC 4304C and MAS 3105. This course provides numerical discretization of differential equations and implementation for case studies drawn from several science areas. Finite-difference, finite-element, and spectral methods are introduced and standard software packages are used. The laboratory component aims to illustrate the concepts learned on a variety of application-driven problems.

\subsection{ISC 4304C - Programming for Scientific Applications}
% url: https://www.sc.fsu.edu/index.php?option=com_content&view=article&id=656&Itemid=288
\textbf{Last Offered:} Spring 2025 \\
ISC 4304C. Programming for Science Applications (4). Prerequisite: MAC 2311. This course provides knowledge of a scripting language that serves as a front-end to popular packages and frameworks, along with a compiled language such as (C++). Students study and practice scientific programming with the scripting language and practice how to interface it with a more traditional programming language to improve the speed of the programs developed in the course. In the laboratory component of this course, students apply the concepts learned in class. Students analyze large data sets by translating from mathematical expressions and algorithms to working computer code that is then used to visualize and summarize the results.

\subsection{ISC 4931r - Junior Seminar in Scientific Computing [1]}
% url: https://www.sc.fsu.edu/index.php?option=com_content&view=article&id=720&Itemid=288
\textbf{Last Offered:} Fall 2025 \\
ISC 4931r. Junior Seminar in Scientific Computing (1–2). (S/U grade only.) Prerequisite: Junior standing (sixty plus hours). This is a special topics course in computational science. May be repeated two times to a maximum of four semester hours, with a maximum of only two semester hours credit allowed to be applied to the Computational Science degree.

\subsection{ISC 4932 - Senior Seminar in Scientific Computing [1]}
% url: https://www.sc.fsu.edu/index.php?option=com_content&view=article&id=721&Itemid=288
\textbf{Last Offered:} Fall 2024 \\
ISC 4932r. Senior Seminar in Scientific Computing (1–2). (S/U grade only.) Prerequisite: Senior standing (ninety plus hours). This is a special topics course in computational science. May be repeated one time to a maximum of four semester hours, with a maximum of only one semester hour credit allowed to be applied to the Computational Science degree.

\subsection{ISC 4933 - Understanding Covid}
% url: https://www.sc.fsu.edu/index.php?option=com_content&view=article&id=1650&Itemid=288
\textbf{Offered:} Fall 2022 \\
\textbf{Schedule \& Location:} MWF 10:40-11:30, 499 DSL \\
In this course, students will explore the different elements that determine the outcome of infectious disease in humans, using COVID-19 as a case study. Starting with a very basic model that includes susceptible, infected, and recovered individuals, students will gradually incorporate factors that may affect the outcome of an outbreak, such as masking/quarantining, gain and loss of natural and vaccine-based immunity, and changing virulence/strains. After summarizing data already collected during the recent pandemic, students will inform the model to determine the conditions under which different outcomes may occur. Students will summarize their results graphically and present their findings. No prerequisites or programming experience required. The course is designed to be accessible to all students, regardless of background or major.

\subsection{ISC 4933/5318 - High-Performance Computing}
% url: https://www.sc.fsu.edu/index.php?option=com_content&view=article&id=117&Itemid=296
\textbf{Last Offered:} Spring 2025 \\
ISC 4933/ISC 5318. High-Performance Computing (3). Prerequisites: ISC 5305 or equivalent or instructor permission. This course introduces high-performance computing, term which refers to the use of parallel supercomputers, computer clusters, as well as software and hardware in order to speed up computations. Students learn to write faster code that is highly optimized for modern multi-core processors and clusters, using modern software-development tools and performance analyzers, specialized algorithms, parallelization strategies, and advanced parallel programming constructs.

\subsection{CAP 5771 - Data Mining}
% url: https://www.sc.fsu.edu/index.php?option=com_content&view=article&id=1231&Itemid=296
\textbf{Last Offered:} Spring 2025 \\
CAP 5771. Data Mining (3). Prerequisite: ISC 3222 or ISC 3313 or ISC 4304C or COP 3330 or COP 4530 or instructor permission. This course enables students to study concepts and techniques of data mining, including characterization and comparison, association rules mining, classification and prediction, cluster analysis, and mining complex types of data. Students also examine applications and trends in data mining. \\
\textit{Undergraduate Level (ISC 4245C)}

\subsection{ISC 5228 / ISC 4933 - Monte Carlo Method [3]}
% url: https://www.sc.fsu.edu/index.php?option=com_content&view=article&id=125&Itemid=296
\textbf{Last Offered:} Fall 2025 \\
ISC 5228. Monte Carlo Methods (3). Prerequisites: ISC 5305; MAC 2311, 2312. This course provides an introduction to probabilistic modeling and Monte Carlo methods (MCMs) suitable for graduate students in science, technology, and engineering. It provides an introduction to discrete event simulation, MCMs and their probabilistic foundations, and the application of MCMs to various fields. In particular, Markov chain MCMs are introduced, as are the application of MCMs to problems in linear algebra and the solution of partial differential equations.

\subsection{ISC 5305 - Scientific Programming [3]}
% url: https://www.sc.fsu.edu/index.php?option=com_content&view=article&id=123&Itemid=296
\textbf{Last Offered:} Fall 2025 \\
ISC 5305. Scientific Programming (3). Prerequisites: working knowledge of one programming language (C++, Fortran, Java), or instructor permission. This course focuses on object-oriented coding in C++, Java, and Fortran 90 with applications to scientific programming. Discussion of class hierarchies, pointers, function, and operator overloading and portability. Examples include computational grids and multidimensional arrays.

\subsection{ISC 5307 - Scientific Visualization}
% url: https://www.sc.fsu.edu/index.php?option=com_content&view=article&id=121&Itemid=296
\textbf{Last Offered:} Spring 2025 \\
ISC 5307. Scientific Visualization (3). Prerequisites: CGS 4406, ISC 5305, or instructor permission. The course covers the theory and practice of scientific visualization. Students learn how to use state-of-the-art visualization toolkits, create their own visualization tools, represent both 2-D and 3-D data sets, and evaluate the effectiveness of their visualizations. \\
\textit{Undergraduate Level (ISC 4302)}

\subsection{ISC 5308 - Computational Aspects of Data Assimilation}
% url: https://www.sc.fsu.edu/index.php?option=com_content&view=article&id=596&Itemid=296
\textbf{Last Offered:} Spring 2025 \\
Data assimilation methods combine numerical models and observations to arrive at the best possible representation of a physical system. This course aims to build a robust theoretical foundation in the subject and explore some of the computational challenges in large scientific and engineering applications. Students will gain hands-on experience by implementing their own algorithms and will complete a final project on a preferred research topic. Prerequisites: Applied Statistics for Engineers and Scientists (STA 3032), Applied Linear Algebra I/II (MAS 3105/MAS 4106) and Programming for Scientific Applications (ISC 4304) or Instructor Permission Required.

\subsection{ISC 5314 - Verification and Validation in Computational Science}
% url: https://www.sc.fsu.edu/index.php?option=com_content&view=article&id=346&Itemid=296
\textbf{Last Offered:} Spring 2024 \\
\textbf{Schedule \& Location:} T R 1:20-2:35 (152 DSL) \\
ISC 5314. Verification and Validation in Computational Science (3). Prerequisites: MAC 2312, MAS 3105, or ISC 5315; or instructor permission. This course covers the theory and practice of verification and validation in computational sciences. Students learn basic terminology, are exposed to procedures and practical methods used in software implementation validation and in solution verification, employ exact and manufactured solutions, and explore elements of software quality assurance. The course introduces essential data analysis techniques and reviews software development and maintenance tools. Examples from physical sciences and engineering are used to illustrate aspects of code variation, including validation hierarchy, validation benchmarks, uncertainty quantification and simulation code predictive capabilities. The computational laboratory is an essential part of this course.

\subsection{ISC 5315 - Applied Computational Science I [4]}
% url: https://www.sc.fsu.edu/index.php?option=com_content&view=article&id=120&Itemid=296
\textbf{Last Offered:} Fall 2025 \\
ISC 5315. Applied Computational Science I (4). Prerequisites: ISC 5305; MAP 2302; or instructor permission. This course provides students with high-performance computational tools necessary to investigate problems arising in science and engineering, with an emphasis on combining them to accomplish more complex tasks. A combination of course work and lab work provides the proper blend of theory and practice with problems culled from the applied sciences. Topics include numerical solutions to ODEs and PDEs, data handling, interpolation and approximation, and visualization.

\subsection{ISC 5316 - Applied Computational Science II}
% url: https://www.sc.fsu.edu/index.php?option=com_content&view=article&id=119&Itemid=296
\textbf{Last Offered:} Spring 2025 \\
ISC 5316. Applied Computational Science II (4). Prerequisite: ISC 5315 or instructor permission. This course provides students with high-performance computational tools necessary to investigate problems arising in science and engineering, with an emphasis on combining them to accomplish more complex tasks. A combination of course work and lab work provides the proper blend of theory and practice with problems culled from the applied sciences. Topics include mesh generation, stochastic methods, basic parallel algorithms and programming, numerical optimization, and nonlinear solvers.

\subsection{ISC 5317 / ISC 4933 - Computational Evolutionary Biology [3]}
% url: https://www.sc.fsu.edu/index.php?option=com_content&view=article&id=118&Itemid=296
\textbf{Last Offered:} Fall 2025 \\
ISC 5317. Computational Evolutionary Biology (4). Prerequisites: ISC 5224, ISC 5306, or instructor permission. This course presents computational methods for evolutionary inferences. Topics include the underlying models, the algorithms that analyze models, and the creation of software to carry out the analysis.

\subsection{ISC 5934r - Introductory Seminar on Research in Computational Science [1]}
% url: https://www.sc.fsu.edu/index.php?option=com_content&view=article&id=114&Itemid=296
\textbf{Last Offered:} Spring 2025 \\
ISC 5934r. Introductory Seminar on Research in Computational Science (1). (S/U grade only). A series of lectures given by faculty on research being conducted in the Department of Scientific Computing. \\
\textit{Note: This course should be taken twice.}

\subsection{ISC 5935 - Understanding Covid}
% url: https://www.sc.fsu.edu/index.php?option=com_content&view=article&id=1651&Itemid=296
\textbf{Offered:} Fall 2022 \\
\textbf{Schedule \& Location:} MWF 10:40-11:30, 499 DSL \\
In this course, students will explore the different elements that determine the outcome of infectious disease in humans, using COVID-19 as a case study. Starting with a very basic model that includes susceptible, infected, and recovered individuals, students will gradually incorporate factors that may affect the outcome of an outbreak, such as masking/quarantining, gain and loss of natural and vaccine-based immunity, and changing virulence/strains. After summarizing data already collected during the recent pandemic, students will inform the model to determine the conditions under which different outcomes may occur. Students will summarize their results graphically and present their findings. No prerequisites or programming experience required. The course is designed to be accessible to all students, regardless of background or major.

\subsection{ISC 5935 / ISC 4933 - Science Professional Development [3]}
% url: https://www.sc.fsu.edu/index.php?option=com_content&view=article&id=1739&Itemid=296
\textbf{Last Offered:} Fall 2024 \\
The course would cover topics such as improving communication in science (posters, talks, manuscripts, grants, etc), as well as some work on primary literature, and the publication process. Finally there would be a component on applying for grad school or jobs.

\subsection{ISC 5935 / ISC 4933 - AI Methods and Applications [3]}
% url: https://www.sc.fsu.edu/index.php?option=com_content&view=article&id=1741&Itemid=296
\textbf{Last Offered:} Fall 2025 \\
AI is a fast-moving discipline which is impacting many areas of our daily life. In this course, students will get an introduction into deep learning and reinforcement learning and will also learn novel concepts such as graph neural networks, encoders and attention networks. Applications from several areas such as medical imaging, weather forecast and finance are introduced.

\subsection{ISC 5935 - Atomistic Modeling of Molecules and Materials}
% url: https://www.sc.fsu.edu/index.php?option=com_content&view=article&id=1756&Itemid=296
\textbf{Last Offered:} Spring 2025 \\
The course is designed for students who are interested in atomistic simulations of molecules and materials. Two popular methods will be introduced: density functional theory (DFT) and molecular dynamics. DFT has become the workhorse in industry and academia for calculating various properties of materials and molecules, such as electronic properties, crystal structures, and chemical reaction energies. We will learn both the theoretical and numerical aspects of DFT. Molecular dynamics are invaluable for understanding the dynamical processes of materials at the atomic scale. We will introduce the theories underlying molecular dynamics simulations and learn how to calculate various properties of materials using molecular dynamics. Popular software in these two fields will be introduced. Please contact \textit{[Email Protected]} for additional details.

\subsection{ISC 5935 - Computational Methods in Fire Science}
% url: https://www.sc.fsu.edu/index.php?option=com_content&view=article&id=1757&Itemid=296
\textbf{Last Offered:} Spring 2024 \\
\textbf{Schedule \& Location:} M W F 10:40-11:30, 499 DSL \\
Have you ever wondered why fires spread and grow in size so quickly or how smoke plumes can travel thousands of kilometers? These behaviors are governed by fuel properties, atmospheric conditions, topography, and more. This course introduces physics-based and data-driven models in fire science, and investigates the sensitivity and uncertainty of these models. We will discuss computational tools including cellular automata, level set methods, and data-driven methods for discovering equations from measured data. Finally, we will explore techniques for analyzing both simulated and measured fire data. Please contact \textit{[Email Protected]} for additional details. \\
\textit{Undergraduate Level (ISC 4933)}

\subsection{ISC 5935 - Data Science Meets Health Science}
% url: https://www.sc.fsu.edu/index.php?option=com_content&view=article&id=1676&Itemid=296
\textbf{Last Offered:} Spring 2025 \\
This course will focus on the applied data science pipeline of data acquisition, data processing and integration, data modeling and analysis, and validation and delivery, commonly used in the Health industry. Topics include data normalization, scientific visualization, multivariate regression, and Artificial Neural Networks (dense, convolutional, recurrent, and adversarial). The examples and projects of this course contain 1D to 4D health data of electrocardiogram sequences, X-ray, Magnetic resonance imaging (MRI), and functional MRI images. \\
\textit{Undergraduate Level (ISC 4933)}

\subsection{ISC 5935 - Computational Probabilistic Modeling}
% url: https://www.sc.fsu.edu/index.php?option=com_content&view=article&id=1677&Itemid=296
\textbf{Last Offered:} Spring 2025 \\
ISC 4933/ISC 5935. Computational Probabilistic Modeling (3). Prerequisites: MAC 2312 - Calculus II, MAS 3105 – Applied Linear Algebra, and STA 4442/5440: Introduction To Probability or STA 4321/5323: Introduction to Mathematical Statistics, or the permission of the instructor. In this course, students are introduced to probabilistic programming and modeling for modern data science and machine learning applications. Algorithms for predictive inference are covered from a theoretical and practical viewpoint with an emphasis on implementation in Python. Topics include an introduction to probability and learning theory, graph-based methods, machine learning with neural networks, dimensionality reduction, and algorithms for big data.

\subsection{ISC 5939 - Advance Grad Seminar in Fire Dynamics}
% url: https://www.sc.fsu.edu/index.php?option=com_content&view=article&id=1678&Itemid=296
\textbf{Last Offered:} Spring 2025 \\
Please contact \textit{[Email Protected]} for details.

\subsection{GFD 5500L - Fire Dynamics Field School}
% url: https://www.sc.fsu.edu/index.php?option=com_content&view=article&id=1680&Itemid=296
\textbf{Last Offered:} Spring 2025 \\
This course will provide a multidisciplinary hands-on apprenticeship to the field methods most commonly used in prescribed fire. It will give graduate students the opportunity to gain a greater appreciation of the complexity of weather and the atmospheric boundary layer and fire as a fluid system, and its dynamics through the active participation in prescribed fire field research, operations, organizational and management concepts. It is an entry-level preparation for students considering a career in fire dynamics, fire science and management. This course represents a "real-world" practical project experience with time and budget constraints. Students will propose field measurements and produce a short research paper, and present scientific results. Students will have the opportunity to learn and practice basic land survey techniques, basic site fuels evaluation, basic preparation of burn units, public outreach activities via the project website, and involvement in the organizational and logistic requirements of staging and operating a field project. Students will be introduced to, and work with, various types of surveying, photographic, video, and computer equipment during the course of the field school. Interested graduate students from related natural sciences (Physics, Meteorology, Environmental Sciences or Studies, Biology, Geosciences, or Chemistry, etc.) are welcome. \\
\textit{Fire Dynamics Field School Research and Operations in Prescribed Fire}

\subsection{GFD 5936 / GFD 4934 - Advanced Topics in Fire Dynamics [1]}
% url: https://www.sc.fsu.edu/index.php?option=com_content&view=article&id=1759&Itemid=296
\textbf{Last Offered:} Spring 2025 \\
Please contact \textit{[Email Protected]} for details. \\
This graduate seminar course will expose students to a set of selected topics in fire dynamics research through a variety of methods(group discussion of ongoing student research, faculty research, and outside speakers). Students will additionally prepare and present a short research presentation suitable for a conference. Topics will be drawn from different disciplines of fire dynamics, including fluid dynamics, meteorology, physics, computational sciences, ecology, forestry. Interested graduate students from related natural sciences (Physics, Meteorology, Environmental Sciences or Studies, Biology, Geosciences, or Chemistry, etc.) are welcome. Instructor permission required.

\section{Student Advising}
\url{https://www.sc.fsu.edu/advising}

The Department of Scientific Computing is staffed by a full-time academic advisor, located in room 406 Dirac Science Library (click here for directions). Students are seen on a walk-in basis during normal business hours.

Students are strongly advised to seek academic advising at least once a semester, preferably before the registration cycle. Students should see individual faculty members for advice about research interests and potential Directed Individual Study (DIS) opportunities.

To make an academic advising appointment please email (This email address is being protected from spambots. You need JavaScript enabled to view it.) or call (850-644-0143).

\subsection*{Data Science Advising}
Please see \textbf{Jennifer Clark}.

\subsection*{Karey G. Fowler}
\textit{Academic Advisor}
\vspace{1em}

\textbf{Contact Info:} \\
This email address is being protected from spambots. You need JavaScript enabled to view it. \\
406~DSL \\
(850)~644-0143
\vspace{1em}

\textbf{Address:} \\
Karey~Fowler \\
Dept. of Scientific Computing \\
Florida State University \\
400 Dirac Science Library \\
Tallahassee, FL 32306-4120

\section{Remote Course Resources}
\url{https://www.sc.fsu.edu/remote-course-resources}

\subsection{FSU News - Aug 26, 2020}

\subsection{Twelve technology resources for teaching remotely}

\subsection{User Generated Education}

\subsection{Increasing Student Participation During Zoom Synchronous Teaching Meetings}

\section{Emergency Info}
\url{https://www.sc.fsu.edu/emergency}

\subsection{University-Wide Information}
\begin{itemize}
    \item \textbf{STAY HEALTHY FSU}
    \item \textbf{ALIGNED WITH THE FLORIDA BOARD OF GOVERNORS RE-OPENING BLUEPRINT}
    \item \textbf{COVID-19 CASE STATISTICS}
    \item \textbf{FSU ALERT - Emergency Information and Instructions} \\
    CURRENT UNIVERSITY STATUS. OFFICIAL ANNOUNCEMENTS. SITUATION REPORTS.
    \item \textbf{FSU's coronavirus response}
    \item \textbf{Coronavirus: Frequently Asked Questions}
\end{itemize}

\subsection{Remote Learning and Work Resources}
\begin{itemize}
    \item \textbf{Office of Distance Learning (ODL) Emergency Online Instruction} \\
    When an emergency affects your ability to teach your courses, ODL is here to help. Even if you've never taught online, learning a few basics can help you keep your courses running in spite of a campus closure. It can also help you recover after the emergency is over. Whether you're new to Canvas or have some experience, we've assembled resources for delivering your classes online.
    
    \item \textbf{FSU ITS Employee Remote Work Essentials} \\
    To successfully work remotely, you’ll need a computer, a reliable internet connection and your FSU login credentials. We also recommend bookmarking your essential websites to make accessing your go-to files easier. Make sure to take the time to test everything out before you plan to be away and make sure you are set up for success.
\end{itemize}

\subsection{External Resources and Data}
\begin{itemize}
    \item \textbf{Who’s staying home because of COVID-19?} \\
    A list of all the companies WFH or events changed because of covid-19.
    
    \item \textbf{MAP: Coronavirus Resource Center - Johns Hopkins} \\
    Track Reported Cases of COVID-19.
\end{itemize}

\subsection{Attachments}
\begin{center}
\begin{tabular}{p{6cm} p{6cm} r}
\hline
\textbf{File} & \textbf{Description} & \textbf{File size} \\
\hline
COVID-19 CampusEventsGuidance.pdf & & 144 kB \\
\hline
\end{tabular}
\end{center}

\section{Administration}
\texttt{\url{https://www.sc.fsu.edu/people/administration}}

\subsection{Administrative Staff}

\paragraph{Davis, Liston}
\textit{Administrative Associate}\\
\textbf{Contact}: [email protected] / (850) 645-0304 / 409 DSL\\
\textbf{Interests}: Procurement, HR, Travel\\
\textbf{Address}: Dept. of Scientific Computing, Florida State University, 400 Dirac Science Library, Tallahassee, FL 32306-4120

\paragraph{Farmer, Cecelia}
\textit{Admin Support Assistant}\\
\textbf{Contact}: [email protected] / 408A DSL\\
\textbf{Address}: Dept. of Scientific Computing, Florida State University, 400 Dirac Science Library, Tallahassee, FL 32306-4120

\paragraph{Fowler, Karey G.}
\textit{Academic Advisor}\\
\textbf{Contact}: [email protected] / (850) 644-0143 / 406 DSL\\
\textbf{Social Media}: \href{https://www.linkedin.com/in/karey-fowler-90a18430}{LinkedIn Profile}\\
\textbf{Address}: Dept. of Scientific Computing, Florida State University, 400 Dirac Science Library, Tallahassee, FL 32306-4120

\paragraph{Clark, Jennifer S.}
\textit{Administrative Director, Interdisciplinary Data Science}\\
\textbf{Contact}: [email protected] / (850) 645-8887 / 497 DSL\\
\textbf{Homepage}: \href{https://datascience.fsu.edu/}{Personal Homepage}\\
\textbf{Social Media}: \href{https://www.linkedin.com/in/jennifer-s-clark-ids/}{LinkedIn Profile}\\
\textbf{Address}: Longmire Bldg., Room 208, 125 Convocation Way, Tallahassee, FL 32306-1280

\paragraph{Keyser, Paul}
\textit{Administrative Assistant / Facilities, Property, \& Immigration}\\
\textbf{Contact}: [email protected] / (850) 644-1864 / 404 DSL\\
\textbf{Address}: Dept. of Scientific Computing, Florida State University, 400 Dirac Science Library, Tallahassee, FL 32306-4120

\paragraph{Overturf, Daniel}
\textit{Business Manager}\\
\textbf{Contact}: [email protected] / (850) 644-2273 / 405 DSL\\
\textbf{Address}: Dept. of Scientific Computing, Florida State University, 400 Dirac Science Library, Tallahassee, FL 32306-4120

\subsection{Technical Support Group}
\textit{Email Support at [email protected]}

\paragraph{Amato, Anthony}
\textit{IT Support Specialist}\\
\textbf{Contact}: [email protected] / (850) 645-1422 / 452 DSL\\
\textbf{Social Media}: \href{https://www.linkedin.com/in/anthony-amato-844438186/}{LinkedIn Profile}\\
\textbf{Address}: Dept. of Scientific Computing, Florida State University, 400 Dirac Science Library, Tallahassee, FL 32306-4120

\paragraph{Li, Xiaoguang}
\textit{Systems Administrator}\\
\textbf{Contact}: [email protected] / (850) 644-0188 / 432 DSL\\
\textbf{Homepage}: \href{http://people.sc.fsu.edu/~xli3/}{Personal Homepage}\\
\textbf{Social Media}: \href{https://www.linkedin.com/in/xiaoguang-li-9243729/}{LinkedIn Profile}\\
\textbf{Address}: Dept. of Scientific Computing, Florida State University, 400 Dirac Science Library, Tallahassee, FL 32306-4120

\paragraph{McDonald, Michael}
\textit{Sr. Research Associate / Systems Administrator}\\
\textbf{Contact}: [email protected] / (772) 600-4665 / 408 DSL\\
\textbf{Homepage}: \href{http://people.sc.fsu.edu/~emm2013/}{Personal Homepage}\\
\textbf{Social Media}: \href{https://www.linkedin.com/in/mcdonald/}{LinkedIn Profile}\\
\textbf{Research Interests}: High-performance Computing, Data Storage, Web Design, Video Conferencing, Social Networks\\
\textbf{Education}:
\begin{itemize}
    \item M.S., Electrical Engineering, Florida State University, 2007
    \item B.S., Computer Engineering, Florida State University, 2005
    \item B.S., Electrical Engineering, Florida State University, 2005
\end{itemize}
\textbf{Publications}: \href{https://scholar.google.com/citations?user=GEXo59sAAAAJ}{Google Scholar Citations}\\
\textbf{Address}: Dept. of Scientific Computing, Florida State University, 400 Dirac Science Library, Tallahassee, FL 32306-4120

\paragraph{Thompson, John}
\textit{IT Support Specialist}\\
\textbf{Contact}: [email protected] / (850) 933-5866 / 450 DSL\\
\textbf{Address}: Dept. of Scientific Computing, Florida State University, 400 Dirac Science Library, Tallahassee, FL 32306-4120

\section{Faculty}
\texttt{\url{https://www.sc.fsu.edu/people/faculty}}

\subsection{Core Faculty}

\paragraph{Beerli, Peter}
\textit{Professor \& Chair, Department of Scientific Computing}\\
\textbf{Contact}: [email protected] / (850) 644-1010 / 402 DSL\\
\textbf{Homepage}: \href{http://people.sc.fsu.edu/~pbeerli/}{Personal Homepage}\\
\textbf{Social Media}: \href{https://twitter.com/peterbeerli}{Twitter}, \href{https://www.instagram.com/peterbeerli/}{Instagram}, \href{https://www.linkedin.com/in/peter-beerli-4b55345/}{LinkedIn Profile}\\
\textbf{Research Interests}: Biological Sciences, Population Genetics, Phylogenetics, Bayesian inference, Model selection. Developer of MIGRATE software for analyzing large-scale DNA datasets.\\
\textbf{Education}:
\begin{itemize}
    \item 1994 PhD Zoology, University of Zurich, Switzerland
    \item 1994-2003 Postdoc, Research assistant, Research assistant professor with Joseph Felsenstein, University of Washington, Seattle WA
\end{itemize}
\textbf{Publications}: \href{https://scholar.google.com/citations?user=4ws5d1kAAAAJ}{Google Scholar Citations}\\
\textbf{Address}: Dept. of Scientific Computing, Florida State University, 400 Dirac Science Library, Tallahassee, FL 32306-4120

\paragraph{Chipilski, Hristo}
\textit{Assistant Professor}\\
\textbf{Contact}: [email protected] / 483 DSL\\
\textbf{Social Media}: \href{https://twitter.com/hristochipilski}{Twitter}, \href{https://www.linkedin.com/in/hristo-chipilski-b12480119/}{LinkedIn Profile}\\
\textbf{Research Interests}: Data Assimilation, Artificial Intelligence, Numerical Weather Prediction, Atmospheric Dynamics. Focus on developing new data assimilation methods using AI and improving representation of atmospheric convection.\\
\textbf{Education}:
\begin{itemize}
    \item 2021-2023: ASP Postdoctoral Fellow, National Center for Atmospheric Research
    \item 2016-2021: PhD, Meteorology, University of Oklahoma
    \item 2012-2016: MMet, Meteorology and Climate, University of Reading (UK)
\end{itemize}
\textbf{Publications}: \href{https://scholar.google.com/citations?user=wU9kYnUAAAAJ}{Google Scholar Citations}\\
\textbf{Address}: Dept. of Scientific Computing, Florida State University, 400 Dirac Science Library, Tallahassee, FL 32306-4120

\paragraph{Dexter, Nicholas}
\textit{Assistant Professor}\\
\textbf{Contact}: [email protected] / 489 DSL\\
\textbf{Homepage}: \href{https://nickdexter.github.io/}{Personal Homepage}\\
\textbf{Social Media}: \href{https://www.linkedin.com/in/nicholas-dexter-b9a352166/}{LinkedIn Profile}\\
\textbf{Research Interests}: Data Science, Scientific Computing, Deep Learning, Compressed Sensing, Mathematical Optimization, Uncertainty Quantification, Numerical Analysis, Inverse problem, Computational Epidemiology, Computational Genomics, Harmonic Analysis.\\
\textbf{Education}:
\begin{itemize}
    \item Ph.D., Mathematics, University of Tennessee, 2018
    \item B.S., Computational Mathematics, Rochester Institute of Technology, 2011
\end{itemize}
\textbf{Publications}: \href{https://scholar.google.com/citations?user=Y4k7e7QAAAAJ}{Google Scholar Citations}\\
\textbf{Address}: Dept. of Scientific Computing, Florida State University, 400 Dirac Science Library, Tallahassee, FL 32306-4120

\paragraph{Erlebacher, Gordon}
\textit{Professor, Department of Scientific Computing \& Program Director, Interdisciplinary Data Science}\\
\textbf{Contact}: [email protected] / (850) 322-0194 / 464 DSL\\
\textbf{Social Media}: \href{https://www.linkedin.com/in/gordon-erlebacher-389178/}{LinkedIn Profile}\\
\textbf{Research Interests}: Data Science, Artificial Intelligence. Development of AI since 2014, including autoencoders, GNNs, topic modeling, and transformers. Current focus on applying LLMs in education and agentic systems for administrative efficiency.\\
\textbf{Education}:
\begin{itemize}
    \item B.Sc., Free University of Brussels, Belgium
    \item M.Sc., Free University of Brussels, Belgium
    \item Ph.D., Columbia University, New York, NY
\end{itemize}
\textbf{Publications}: \href{https://scholar.google.com/citations?user=j52f5hIAAAAJ}{Google Scholar Citations}\\
\textbf{Address}: Dept. of Scientific Computing, Florida State University, 400 Dirac Science Library, Tallahassee, FL 32306-4120

\paragraph{Huang, Chen}
\textit{Associate Professor}\\
\textbf{Contact}: [email protected] / (850) 644-2434 / 484 DSL\\
\textbf{Homepage}: \href{http://people.sc.fsu.edu/~chuang3/}{Personal Homepage}\\
\textbf{Social Media}: \href{https://www.linkedin.com/in/chen-huang-09068010/}{LinkedIn Profile}\\
\textbf{Research Interests}: Computational Materials Science, Scientific Computing. Developing new multiphysics methods and orbital-free density functional theory for large-scale, accurate simulations of functional materials.\\
\textbf{Education}:
\begin{itemize}
    \item Ph.D. Princeton University
    \item B.Sc. Tsinghua University, P. R. China
\end{itemize}
\textbf{Publications}: \href{https://scholar.google.com/citations?user=W4yV4dAAAAAJ}{Google Scholar Citations}\\
\textbf{Address}: Dept. of Scientific Computing, Florida State University, 400 Dirac Science Library, Tallahassee, FL 32306-4120

\paragraph{Lemmon, Alan}
\textit{Professor}\\
\textbf{Contact}: [email protected] / (850) 445-4393 / 150D DSL\\
\textbf{Homepage}: \href{http://lemmon.sc.fsu.edu}{Personal Homepage}\\
\textbf{Research Interests}: Biological Sciences\\
\textbf{Publications}: \href{https://scholar.google.com/citations?user=FxnE47sAAAAJ}{Google Scholar Citations}\\
\textbf{Address}: Dept. of Scientific Computing, Florida State University, 400 Dirac Science Library, Tallahassee, FL 32306-4120

\paragraph{Meyer-Baese, Anke}
\textit{Professor}\\
\textbf{Contact}: [email protected] / (850) 644-3494 / 476 DSL\\
\textbf{Homepage}: \href{http://www.sc.fsu.edu/~ameyerbaese/}{Personal Homepage}\\
\textbf{Social Media}: \href{https://www.linkedin.com/in/anke-meyer-baese-b4a1b87/}{LinkedIn Profile}\\
\textbf{Research Interests}:
\begin{itemize}
    \item Medical imaging: pattern recognition for breast MRI, computer-aided diagnosis, fMRI.
    \item Computational biology: dynamical analysis of gene regulatory networks, graph theory in therapeutics.
    \item Computational neuroscience: brain-based classification, stability analysis of cortical systems.
\end{itemize}
\textbf{Education}:
\begin{itemize}
    \item 1995 Ph.D., Electrical and Computer Engineering, Darmstadt University of Technology, Germany
    \item 1990 M.S., Electrical and Computer Engineering Darmstadt University of Technology, Germany
\end{itemize}
\textbf{Publications}: \href{https://scholar.google.com/citations?user=Y4fWz-sAAAAJ}{Google Scholar Citations}\\
\textbf{Address}: Dept. of Scientific Computing, Florida State University, 400 Dirac Science Library, Tallahassee, FL 32306-4120

\paragraph{Plewa, Tomasz}
\textit{Professor}\\
\textbf{Contact}: [email protected] / 413 DSL\\
\textbf{Homepage}: \href{http://plewa.sc.fsu.edu/}{Personal Homepage}\\
\textbf{Social Media}: \href{https://www.linkedin.com/in/tomasz-plewa-1b422a4/}{LinkedIn Profile}\\
\textbf{Research Interests}: Computational Astrophysics, Scientific Computing, fluid dynamics, reactive flows, turbulence, AMR, machine learning for subgrid scale modeling, supernovae, high-energy density physics, HPC.\\
\textbf{Publications}: \href{https://scholar.google.com/citations?user=7i-0zcAAAAAJ}{Google Scholar Citations}\\
\textbf{Address}: Dept. of Scientific Computing, Florida State University, 400 Dirac Science Library, Tallahassee, FL 32306-4120

\paragraph{Quaife, Bryan}
\textit{Associate Professor}\\
\textbf{Contact}: [email protected] / 444 DSL\\
\textbf{Homepage}: \href{http://people.sc.fsu.edu/~bquaife/}{Personal Homepage}\\
\textbf{Social Media}: \href{https://www.linkedin.com/in/bryan-quaife-591b9356/}{LinkedIn Profile}\\
\textbf{Research Interests}: Integral equation methods for complex fluids, PDEs on surfaces, viscous flow in porous media; Efficient and high-order methods for solving integral equations; Adaptive time stepping schemes; Preconditioners.\\
\textbf{Publications}: \href{https://scholar.google.com/citations?user=Y9o12rsAAAAJ}{Google Scholar Citations}\\
\textbf{Address}: Dept. of Scientific Computing, Florida State University, 400 Dirac Science Library, Tallahassee, FL 32306-4120

\paragraph{Shanbhag, Sachin}
\textit{Professor}\\
\textbf{Contact}: [email protected] / (850) 644-6548 / 488 DSL\\
\textbf{Homepage}: \href{https://sachin.sc.fsu.edu/}{Personal Homepage}\\
\textbf{Research Interests}: Polymer Physics, Rheology, Complex Fluids, Modeling for Biological and Materials Applications.\\
\textbf{Education}: B.Tech, IIT Bombay (1999)\\
\textbf{Publications}: \href{https://scholar.google.com/citations?user=1pWDPk0AAAAJ}{Google Scholar Citations}\\
\textbf{Address}: Dept. of Scientific Computing, Florida State University, 400 Dirac Science Library, Tallahassee, FL 32306-4120

\paragraph{Speer, Kevin}
\textit{Professor}\\
\textbf{Contact}: [email protected] / (850) 644-5594\\
\textbf{Research Interests}: Director of Geophysical Fluid Dynamics Institute (GFDI). Sea-going oceanographer studying global ocean circulation and dynamics of hydrothermal plumes.\\
\textbf{Address}: Dept. of Scientific Computing, Florida State University, 400 Dirac Science Library, Tallahassee, FL 32306-4120

\paragraph{Wang, Xiaoqiang}
\textit{Professor}\\
\textbf{Contact}: [email protected] / 495 DSL\\
\textbf{Homepage}: \href{http://people.sc.fsu.edu/~wwang3/}{Personal Homepage}\\
\textbf{Social Media}: \href{https://www.linkedin.com/in/xiaoqiang-wang-34078512/}{LinkedIn Profile}\\
\textbf{Research Interests}: Numerical analysis and applied partial differential equations, Mathematical biology, Image processing, scientific visualization and data mining, High-performance scientific computing.\\
\textbf{Education}: Ph.D., Pennsylvania State University, 2005\\
\textbf{Address}: Dept. of Scientific Computing, Florida State University, 400 Dirac Science Library, Tallahassee, FL 32306-4120

\paragraph{Zavala Romero, Olmo}
\textit{Assistant Professor}\\
\textbf{Contact}: [email protected] / 445 DSL\\
\textbf{Homepage}: \href{http://people.sc.fsu.edu/~osz09/}{Personal Homepage}\\
\textbf{Social Media}: \href{https://www.linkedin.com/in/olmozavala/}{LinkedIn Profile}\\
\textbf{Research Interests}: Data Science, Scientific Computing, Medical Image Processing. Applied ML in medical imaging and earth sciences, climate change, Scientific Machine Learning (PINNs).\\
\textbf{Education}:
\begin{itemize}
    \item Ph.D., Computation Science, Florida State University, 2015
    \item M.S., Computation Science, Florida State University, 2013
\end{itemize}
\textbf{Publications}: \href{https://scholar.google.com/citations?user=m2A2s9wAAAAJ}{Google Scholar Citations}\\
\textbf{Address}: Dept. of Scientific Computing, Florida State University, 400 Dirac Science Library, Tallahassee, FL 32306-4120

\subsection{Emeritus Faculty}

\paragraph{Gunzburger, Max}
\textit{Robert O. Lawton Distinguished Professor / Krafft Professor Emeritus}\\
\textbf{Contact}: [email protected]\\
\textbf{Homepage}: \href{http://people.sc.fsu.edu/~mgunzburger/}{Personal Homepage}\\
\textbf{Research Interests}: Mathematics, Scientific Computing\\
\textbf{Education}:
\begin{itemize}
    \item Ph.D., New York University, 1969
    \item M.S., New York University, 1967
    \item B.S., New York University, 1966
\end{itemize}
\textbf{Publications}: \href{https://scholar.google.com/citations?user=c22gq9kAAAAJ}{Google Scholar Citations}\\
\textbf{Address}: Dept. of Scientific Computing, Florida State University, 400 Dirac Science Library, Tallahassee, FL 32306-4120

\paragraph{Navon, Mike}
\textit{Professor Emeritus}\\
\textbf{Contact}: [email protected]\\
\textbf{Homepage}: \href{http://people.sc.fsu.edu/~inavon/}{Personal Homepage}\\
\textbf{Social Media}: \href{https://www.linkedin.com/in/mike-navon-5932599/}{LinkedIn Profile}\\
\textbf{Research Interests}: Reduced order modeling\\
\textbf{Publications}: \href{https://scholar.google.com/citations?user=R9g8FzoAAAAJ}{Google Scholar Citations}\\
\textbf{Address}: Dept. of Scientific Computing, Florida State University, 400 Dirac Science Library, Tallahassee, FL 32306-4120

\paragraph{Peterson, Janet}
\textit{Professor Emeritus}\\
\textbf{Contact}: [email protected]\\
\textbf{Homepage}: \href{http://people.sc.fsu.edu/~jpeterson/}{Personal Homepage}\\
\textbf{Social Media}: \href{https://www.linkedin.com/in/janet-peterson-14022a4/}{LinkedIn Profile}\\
\textbf{Research Interests}: Scientific Computing\\
\textbf{Address}: Dept. of Scientific Computing, Florida State University, 400 Dirac Science Library, Tallahassee, FL 32306-4120

\subsection{Affiliated Faculty}

\paragraph{Algee-Hewitt, Bridget FB}
\textit{Senior Research Scientist, Stanford University}
\paragraph{Barbu, Adrian}
\textit{Associate Professor, Department of Statistics, Florida State University}
\paragraph{Chi, Hongmei}
\textit{Department of computer and information sciences, FAMU}
\paragraph{Crock, Nathan}
\textit{Director, NewSci Labs}
\paragraph{Duke, Dennis}
\textit{Professor, Physics}
\paragraph{Ke, Fengfeng}
\textit{Associate Professor, Educational Psychology and Learning Systems Department, College of Education}
\paragraph{Linn, Rodman}
\textit{Los Alamos National Lab}
\paragraph{Mascagni, Michael}
\textit{Professor, Department of Computer Science, Florida State University}
\paragraph{Mashayekhi, Somayeh}
\textit{Affiliated Faculty}
\paragraph{Moore, Nick}
\textit{Associate Professor of Mathematics}
\paragraph{Petersen, Mark}
\textit{Los Alamos National Lab}
\paragraph{Pinker-Domenig, Katja}
\textit{Associate Professor of Radiology / Medical University Vienna / Memorial Sloan Kettering Cancer Center}
\paragraph{Ridley, Dennis}
\textit{Professor of Operations Management/Global Logistics, Florida A\&M University}
\paragraph{Tahmassebi, Amirhessam}
\textit{Senior Data Scientist, Rivian}
\paragraph{Ye, Ming}
\textit{Professor, EOAS}


\section{Post Docs}
\texttt{\url{https://www.sc.fsu.edu/people/post-docs}}
\textit{Email the Post Docs at [email protected]}

\paragraph{Chouhan, Pankaj}
\textit{Postdoctoral Research Associate}\\
\textbf{Contact}: [email protected] / 421E DSL\\
\textbf{Social Media}: \href{https://www.linkedin.com/in/pankajchouhan/}{LinkedIn Profile}\\
\textbf{Research Interests}: Machine Learning, Data Science\\
\textbf{Education}: Ph.D., Computational Science, Florida State University, Summer 2024\\
\textbf{Address}: Dept. of Scientific Computing, Florida State University, 400 Dirac Science Library, Tallahassee, FL 32306-4120

\paragraph{Khodaei, Tara}
\textit{Postdoctoral Research Associate}\\
\textbf{Contact}: [email protected]\\
\textbf{Social Media}: \href{https://www.linkedin.com/in/tara-khodaei-80b72a11b/}{LinkedIn Profile}\\
\textbf{Education}:
\begin{itemize}
    \item Ph.D., Computational Science, Florida State University, Fall 2023
    \item M.S., Computational Science, Florida State University, Fall 2020
\end{itemize}
\textbf{Address}: Dept. of Scientific Computing, Florida State University, 400 Dirac Science Library, Tallahassee, FL 32306-4120

\section{Students}
\texttt{\url{https://www.sc.fsu.edu/people/students}}

\subsection{Graduate Students}
\textit{Email the Grad Students at [email protected]}
\begin{itemize}
    \item \textbf{Anumandla, Sri Ram Reddy} --- \textit{Graduate Student}, Authentication Required
    \item \textbf{Asare, Stephen} --- \textit{Graduate Student}, Authentication Required
    \item \textbf{Bellaire, Jessica} --- \textit{Graduate Student}, Authentication Required
    \item \textbf{Broling, Mary} --- \textit{Graduate Student}, Authentication Required
    \item \textbf{Brooker, Ezra} --- \textit{Graduate Student}, \href{https://www.linkedin.com/in/ezra-brooker-3957a2228/}{LinkedIn Profile}, Authentication Required
    \item \textbf{Calvo, Maria} --- \textit{Graduate Student}, Authentication Required
    \item \textbf{Chellu, Srirama Murthy} --- \textit{Graduate Student}, Authentication Required
    \item \textbf{Chen, Zehao} --- \textit{Graduate Student}, \href{https://www.linkedin.com/in/zehao-chen-5b5657140}{LinkedIn Profile}, Authentication Required
    \item \textbf{Cherry, Jake} --- \textit{Graduate Student}, Authentication Required
    \item \textbf{Chintagumpula, Jayakrishna Sai} --- \textit{Graduate Student}, Authentication Required
    \item \textbf{Choudhary, Piyush} --- \textit{Graduate Student}, Authentication Required
    \item \textbf{Galla Wellalage, Kavindu} --- \textit{Graduate Student}, Authentication Required
    \item \textbf{Ghosh, Kisalay} --- \textit{Graduate Student}, Authentication Required
    \item \textbf{Gowram, Yashwanth Goud} --- \textit{Graduate Student}, Authentication Required
    \item \textbf{Greenwood, Jhamieka} --- \textit{Graduate Student}, Authentication Required
    \item \textbf{Gunasekara, Haputhanthrige Lasitha} --- \textit{Graduate Student}, Authentication Required
    \item \textbf{Hechirla, Shirisha} --- \textit{Graduate Student}, Authentication Required
    \item \textbf{Jahan, Iffat} --- \textit{Graduate Student}, Authentication Required
    \item \textbf{Kamble, Anand} --- \textit{Graduate Student}, Authentication Required
    \item \textbf{Kanawade, Kunal} --- \textit{Graduate Student}, Authentication Required
    \item \textbf{Katiyar, Kirti} --- \textit{Graduate Student}, Authentication Required
    \item \textbf{Kim, Young Hwan} --- \textit{Graduate Student}, Authentication Required
    \item \textbf{Kinnane, Keely} --- \textit{Graduate Student}, Authentication Required
    \item \textbf{Miranda, Jose} --- \textit{Graduate Student}, Authentication Required
    \item \textbf{Mukherjee, Eesha} --- \textit{Graduate Student}, Authentication Required
    \item \textbf{Murikipudi, Sudha} --- \textit{Graduate Student}, Authentication Required
    \item \textbf{Nellepalli, Divya} --- \textit{Graduate Student}, Authentication Required
    \item \textbf{Ouyang, Haoyong} --- \textit{Graduate Student}, Authentication Required
    \item \textbf{Perez, Dorianis} --- \textit{Graduate Student}, Authentication Required
    \item \textbf{Poudel, Sanjeeb} --- \textit{Graduate Student}, Authentication Required
    \item \textbf{Rompicherla, Sameera} --- \textit{Graduate Student}, Authentication Required
    \item \textbf{Sagel, Daryn} --- \textit{Graduate Student}, \href{https://www.linkedin.com/in/daryn-sagel-a14002131/}{LinkedIn Profile}, Authentication Required
    \item \textbf{Shakil, Shifur Rahman} --- \textit{Graduate Student}, Authentication Required
    \item \textbf{Shooshtari, MJ} --- \textit{Graduate Student}, Authentication Required
    \item \textbf{Sokolikj, Zlatko} --- \textit{Graduate Student}, \href{https://www.linkedin.com/in/zlatko-sokolikj-053b51134/}{LinkedIn Profile}, Authentication Required
    \item \textbf{Sreeramoju, Ankitha} --- \textit{Graduate Student}, Authentication Required
    \item \textbf{Taiyebah, Farhana} --- \textit{Graduate Student}, Authentication Required
    \item \textbf{Tummala, Purna} --- \textit{Graduate Student}, Authentication Required
    \item \textbf{Tyagi, Mani} --- \textit{Graduate Student}, Authentication Required
    \item \textbf{Velasco Zavala, Jorge Eduardo} --- \textit{Graduate Student}, Authentication Required
    \item \textbf{Wade, Peyton} --- \textit{Graduate Student}, Authentication Required
    \item \textbf{Wang, Yifan} --- \textit{Graduate Student}, Authentication Required
    \item \textbf{White, Liam} --- \textit{Graduate Student}, \href{https://www.linkedin.com/in/liam-white-090906154/}{LinkedIn Profile}, Authentication Required
    \item \textbf{Ziegler, Kevin} --- \textit{Graduate Student}, Authentication Required
\end{itemize}

\subsection{Affiliated Students}
\paragraph{Mittal, Shivangi}
\textit{Visiting Fulbright Research Scholar}\\
\textbf{Contact}: [email protected] / 472 DSL\\
\textbf{Address}: Dept. of Scientific Computing, Florida State University, 400 Dirac Science Library, Tallahassee, FL 32306-4120

\section{Alumni}
\texttt{\url{https://www.sc.fsu.edu/people/students/alumni}}

\begin{itemize}
    \item \textbf{Alvarez Illan, Ignacio} --- Former Postdoc, \href{https://www.linkedin.com/in/ialvarezillan/}{LinkedIn}
    \item \textbf{Aovida, Ronney} --- Former Undergraduate Student, \href{https://www.linkedin.com/in/ronney-aovida-411a01104/}{LinkedIn}
    \item \textbf{Ashki, Haleh} --- Former Graduate Student, \href{https://www.linkedin.com/in/haleh-ashki-a4427429/}{LinkedIn}
    \item \textbf{Ataman, Efe} --- Former Graduate Student
    \item \textbf{Ayoub, Alex} --- Former Undergraduate Student, \href{https://www.linkedin.com/in/alex-ayoub-212a41132/}{LinkedIn}
    \item \textbf{Azbill, Bryan} --- Former Graduate Student, \href{https://www.linkedin.com/in/bryan-azbill-a9003513b/}{LinkedIn}
    \item \textbf{Azoulay, Ariel} --- Former Graduate Student, \href{https://www.linkedin.com/in/ariel-azoulay-9566271b/}{LinkedIn}
    \item \textbf{Baker, Christopher} --- Former Graduate Student, \href{https://www.linkedin.com/in/christopher-baker-455b8535/}{LinkedIn}
    \item \textbf{Bartoldson, Brian} --- Former Graduate Student, \href{https://www.linkedin.com/in/brian-bartoldson-98585934/}{LinkedIn}
    \item \textbf{Berkley, Cameron} --- Former Graduate Student, \href{https://www.linkedin.com/in/cameron-berkley-37058285/}{LinkedIn}
    \item \textbf{Bertagna, Luca} --- Former Postdoc, \href{https://www.linkedin.com/in/lucabertagna/}{LinkedIn}
    \item \textbf{Bie, Ryan} --- Former Undergraduate Student
    \item \textbf{Birch, Kirby} --- Former Undergraduate Student, \href{https://www.linkedin.com/in/kirby-birch-53a547b7/}{LinkedIn}
    \item \textbf{Bishnu, Siddhartha} --- Former Graduate Student
    \item \textbf{Boehner, Philip} --- Former Graduate Student, \href{https://www.linkedin.com/in/philip-boehner-61a7a288/}{LinkedIn}
    \item \textbf{Bollig, Evan} --- Former Graduate Student, \href{https://www.linkedin.com/in/evan-bollig-11b01211/}{LinkedIn}
    \item \textbf{Boren, Seth} --- Former Graduate Student, \href{https://www.linkedin.com/in/seth-boren-b17b2b115/}{LinkedIn}
    \item \textbf{Bricker, Justin} --- Former Graduate Student, \href{https://www.linkedin.com/in/justin-bricker-65022849/}{LinkedIn}
    \item \textbf{Britton, Thomas} --- Former Undergraduate Student, \href{https://www.linkedin.com/in/thomas-britton-iii-743120b4/}{LinkedIn}
    \item \textbf{Burkovska, Olena} --- Former Postdoc, \href{https://www.linkedin.com/in/olena-burkovska-01932598/}{LinkedIn}
    \item \textbf{Bystricky, Lukas} --- Former Graduate Student, \href{https://www.linkedin.com/in/lukas-bystricky-354a3297/}{LinkedIn}
    \item \textbf{Chen, Xiao} --- Former Affiliate Graduate Student, \href{https://www.linkedin.com/in/xiao-chen-63200a26/}{LinkedIn}
    \item \textbf{Chen, Xi} --- Former Graduate Student
    \item \textbf{Cheung, James} --- Former Graduate Student, \href{https://www.linkedin.com/in/james-cheung-02018890/}{LinkedIn}
    \item \textbf{Chi, Yu-Chieh} --- Former Graduate Student
    \item \textbf{Chiu, Shang-Huan} --- Former Postdoc
    \item \textbf{Clement, Zachary} --- Former Undergraduate Student
    \item \textbf{Conry, Michael} --- Former Graduate Student
    \item \textbf{Cothrun, John} --- Former Graduate Student, \href{https://www.linkedin.com/in/john-cothrun-b8b26588/}{LinkedIn}
    \item \textbf{Cresswell-Clay, Evan} --- Former Graduate Student, \href{https://www.linkedin.com/in/evan-cresswell-clay-557345100/}{LinkedIn}
    \item \textbf{Crysup, Benjamin} --- Former Graduate Student, \href{https://www.linkedin.com/in/benjamin-crysup-51630a109/}{LinkedIn}
    \item \textbf{Dai, Heng} --- Former Graduate Student, \href{https://www.linkedin.com/in/heng-dai-07450519/}{LinkedIn}
    \item \textbf{Dearden, Albert} --- Former Postdoc, \href{https://www.linkedin.com/in/albert-dearden-67057088/}{LinkedIn}
    \item \textbf{Ding, Liangjing} --- Former Graduate Student
    \item \textbf{Dubey, Santosh} --- Former Graduate Student, \href{https://www.linkedin.com/in/santosh-dubey-b333a5b/}{LinkedIn}
    \item \textbf{Ehtemami, Anahid} --- Former Graduate Student, \href{https://www.linkedin.com/in/anahid-ehtemami-53b75487/}{LinkedIn}
    \item \textbf{Eovito, Austin} --- Former Graduate Student, \href{https://www.linkedin.com/in/austin-eovito-12b77a151/}{LinkedIn}
    \item \textbf{Fenn, Daniel} --- Former Graduate Student, \href{https://www.linkedin.com/in/daniel-fenn-192135118/}{LinkedIn}
    \item \textbf{Ficarra, Cody} --- Former Graduate Student, \href{https://www.linkedin.com/in/cody-ficarra-520336125/}{LinkedIn}
    \item \textbf{Fischer, Matthew} --- Former Undergraduate Student
    \item \textbf{Flaig, Markus} --- Former Postdoc, \href{https://www.linkedin.com/in/markusflaig/}{LinkedIn}
    \item \textbf{Forinash, Nicholas} --- Former Graduate Student, \href{https://www.linkedin.com/in/nicholas-forinash-a20463b2/}{LinkedIn}
    \item \textbf{Fratte, Daniel} --- Former Graduate Student, \href{https://www.linkedin.com/in/daniel-fratte-13837910/}{LinkedIn}
    \item \textbf{Gannon, Ashley} --- Former Graduate Student, \href{https://www.linkedin.com/in/ashley-gannon-a7966a152/}{LinkedIn}
    \item \textbf{Geshel, Christine} --- Former Undergraduate Student, \href{https://www.linkedin.com/in/christine-geshel-1049b494/}{LinkedIn}
    \item \textbf{Golden, Yusef} --- Former Undergraduate Student, \href{https://www.linkedin.com/in/yusef-golden-157790104/}{LinkedIn}
    \item \textbf{Green, Brett-Michael} --- Former Undergraduate Student, \href{https://www.linkedin.com/in/brett-michael-green-2253b7119/}{LinkedIn}
    \item \textbf{Grunthal, Mont} --- Former Undergraduate Student
    \item \textbf{Gu, Rui} --- Former Graduate Student, \href{https://www.linkedin.com/in/rui-gu-9a287957/}{LinkedIn}
    \item \textbf{Guan, Qingguang} --- Former Graduate Student, \href{https://www.linkedin.com/in/qingguang-guan-84955b25/}{LinkedIn}
    \item \textbf{Gusto, Brandon} --- Former Graduate Student
    \item \textbf{Handy, Timothy} --- Former Graduate Student \& Postdoc, \href{https://www.linkedin.com/in/timothy-handy-9a84a617/}{LinkedIn}
    \item \textbf{Harper, Mario} --- Former Graduate Student, \href{https://www.linkedin.com/in/mario-harper-04473b115/}{LinkedIn}
    \item \textbf{Henke, Steven} --- Former Graduate Student, \href{https://www.linkedin.com/in/steven-henke-b816087b/}{LinkedIn}
    \item \textbf{Ho, Minh} --- Former Undergraduate Student
    \item \textbf{Hogan, John} --- Former Graduate Student
    \item \textbf{Iglesias, Albert} --- Former Graduate Student
    \item \textbf{Jacobsen, Douglas} --- Former Graduate Student, \href{https://www.linkedin.com/in/douglas-jacobsen-5a98a89/}{LinkedIn}
    \item \textbf{Johnson, Brandon} --- Former Undergraduate Student, \href{https://www.linkedin.com/in/brandon-johnson-28b97163/}{LinkedIn}
    \item \textbf{Jordan, Cody} --- Former Undergraduate Student
    \item \textbf{Klion, Jarod} --- Former Graduate Student
    \item \textbf{Lambert, Mark} --- Former Graduate Student, \href{https://www.linkedin.com/in/mark-lambert-7058a984/}{LinkedIn}
    \item \textbf{Laseur, Lydia} --- Former Undergraduate Student
    \item \textbf{Latchireddy, Amar} --- Former Graduate Student
    \item \textbf{Lay, Nathan} --- Former Graduate Student, \href{https://www.linkedin.com/in/nathan-lay-3945a034/}{LinkedIn}
    \item \textbf{Learn, Ryan} --- Former Graduate Student, \href{https://www.linkedin.com/in/ryan-learn-33121588/}{LinkedIn}
    \item \textbf{Lees, Eitan} --- Former Graduate Student, \href{https://www.linkedin.com/in/eitan-lees-84b238104/}{LinkedIn}
    \item \textbf{Lei, Hongzhuan} --- Former Graduate Student, \href{https://www.linkedin.com/in/hongzhuan-lei-317201b1/}{LinkedIn}
    \item \textbf{Li Liu, Jerrison} --- Former Undergraduate Student
    \item \textbf{Llanos, Juan} --- Former Graduate Student, \href{https://www.linkedin.com/in/juan-llanos-537443a0/}{LinkedIn}
    \item \textbf{Lofman, Gwen} --- Former Undergraduate Student, \href{https://www.linkedin.com/in/gwen-lofman-2b992014b/}{LinkedIn}
    \item \textbf{Lopez, Nicolas} --- Former Graduate Student, \href{https://www.linkedin.com/in/nicolas-lopez-81781b89/}{LinkedIn}
    \item \textbf{Lorini, Michelle} --- Former Undergraduate Student
    \item \textbf{Lozano, Ian} --- Former Undergraduate Student, \href{https://www.linkedin.com/in/ian-lozano-73600985/}{LinkedIn}
    \item \textbf{Lu, Dan} --- Former Graduate Student, \href{https://www.linkedin.com/in/dan-lu-77677023/}{LinkedIn}
    \item \textbf{Lyngaas, Isaac} --- Former Graduate Student, \href{https://www.linkedin.com/in/isaac-lyngaas-783515bb/}{LinkedIn}
    \item \textbf{McCann, Ian} --- Former Graduate Student, \href{https://www.linkedin.com/in/ian-mccann-53a55734/}{LinkedIn}
    \item \textbf{McLaughlin, Ben} --- Former Graduate Student, \href{https://www.linkedin.com/in/ben-mclaughlin-05b16982/}{LinkedIn}
    \item \textbf{Mechtley, Alisha} --- Former Graduate Student, \href{https://www.linkedin.com/in/alisha-mechtley-ph-d-28562444/}{LinkedIn}
    \item \textbf{Miller, Geoffery} --- Former Graduate Student, \href{https://www.linkedin.com/in/geoffery-miller-775b484b/}{LinkedIn}
    \item \textbf{Miyar, Kathryn} --- Former Postdoc for Dr. Slice
    \item \textbf{Mohamed, Mamdouh} --- Former Graduate Student, \href{https://www.linkedin.com/in/mamdouh-mohamed-ph-d-p-e-9080081a/}{LinkedIn}
    \item \textbf{Mohebali, Behshad} --- Former Graduate Student, \href{https://www.linkedin.com/in/behshad-mohebali-ph-d-6691456a/}{LinkedIn}
    \item \textbf{Morgan, Brittany} --- Former Undergraduate Student
    \item \textbf{Mueller, Kevin} --- Former Graduate Student, \href{https://www.linkedin.com/in/kevin-mueller-43890214b/}{LinkedIn}
    \item \textbf{Nagales, Marcelina} --- Former Graduate Student, \href{https://www.linkedin.com/in/marcelina-nagales-608b8b80/}{LinkedIn}
    \item \textbf{Nason, Livia} --- Former Undergraduate Student, \href{https://www.linkedin.com/in/livia-nason-946766150/}{LinkedIn}
    \item \textbf{Nosowitz, Jonathon} --- Former Graduate Student, \href{https://www.linkedin.com/in/jonathannosowitz/}{LinkedIn}
    \item \textbf{Oliveto, Julia} --- Former Graduate Student, \href{https://www.linkedin.com/in/julia-oliveto-3a1197171/}{LinkedIn}
    \item \textbf{Palczewski, Michal} --- Former Graduate Student, \href{https://www.linkedin.com/in/michal-palczewski-208b868/}{LinkedIn}
    \item \textbf{Pendala, Yagna Sree Bhavani} --- Former Graduate Student
    \item \textbf{Pennington, Charles} --- Former Graduate Student
    \item \textbf{Perry Kuchera, Michelle} --- Former Graduate Student, \href{https://www.linkedin.com/in/michelle-perry-kuchera-56b9087/}{LinkedIn}
    \item \textbf{Peterson, Alexander} --- Former Graduate Student, \href{https://www.linkedin.com/in/alexander-peterson-a6331917b/}{LinkedIn}
    \item \textbf{Pettit, Jacob} --- Former Undergraduate Student, \href{https://www.linkedin.com/in/jacob-pettit-9524b8178/}{LinkedIn}
    \item \textbf{Pham, Serena} --- Former Graduate Student, \href{https://www.linkedin.com/in/serena-pham-33b66667/}{LinkedIn}
    \item \textbf{Pieper, Konstantin} --- Former Postdoc
    \item \textbf{Pomidor, Benjamin} --- Former Graduate Student, \href{https://www.linkedin.com/in/benjamin-pomidor-789a7442/}{LinkedIn}
    \item \textbf{Quinto, Jeremy} --- Former Graduate Student, \href{https://www.linkedin.com/in/jeremy-quinto-6110a5142/}{LinkedIn}
    \item \textbf{Reese, Jill} --- Former Postdoc, \href{https://www.linkedin.com/in/jill-reese-2516484/}{LinkedIn}
    \item \textbf{Robinson, David} --- Former Graduate Student, \href{https://www.linkedin.com/in/david-robinson-8a6239120/}{LinkedIn}
    \item \textbf{Rodriguez, Michael} --- Former Undergraduate Student
    \item \textbf{Romash, Brennan} --- Former Graduate Student, \href{https://www.linkedin.com/in/brennan-romash-9a6422b4/}{LinkedIn}
    \item \textbf{Rosales Giron, Daniel} --- Former Graduate Student
    \item \textbf{Sadeghi, Marjan} --- Former Graduate Student, \href{https://www.linkedin.com/in/marjansadeghi/}{LinkedIn}
    \item \textbf{Saha, Bikash} --- Former Graduate Student, \href{https://www.linkedin.com/in/bikash-saha-2a3a5530/}{LinkedIn}
    \item \textbf{Scarboro, Rachel} --- Former Undergraduate Student
    \item \textbf{Schama Lellis, Renata} --- Former Graduate Student
    \item \textbf{Schneier, Michael} --- Former Graduate Student, \href{https://www.linkedin.com/in/michael-schneier-51610497/}{LinkedIn}
    \item \textbf{Schueller, Kyle} --- Former Graduate Student, \href{https://www.linkedin.com/in/kyle-schueller-071a9916b/}{LinkedIn}
    \item \textbf{Seleson, Pablo} --- Former Graduate Student
    \item \textbf{Shaban Tameh, Maliheh} --- Former Graduate Student
    \item \textbf{Sharkey, Eric} --- Former Graduate Student, \href{https://www.linkedin.com/in/eric-sharkey-35b80916a/}{LinkedIn}
    \item \textbf{Sharma, Sarthak} --- Former Graduate Student, \href{https://www.linkedin.com/in/sarthak-sharma-636307135/}{LinkedIn}
    \item \textbf{Shaw, Kyle} --- Former Graduate Student
    \item \textbf{Smith, Danial} --- Former Graduate Student
    \item \textbf{Smith, Aria} --- Former Graduate Student, \href{https://www.linkedin.com/in/aria-g-smith-66385a53/}{LinkedIn}
    \item \textbf{Sockwell, Kenneth} --- Former Graduate Student, \href{https://www.linkedin.com/in/kenneth-sockwell-321151a1/}{LinkedIn}
    \item \textbf{Soda, James} --- Former Graduate Student, \href{https://www.linkedin.com/in/james-soda-2a781b83/}{LinkedIn}
    \item \textbf{Solimine, Philip} --- Former Graduate Student, \href{https://www.linkedin.com/in/philip-solimine-02b48995/}{LinkedIn}
    \item \textbf{Song, Xuehang} --- Former Postdoc, \href{https://www.linkedin.com/in/xuehang-song-3037893a/}{LinkedIn}
    \item \textbf{Steward, Jeffrey} --- Former Graduate Student, \href{https://www.linkedin.com/in/jeffrey-steward-66324b10/}{LinkedIn}
    \item \textbf{Stoyanova, Detelina} --- Former Graduate Student \& Postdoc, \href{https://www.linkedin.com/in/detelina-stoyanova-54157833/}{LinkedIn}
    \item \textbf{Sun, Huaiwei} --- Former Postdoc
    \item \textbf{Sung, Gi} --- Former Undergraduate Student
    \item \textbf{Takeh, Arsia} --- Former Graduate Student, \href{https://www.linkedin.com/in/arsia-takeh-b6205932/}{LinkedIn}
    \item \textbf{Teckentrup, Aretha} --- Former Postdoc
    \item \textbf{Tong, Xin} --- Former Postdoc, \href{https://www.linkedin.com/in/xin-tong-3980313a/}{LinkedIn}
    \item \textbf{Townsend, Stephen} --- Former Graduate Student, \href{https://www.linkedin.com/in/stephen-townsend-88225096/}{LinkedIn}
    \item \textbf{Tran, Toan} --- Former Graduate Student, \href{https://www.linkedin.com/in/toan-tran-6b3a03195/}{LinkedIn}
    \item \textbf{Van Popering, Luke} --- Former Graduate Student
    \item \textbf{Van Wyk, Hans-Werner} --- Former Postdoc, \href{https://www.linkedin.com/in/hvanwyk/}{LinkedIn}
    \item \textbf{Venkatesh, Kushal} --- Former Graduate Student, \href{https://www.linkedin.com/in/kushal-venkatesh-29896752/}{LinkedIn}
    \item \textbf{Wang, Yang} --- Former Graduate Student
    \item \textbf{Wang, Jilu} --- Former Postdoc
    \item \textbf{Wang, Chaolun} --- Former Graduate Student, \href{https://www.linkedin.com/in/chaolun-wang-b4703a110/}{LinkedIn}
    \item \textbf{Wang, Jie} --- Former Graduate Student, \href{https://www.linkedin.com/in/jie-wang-53372221/}{LinkedIn}
    \item \textbf{Webster, Clayton} --- Former Graduate Student, \href{https://www.linkedin.com/in/clayton-webster-3b478313/}{LinkedIn}
    \item \textbf{Witman, David} --- Former Graduate Student, \href{https://www.linkedin.com/in/david-witman-49931b74/}{LinkedIn}
    \item \textbf{Womeldorff, Geoffrey} --- Former Graduate Student, \href{https://www.linkedin.com/in/geoffrey-womeldorff-48425211/}{LinkedIn}
    \item \textbf{Woods, Jonathan} --- Former Undergraduate Student
    \item \textbf{Xu, Feifei} --- Former Graduate Student, \href{https://www.linkedin.com/in/feifei-xu-373b9840/}{LinkedIn}
    \item \textbf{Yannakopoulos, Anna} --- Former Undergraduate Student
    \item \textbf{Young, Andrew} --- Former Graduate Student, \href{https://www.linkedin.com/in/andrew-young-9029a544/}{LinkedIn}
    \item \textbf{Yuan, Haiping} --- Former Graduate Student
    \item \textbf{Zaman, Sayem} --- Former Postdoc
    \item \textbf{Zhang, Jingze} --- Former Graduate Student, \href{https://www.linkedin.com/in/jingze-zhang-669299150/}{LinkedIn}
    \item \textbf{Zhang, Guannan} --- Former Graduate Student
    \item \textbf{Zhao, Wenju} --- Former Graduate Student, \href{https://www.linkedin.com/in/wenju-zhao-80126a117/}{LinkedIn}
    \item \textbf{Zheng, Xueming} --- Former Graduate Student, \href{https://www.linkedin.com/in/xueming-zheng-76506412b/}{LinkedIn}
    \item \textbf{Zhu, Yan} --- Former Postdoc
\end{itemize}

\section{Testimonials}
\texttt{\url{https://www.sc.fsu.edu/people/students/testimonials}}

\begin{quote}
"Thank you for teaching me R! Since completing my Masters a few years ago I’ve been working in business analytics. The last year and a half with a movie theater chain mining their customer databases and performing website experiments to help improve the business. I got that job 100\% because of a report I put together using R."
\par\raggedleft\textit{--- GF}
\end{quote}

\begin{quote}
"I was surrounded all the time by the most extraordinary and intelligent people. The best experience of my life."
\par\raggedleft\textit{--- Daniel Fratte}
\end{quote}

\begin{quote}
"I feel like the training that I got at this department was perfectly suited for the position I have now. ...I know enough [physics, math and computer science] to function and contribute in circles with people who are experts and be the glue between those disciplines."
\par\raggedleft\textit{--- Geoffrey Womeldorff}
\end{quote}

\begin{quote}
"I had a wonderful five years at the Department of Scientific Computing!"
\par\raggedleft\textit{--- Dan Lu}
\end{quote}

\begin{quote}
"My experience at the Department of Scientific Computing was exceptional."
\par\raggedleft\textit{--- Myrna Merced-Serrano}
\end{quote}

\subsection{Contribute a Testimonial}
\texttt{\url{https://www.sc.fsu.edu/people/students/testimonials/add}}\\
Please Email your Testimonial to the Webmaster.

\section{Email Lists and Groups}
\texttt{\url{https://www.sc.fsu.edu/people/lists}}

The following lists are currently being served by FSU lists server. Most of these mailing lists, unless otherwise noted, are reached by mailing \texttt{listname@lists.fsu.edu}. Please email \texttt{support@sc.fsu.edu} for any list modifications.

\subsection{General Lists}
\begin{description}
    \item[\texttt{all@sc.fsu.edu}] ($\supseteq$ of admin@, faculty@, postdocs@, grad\_student@, ug@, affiliated-faculty@)
    \item[\texttt{4th-floor@lists.fsu.edu}] 4th-Floor Email List (Mailman)
    \item[\texttt{tsg@sc.fsu.edu}] Technical Support Email List (AD: dl-sc-tsg)
    \item[\texttt{admin@sc.fsu.edu}] ($\supseteq$ of tsg@) Administration Email List (+ Dept. Chair) (AD: dl-sc-adm)
    \item[\texttt{faculty@sc.fsu.edu}] Faculty Email List (AD: dl-sc-faculty)
    \item[\texttt{postdocs@sc.fsu.edu}] Postdoc Email List (Mailman)
    \item[\texttt{grad\_students@sc.fsu.edu}] (alias: \texttt{grads@}, \texttt{grad@}, \texttt{grad\_student@}) Graduate Student Email List (Mailman)
    \item[\texttt{ug@sc.fsu.edu}] Undergraduate Student Email List (Mailman)
    \item[\texttt{affiliated-faculty@sc.fsu.edu}] Affiliated Faculty Email List (Mailman)
    \item[\texttt{alumni@lists.fsu.edu}] DSC Alumni List (Mailman)
\end{description}

\subsection{Committees}
\begin{description}
    \item[\texttt{grad-committee@sc.fsu.edu}] Graduate Committee Email List (AD: SC-Graduate-Committee)
    \item[\texttt{ug-committee@sc.fsu.edu}] Undergraduate Committee Email List (AD: SC-Undergraduate-Committee)
    \item[\texttt{exec-committee@sc.fsu.edu}] Executive Committee Email List (AD: SC-Executive-Committee)
    \item[\texttt{promo-committee@sc.fsu.edu}] Promotion Committee Email List (AD: SC-Promotion-Committee)
    \item[\texttt{research-committee@sc.fsu.edu}] Research Committee Email List (AD: SC-Research-Committee)
    \item[\texttt{recruit@sc.fsu.edu}] Faculty Search Committee (AD: dl-sc-recruit)
    \item[\texttt{gpc-physrev-external@sc.fsu.edu}] Graduate Policy Committee subcommittee (non-physics, external) [Plewa 2023] (AD: dl-sc-gpc-physrev-external)
    \item[\texttt{gpc-physrev@sc.fsu.edu}] Graduate Policy Committee subcommittee (all members) [Plewa 2023] (AD: dl-sc-gpc-physrev)
\end{description}

\subsection{Other Lists}
\begin{description}
    \item[\texttt{seminar@lists.fsu.edu}] Seminar Announce Mailing list
    \item[\texttt{morphlab@lists.fsu.edu}] Dr. Slice's Morphlab Mailing list
    \item[\texttt{logs@sc.fsu.edu}] System Logs
    \item[\texttt{ml@sc.fsu.edu}] Machine Learning (ML) Group discussion list
\end{description}

\subsection{Legacy Lists}
\begin{description}
    \item[\texttt{mbadmin@lists.fsu.edu}] Mbadmin Mailing list
\end{description}

\subsection{Aliases Controlled by FSU AD}
\begin{description}
    \item[\texttt{sc-advising@fsu.edu}] Academic Advising
    \item[\texttt{sc-billing@fsu.edu}] (\texttt{sc-invoices@fsu.edu}) Internal Billing/Invoice Notices
    \item[\texttt{sc-connect@fsu.edu}] (\texttt{sc-portal@fsu.edu}) connect.fsu.edu portal
    \item[\texttt{sc-education@fsu.edu}] Education Group
    \item[\texttt{sc-info@fsu.edu}] Education Information
    \item[\texttt{sc-pr@fsu.edu}] Public Relations
    \item[\texttt{sc-premajor@fsu.edu}] Pre-major Undergraduate Students
    \item[\texttt{sc-social@fsu.edu}] Social Media Team
    \item[\texttt{sc-vislab@fsu.edu}] Visualization Lab
    \item[\texttt{sc-search@fsu.edu}] Search TBD
\end{description}

\section{Student Groups}
URL: https://www.sc.fsu.edu/student-groups
\newline

\href{https://nolecentral.dsa.fsu.edu/home_login}{Click Here} to Register an Organization with Nole Central

```latex
\section{Chair's Introduction}
% https://www.sc.fsu.edu/chairs-intro

\begin{center}
    Welcome to the \\
    \textbf{Department of Scientific Computing} \\
    at \\
    \textbf{Florida State University!}
\end{center}

When I became a scientist many years ago, computers were still very special for most of us, and only a few would train and use computers for research. This has changed profoundly! Today, we cannot imagine doing science without computers. Most mundane tasks are now run on computers; our cell phones have morphed into mobile pocket computers with power many times beyond the mainframe computers of the 1980s. In 2022, digital computers have matured and become part of our society. The future may bring versatile quantum computers soon, replacing our current computers.

Today computing and theory development often go hand-in-hand, and our department of Scientific Computing is at this junction where theory and computation intersect. Several in our department work on modeling processes and improving the precision and calculation time to achieve accurate predictions of the actual processes. These include work on detecting tumors, explaining the flow of forest fires, or describing details of star explosions or evolutionary patterns.

Our department is interdisciplinary at its core: as of 2022, our faculty consists of mathematicians, physicists, biologists, engineers, material scientists, and computer scientists. We use computers as a tool to improve theory, calculations, modeling, and inference. Our educational mission includes an undergraduate program in computational science that emphasizes the interaction of mathematics, computing, and problem-solving. We also have three graduate programs: a doctoral program in computational science, a master's program in computational science, and a master's program in datascience. The datascience program is a collaboration with 3 other departments on campus.

Explore our website, and learn about our computational science and datascience programs. Feel free to contact us to establish a dialogue. In particular, we would like to hear from you if you are a prospective student seeking to participate in our Departmental programs and activities and become a part of a discipline that changes the world.

\vspace{1em}

\begin{flushright}
    Sincerely, \\
    Peter Beerli \\
    Professor and Chair
\end{flushright}
```

\section{Job Opportunities}
\url{https://www.sc.fsu.edu/jobs}

\vspace{1em}

We are always looking for new collaborators and new students. We invite you to contact our \href{https://www.sc.fsu.edu/people/faculty}{faculty} to discuss the opportunities for you. Right now the following positions are open at the Department of Scientific Computing.

\section{Job Placements}
\url{https://www.sc.fsu.edu/job-placements}

The following list of companies, institutions, and government labs shows the variety and caliber of jobs our students are landing with a degree in \textit{Computational Science} from \textit{Florida State University}.

\begin{center}
\onehalfspacing
\begin{tabular}{l p{8cm} l}
\rowcolor{gray!25}
\textbf{Degree} & \textbf{Company / Institution / Lab} & \textbf{Location} \\ 
\hline \hline
- & Advanced Cooling Technologies, Inc. & Lancaster, PA \\
Ph.D. & Aerojet Rocketdyne & Huntsville, AL \\
Ph.D. & Arizona State University & Tempe, AZ \\
- & Bank of America & Charlotte, NC \\
Ph.D. & Cairo University & Giza, Egypt \\
M.S. & Carnegie Mellon University & Pittsburgh, PA \\
Ph.D. & Cerner & Kansas City, MO \\
M.S. & codeslam & Tallahassee, FL \\
Ph.D. & Datamaxx Group & Tallahassee, FL \\
- & Digitalsmiths & Durham, NC \\
B.S. & Epic Careers & Verona, WI \\
- & Facebook & Menlo Park, CA \\
B.S. & Fidelity Investments & Durham, NC \\
- & First Data & Palo Alto, CA \\
- & Florida Department of Transportation & Tallahassee, FL \\
M.S. & Florida State University & Tallahassee, FL \\
- & General Motors & Milford, MI \\
- & Google & Mountain View, CA \\
- & Infinity Software Development, Inc. & Tallahassee, FL \\
M.S. & Johns Hopkins University & Baltimore, MD \\
Ph.D. & JPL: Radar Science \& Engineering & Los Angeles, CA \\
Ph.D. & King Abdullah University of Science and Technology & Saudi Arabia \\
M.S. & Lawrence Berkeley National Laboratory & Berkeley, CA \\
M.S. & Lowes & Mooresville, NC \\
- & Los Alamos National Laboratory & New Mexico \\
- & Lucidworks & San Francisco, CA \\
- & Lyft & San Francisco, CA \\
Ph.D. & Minnesota Supercomputing Institute & Minneapolis, MN \\
- & NantHealth & Panama City, FL \\
- & NASA Jet Propulsion Lab & Pasadena, CA \\
- & National Institute of Health & Bethesda, MD \\
Ph.D. & National Superconducting Cyclotron Laboratory & East Lansing, MI \\
Ph.D., M.S., B.S. & NewSci / NewSci Labs & Tallahassee, FL \\
M.S. & Nokia Technologies & Sunnyvale, CA \\
M.S., Ph.D. & Oak Ridge National Laboratory & Oak Ridge, TN \\
M.S. & outbrain & Cambridge, MA \\
- & Pacific Northwest National Laboratory & Richland, WA \\
M.S. & Sony, Playstation & San Mateo, CA \\
M.S. & Platfora & San Mateo, CA \\
M.S., Ph.D. & Purdue University & West Lafayette, IN \\
- & Raytheon & Largo, FL \\
- & Science Applications International Corporation & San Diego, CA \\
Ph.D. & Smarter Data & San Ramon, CA \\
B.S. & St. Jude Children's Research Hospital & Memphis, TN \\
- & Stinger Ghaffarian Technologies, Inc. (SGT) & Greenbelt, MD \\
M.S. & Strong Analytics LLC & Chicago, IL \\
- & Synopsys, Inc. & Marlborough, MA \\
- & Tall Timbers & Tallahassee, FL \\
Ph.D. & Tinder & Palo Alto, CA \\
- & Torch Technologies & Huntsville, AL \\
M.S. & U.S. Environmental Protection Agency & Raleigh-Durham, NC \\
Ph.D. & University of California, Los Angeles (UCLA) & Los Angeles, CA \\
Ph.D. & University of California, Santa Barbara (UCSB) & Santa Barbara, CA \\
Ph.D. & University of Michigan & Ann Arbor, MI \\
Ph.D. & University of Petroleum and Energy Studies & Dehradun, India \\
Ph.D. & University of Texas at Austin & Austin, TX \\
M.S. & Vertisystem Inc. & Fremont, CA \\
M.S. & Visual.ly & California \\
\hline
\end{tabular}
\end{center}

\section{Degree Programs Overview}
\texttt{url: https://www.sc.fsu.edu/graduate/ms}

The educational mission of the Department of Scientific Computing (DSC) is to provide innovative, interdisciplinary undergraduate and graduate training programs in computational science and its applications. The graduate and undergraduate degree programs in the DSC are designed to provide students with a broad training in the design, implementation, and use of algorithms for solving science and engineering problems on computers.

\subsection*{The Bachelor of Science (B.S.) degree}
The Department of Scientific Computing offers an innovative \textbf{Bachelor of Science (B.S.) degree program in Computational Science}.

This degree program should be of interest to and is well suited for those who like working on computers and who ordinarily would also be interested in any of the mathematical sciences (mathematics, computer science, statistics), or any of the natural sciences (biology, chemistry, physics, geological sciences, ...), or any engineering discipline. It should be of special interest to students interested in two or more of these areas. Students majoring in Computational Science will learn how to develop and apply new computational tools to solve science and engineering problems.

Please note that \textit{computational science} is \textit{different} from \textit{computer science}. At the risk of oversimplifying things, one can say that computer science is about the science of computers whereas computational science is about the use of computers to solve science and engineering problems.

\subsection*{The Minor degree}
The Minor in Computational Science offers a substantive programming and algorithmic knowledge base to non-Computational Science students. Students who minor in the discipline will develop critical computing and modeling skills that are marketable and attractive to potential employers.

The minor requires at least 14 hours of coursework, and students must make a C- or above in each class for the course to be accepted for minor credit.

\subsection*{The Master of Science (M.S.) degrees}
\begin{itemize}
    \item M.S. in Computational Science
    \item M.S. in Data Science
\end{itemize}

\subsection*{The Doctor of Philosophy (Ph.D.) degree consists of several majors}
\begin{itemize}
    \item Ph.D. in Computational Science
    \item with a Specialization in \textbf{Atmospheric Science}
    \item with a Specialization in \textbf{Biochemistry}
    \item with a Specialization in \textbf{Biological Science}
    \item with a Specialization in \textbf{Geological Science}
    \item with a Specialization in \textbf{Materials Science}
    \item with a Specialization in \textbf{Physics}
\end{itemize}
Students who choose the first major specialize in the more mathematical or computer science aspects of computational science.

Because computational science lies at the intersection of applied mathematics, applied science, engineering, and computer science, the DSC has the unique ability to offer coursework and research opportunities at all levels in topic areas that cut across disciplines. The focus of the research and training activities of the DSC is on the invention, analysis, and implementation of computational algorithms that transcend disciplines and the application of such algorithms to various applications, including, among others, astrophysics, bioinformatics, climate and weather modeling, computational fluid mechanics, computational geometry, computer game design, evolutionary biology, data mining, GPU computing, high-energy density physics, high-performance computing, hydrology, machine learning, material science, medical imaging, morphometrics, nano-materials, numerical analysis, partial differential equations, phylogenetics, polymers, population genetics, scientific visualization, subsurface environmental modeling, superconductivity, systems biology, and uncertainty quantification.

Unlike the faculty of a typical department whose training and research interests lie solely within a single discipline, the faculty of the DSC includes members trained in a variety of disciplines - biology, aeronautical engineering, chemical engineering, geological sciences, electrical engineering, material sciences, mathematics, nuclear engineering, and physics. In addition, the DSC has several affiliated faculty members in other departments at FSU and at other universities, companies, and laboratories that further enhance the educational and research experiences of students in the DSC's degree programs.

In addition to offering degree programs, the DSC contributes to FSU's educational mission by offering courses of interest to students in other departments, by offering tutorials and short courses, and by managing high-performance computational resources available to the campus community.

\section{Undergraduate Programs}

\subsection{Undergraduate Overview}
\texttt{url: https://www.sc.fsu.edu/undergraduate/overview}

The Department of Scientific Computing offers an innovative \textbf{B.S. degree program in Computational Science}.

This degree program should be of interest to and is well suited for those who like working on computers and who ordinarily would also be interested in any of the mathematical sciences (mathematics, computer science, statistics), or any of the natural sciences (biology, chemistry, physics, geological sciences, ...), or any engineering discipline. It should be of special interest to students interested in two or more of these areas. Students majoring in Computational Science will learn how to develop and apply new computational tools to solve science and engineering problems.

The \textbf{Department of Scientific Computing} is the focal point of computational science activities at FSU. Computational science involves the invention, implementation, testing, and application of algorithms and software used to solve large-scale scientific and engineering problems. The Department provides a venue for innovative research and education in computational science. Our faculty come from a wide variety of traditional mathematical, scientific, and engineering disciplines who are primarily interested in developing knowledge and tools for computational science and applying those tools to the solution of problems in a variety of applications.

Please note that computational science is different from computer science. At the risk of oversimplifying things, one can say that computer science is about the science of computers whereas computational science is about the use of computers to solve science and engineering problems.

The undergraduate \textbf{Minor} and \textbf{B.S.} degree programs in Computational Science, along with the already established \textbf{Master's} and \textbf{Ph.D.} programs, trains students in computational science by providing them with a mix of theory and practice, including substantial hands-on experience in the deployment and use of algorithms for solving scientific and engineering problems.

Questions should be sent to \texttt{<email protected>}. Anyone in the neighborhood interested in knowing more about the undergraduate degree program is welcome to stop by the Department offices located on the 4th floor of the Dirac Science Library.

Information about the Department of Scientific Computing is found at \texttt{http://sc.fsu.edu}. We are also on Facebook and Twitter.

\subsection{Bachelor of Science (B.S.) in Computational Science}
\texttt{url: https://www.sc.fsu.edu/undergraduate/bachelor-of-science}

How to \textit{change your major}, \textit{add a second major}, or \textit{seek a dual degree}.

\subsubsection*{Description of Major}
Computational Science is the emerging discipline that provides the tools necessary to solve natural and social science and engineering problems on computers, much the same as mathematics provides the underpinning for theoretical solutions. Graduates of the Computational Science program will be prepared for employment in industry and government laboratories as well as for entry into graduate schools. The skill and knowledge sets acquired by students in this program compliment those of scientists and engineers so that, as part of interdisciplinary teams, those students are ideally positioned to help solve science and engineering problems using computers.

\paragraph{Note to prospective transfer students:} Prospective transfer students should contact \texttt{<email protected>} (Arts \& Sciences Admissions) with specific questions about admission and mapping requirements.

\subsubsection*{Prerequisite Coursework}
The following are proposed as common program prerequisites:
\begin{itemize}
    \item MAC 2311 (4) Calculus I
    \item MAC 2312 (4) Calculus II
    \item ISC 3313 (3) Introduction to Scientific Computing or COP XXXX (3) Introductory programming (in C or C++, JAVA or equivalent language; COP 3014 Programming I at FSU) or other approved programming course.
    \item Science with lab (4) A laboratory-based science course designed for science majors (BSC, CHM, GLY, MET, or PHY).
\end{itemize}
\textbf{Note:} State-wide common prerequisites are always under review. For the most current information and for acceptable alternative courses, visit the “Common Prerequisites Manual.” This is available from the “Student Services” section of \texttt{http://www.flvc.org}.

\subsubsection*{Requirements}
\paragraph{Requirements for graduation in the College of Arts and Sciences include:} The College of Arts and Science requires proficiency in a foreign language through the intermediate (2220 or equivalent) level or sign language through the advanced (2614 or equivalent) level.

\paragraph{Requirements for Progression to the Upper-Division Major:} Students must also have completed a minimum of 52 hours of credit and at least half the required hours in General Education including the required two English and two mathematics courses, or an A.A degree.

\paragraph{Departmental Policy on Grades and Continuation in the Major:} A grade of C minus or better is required in all courses required for the BS Degree in Computational Science. A student who has received more than five unsatisfactory grades (U, F, D-, D+) in computational science, any of the sciences, mathematics, statistics, computer science, or engineering courses taken at Florida State University or elsewhere, including repeated unsatisfactory grades in the same required course, will not be permitted to graduate with a degree in computational science.

\subsubsection*{Major Program of Studies at FSU (54 hours)}
\textbf{All of the following (30 hours)}
\begin{itemize}
    \item ISC 3222 (3) Symbolic and Numerical Computations
    \item ISC 4304 (4) Programming for Scientific Applications
    \item ISC 4220 (4) Continuous Algorithms for Science Applications
    \item ISC 4221 (4) Discrete Algorithms for Science Applications
    \item ISC 4223 (4) Computational Methods for Discrete Problems
    \item ISC 4232 (4) Computational Methods for Continuous Problems
    \item MAS 3105 (4) Applied Linear Algebra I
    \item Approved statistics course designed for statistics majors: STA 3XXX (3) or STA 4XXX (3)
\end{itemize}
\textbf{Seminar Classes (3 hours)}
\begin{itemize}
    \item Three seminars are required: specialized topics and/or new developments in computational sciences.
    \item ISC 4931r (1,1) Junior Seminar in Computational Science
    \item ISC 4932r (1) Senior Seminar in Computational Science
\end{itemize}
\textbf{Practicum (3 hours)}
\begin{itemize}
    \item ISC 4943 (3) Practicum in Computational Science
\end{itemize}
\textbf{Electives (18 hours)}
\begin{itemize}
    \item 9 hours of computational science courses selected from an approved list.
    \item 9 hours of additional electives from the Department of Scientific Computing or other appropriate department at the 3000/4000 level or from approved list. Students should consult the academic advisor regarding the selection of electives.
\end{itemize}

\subsubsection*{Collateral Courses: (26 hours)}
The following courses are required. These may also be used to satisfy prerequisites, General Education or minor requirements.
\begin{itemize}
    \item ISC 3313 (3) Introduction to Scientific Computing or COP XXXX (3) Programming course
    \item MAC 2311 (4) Calculus I
    \item MAC 2312 (4) Calculus II
    \item Two laboratory-based science courses (8 hrs) designed for science majors: BSC, CHM, GLY, MET, or PHY.
\end{itemize}

\paragraph{Minor: (0-12+ hours beyond other requirements)} A minor is required. If not also used to meet the General Education mathematics requirement, the required collateral courses MAC 2311, MAC 2312, and MAS 3105 will satisfy the requirements for a minor in mathematics. The student may select another minor in consultation with the program advisor.

\paragraph{Computer Skills Competency: (0 hours beyond other requirements)} ISC 3313 Introduction to Scientific Computing or COP 3014 Programming I fulfill this requirement for the major.

\paragraph{Oral Communication Competency: (0-3 credits hours)} Students must demonstrate the ability to orally transmit ideas and information clearly. This requirement may be met with an approved college-level course.

\subsubsection*{Minimum Program Requirements - Summary}
\begin{itemize}
    \item Total Hrs. Required 120
    \item General Education 36*
    \item Prerequisite Coursework 15* (also count for collateral coursework)
    \item Major Coursework 54
    \item Collateral Coursework 26*
    \item Minor Coursework 0-12 or more*
    \item Foreign Language 0-12 (depending on placement)
    \item Computer Skills 0-3 beyond major
    \item Oral Communication Competency 0-3
    \item Electives to bring total hours to 120
\end{itemize}
\textbf{Note:} Some prerequisite and/or collateral coursework may also be applied to General Education requirements in math and science or the minor.

\subsubsection*{Mapping}
Mapping is FSU’s academic advising and monitoring system. Academic progress is monitored each Fall and Spring semester to ensure that students are on course to earn their degree in a timely fashion. Transfer students must meet mapping guidelines to be accepted into their majors. You may view the map for this major at \texttt{www.academic-guide.fsu.edu/}

\subsubsection*{Remarks}
\begin{itemize}
    \item A minimum of 45 hours at the 3000 level or above, 30 of which must be taken at this University.
    \item Half of the major course semester hours must be completed in residence at this University.
    \item The final 30 hours must be completed in residence at this University.
\end{itemize}

\subsubsection*{Employment Information}
\paragraph{Salary Information:} National Association of Colleges and Employers, Occupational Outlook Handbook
\paragraph{Representative Job Titles Related to this Major:} Baccalaureate level-computational scientist, computer center consultant, graduate student, programmer, computer modeler. Master of Science level-computational scientist, computer center consultant, Ph.D student, programmer, software engineer, computer modeler. With Ph.D-computational scientist, professor, postdoctoral associate, research scientist, research associate, computer modeler, research group leader.
\paragraph{Representative Employers:} Universities and college computer centers, government and private research laboratories, manufacturing and service businesses, industry, state and federal government agencies.

\subsubsection{Degree In Three}
\texttt{url: https://www.sc.fsu.edu/undergraduate/bachelor-of-science/degree-in-three}

Degree In Three is a special program designed to assist students who wish to graduate in three years or less. While all students are welcome to explore this path, Degree in Three is typically best for students who enter Florida State with college credit earned through AP, IB, AICE, or dual enrollment. Developed in the Division of Undergraduate Studies, Degree in Three is now managed in the Graduation Planning and Strategies Office.

For more information, please contact our Academic Advisor.

\subsubsection{Honors in the Major}
\texttt{url: https://www.sc.fsu.edu/undergraduate/bachelor-of-science/honors-in-the-major}

Students earn Honors in the Major by completing a research thesis or creative project in your major area of study under the guidance of a faculty committee. You will work with this committee to select a topic, develop a prospectus, complete a written document based on your research or creative project, and defend your thesis orally before your committee. This process normally takes two to three semesters, during which you will register for six to nine hours of 4000-level thesis credit. Once you have completed and successfully defended your project, you will graduate "with honors" in your major, a distinction that is announced during commencement and designated on your official transcript (not your diploma).

For more information, please contact our Academic Advisor.

\subsection{Minor in Computational Science}
\texttt{url: https://www.sc.fsu.edu/undergraduate/minor}

The Minor in Computational Science offers a substantive programming and algorithmic knowledge base to non-Computational Science students. Students who minor in the discipline will develop critical computing and modeling skills that are marketable and attractive to potential employers.

A minor in computational science requires a minimum of 14 hours of coursework, including ISC3222 (3) and ISC4304 (4). The student must take at least one Computational Science Algorithms course (ISC4220 or ISC4221 (4)) and a Computational Science course from an approved list. Students must also satisfy stated prerequisites, in particular, the computer programming requirement (ISC3313 or COP3014), before enrolling in each course accepted for minor credit. Grades below C- will not be accepted for minor credit.

\paragraph{Contact Information:} \texttt{<email protected>}, (850) 644-0143

\subsubsection*{Course Requirements}
Students must satisfy prerequisites before enrolling in each course.
\paragraph{Complete \textbf{both} of the following courses:}
\begin{itemize}
    \item ISC 3222 - Symbolic and Numerical Computations
    \item ISC 4304 - Programming for Scientific Applications
\end{itemize}

\paragraph{Complete at least \textbf{one} of the following courses:}
\begin{itemize}
    \item ISC 4220 - Continuous Algorithms for Science Applications (4 credits)
    \item ISC 4221 - Discrete Algorithms for Science Applications (4 credits)
\end{itemize}

\paragraph{Complete at least \textbf{one} of the following courses:}
\begin{itemize}
    \item ISC 4220 - Continuous Algorithms for Science Applications (4 credits) or
    \item ISC 4221 - Discrete Algorithms for Science Applications (4 credits)
    \item ISC 4223 - Computational Methods for Discrete Problems (4 credits)
    \item ISC 4232 - Computational Methods for Continuous Problems (4 credits)
    \item ISC 4933 - Geometric Morphometrics (3 credits)
    \item ISC 4933 - Computational Evolutionary Biology (4 credits)
    \item ISC 4933 - Data Mining (3 credits)
    \item ISC 4933 - Genomic Sequencing and Analysis (4 credits)
    \item ISC 4302 - Scientific Visualization (3 credits)
    \item ISC 4933 - Verification and Validation in Computational Science (3 credits)
    \item DIG 3725 - Introduction to Game and Simulator Design
    \item ISC 4933 Any Special Topic Course (3 credits)
\end{itemize}

\section{Graduate Programs}

\subsection{Graduate Overview}
\texttt{url: https://www.sc.fsu.edu/graduate/overview}

The Department of Scientific Computing offers innovative \textbf{M.S. in Data Science}, \textbf{M.S. in Computational Science}, and \textbf{Ph.D. in Computational Science}.

The Ph.D. program allows specialization in the following areas:
\begin{itemize}
    \item atmospheric science, biochemistry, biological science, geological science, materials science, physics.
    \item as well as in aspects of \textit{computational mathematics and computer science relevant to computational science}.
\end{itemize}

The goal of the \textbf{Ph.D. degree in Computational Science} is to train graduate students to have extensive knowledge in computational science and to allow them to acquire expertise in one or more areas of applied mathematical sciences or engineering. Thus the degree provides the student with breadth as well as depth. Graduates should be able to successfully collaborate with scientists in other disciplines. Ideally, students should learn to develop, use, and analyze new computational procedures that can be utilized in a variety of fields.

Questions should be sent to \texttt{<email protected>}. Anyone in the neighborhood interested in knowing more about the graduate degree programs is welcome to stop by the Department offices located on the 4th floor of the Dirac Science Library.

\subsection{Master of Science (M.S.) in Computational Science}
\texttt{url: https://www.sc.fsu.edu/graduate/ms/computational-science}

In today's economy, there is an increasing need for students who possess scientific and computational expertise as well as management and leadership skills. All students entering this master's degree program should have a strong desire to do computational science. The Master's Degree in Computational Science gives students the tools needed for landing a sucessfull job after graduation.

\subsubsection*{Coursework and Credits}
The coursework for the M.S. degree program is based on \textbf{30 credit hours} for the major track (standard M.S. in Computational Science). Each student must take two core computational science courses (Group A courses) as well as a minimum of 9 credit hours in other computational science courses (Group B courses) plus 6 additional credit hours selected from existing departmental courses in computer science, engineering, mathematics or an applied science (Group C courses). Each student must also take \textbf{2 seminar hours} (details are listed in the Graduate Handbook, Page 10, section 5.1.3).

\paragraph{Required Core Courses (Group A)}
\paragraph{Elective Core Courses (Group B)}

\subsubsection*{Master's tracks}
We have established the following track(s) for the M.S. degree program:
\begin{itemize}
    \item M.S. in Computational Science
    \item M.S. in Data Science
\end{itemize}

\subsubsection*{Application \& Information}
The program is open to students holding a bachelor's degree in mathematical and physical sciences and engineering. You can apply for the graduate programs online.

For further information concerning this M.S. program at FSU, please email \texttt{<email protected>}. Please check back periodically to this website for updates to the program.

\subsection{Master of Science (M.S.) in Data Science}
\texttt{url: https://www.sc.fsu.edu/graduate/ms/datascience}

\subsubsection*{About - M.S. in Data Science}
We are excited to offer a new 30-credit MS degree in Interdisciplinary Data Science (IDS) with a major in Scientific Computing beginning in Fall 2021!

Data is exploding all around us and will keep growing at dizzying rates for the foreseeable future in fields as unique as health, material science, finance, biology, the internet of things, experimental physics, law, art, and psychology, along with many others. Tools of choice to analyze these data revolve around model reduction, probability, data, and machine learning. On the demand side, industry, government, and national labs are clamoring for a workforce skilled in artificial intelligence, machine learning, predictive analysis, and data modeling. Our students will graduate with a Master’s degree, preparing them to succeed in a world where computation, mathematics, and statistics have become fundamental necessities. They will create the next generation computational algorithms necessary to model, manipulate, analyze, visualize, and transform unstructured, structured, static, and streaming data.

A core set of Statistics, Mathematics, Data Mining, Machine Learning, Data Science, and Data Ethics courses will create a strong core of fundamentals upon which to build. Students will further specialize and strengthen their skills through electives taken in Scientific Computing and the other three majors (Statistics, Mathematics, Computer Science), such as High-Performance Computing, Cloud Computing, and Probabilistic Programming.

\subsubsection{Admission}
\texttt{url: https://www.sc.fsu.edu/graduate/ms/datascience/admission}

The IDS graduate program will have broad appeal to students with undergraduate degrees in math, computer science, or statistics, but will also attract students with less traditional backgrounds, e.g., engineering, physics, etc.

In addition to meeting all of the University and College admission requirements for graduate study, each applicant for the MSIDS program must:
\begin{itemize}
    \item have earned a Bachelor’s degree from an accredited institution
    \item have a GRE score of 146 Verbal and 155 Quantitative for GREs. \textbf{GRE is currently waived for the Master's admission.}
    \item have a minimum of 3.0 GPA (B or better average) on the last 60 hours of undergraduate credits; and
    \item be in good standing at the institution of higher learning last attended
    \item provide three letters of recommendation discussing the student’s aptitude for graduate study
    \item provide a statement of intent describe why you wish to join the program
\end{itemize}

\paragraph{Required Background:}
\begin{itemize}
    \item Calculus 2 (MAC2312 or equivalent)
    \item College algebra
    \item Basic probability and statistics (STA 2023 or equivalent)
    \item Experience in one or more object-oriented programming languages such as C++, Python, Julia, and Matlab
\end{itemize}
Admission to some of the more advanced elective courses comprising the different majors may require additional prerequisites, which the students will either have to demonstrate in their backgrounds or take additional coursework.

\subsubsection{Faculty}
\texttt{url: https://www.sc.fsu.edu/graduate/ms/datascience/faculty}

\paragraph{Data Science Program Contact / Academic Advising}
\begin{itemize}
    \item \textbf{Clark, Jennifer} \\ Administrative Director, Interdisciplinary Data Science \\ \texttt{<email protected>} \\ 497 DSL \\ (850) 645-8887
\end{itemize}

\paragraph{Data Science Faculty}
\begin{itemize}
    \item \textbf{Dexter, Nicholas} \\ Assistant Professor \\ \texttt{<email protected>} \\ 489 DSL
    \item \textbf{Erlebacher, Gordon} \\ Professor, Department of Scientific Computing \\ Program Director, Interdisciplinary Data Science \\ \texttt{<email protected>} \\ 464 DSL \\ (850) 322-0194
    \item \textbf{Meyer-Baese, Anke} \\ Professor \\ \texttt{<email protected>} \\ 476 DSL \\ (850) 644-3494
    \item \textbf{Shanbhag, Sachin} \\ Professor \\ \texttt{<email protected>} \\ 488 DSL \\ (850) 644-6548
    \item \textbf{Wang, Xiaoqiang} \\ Professor \\ \texttt{<email protected>} \\ 495 DSL
    \item \textbf{Zavala Romero, Olmo} \\ Assistant Professor \\ \texttt{<email protected>} \\ 445 DSL
\end{itemize}

\paragraph{Data Science Affiliated Faculty}
\begin{itemize}
    \item \textbf{Crock, Nathan} \\ Director, NewSci Labs \\ \texttt{<email protected>} \\ 475 DSL \\ (727) 460-6353
\end{itemize}

\subsubsection{Graduation Requirements}
\texttt{url: https://www.sc.fsu.edu/graduate/ms/datascience/graduation}

Students will graduate with a course-based Master of Science (M.S.) degree.
\begin{itemize}
    \item All students will take \textbf{18 credits of a core curriculum}.
    \item All students will take at least \textbf{12 credits of elective courses}.
\end{itemize}

\subsection{Doctor of Philosophy (Ph.D.) in Computational Science}
\texttt{url: https://www.sc.fsu.edu/graduate/phd}

Many of the important problems facing society today can only be solved by teams of individuals from a variety of disciplines. An individual trained in an interdisciplinary environment is an essential member of such a team because he/she can successfully interact with team members and gain an overall understanding of the problem.

However, most doctoral programs in this country have a very narrow focus and consequently, students rarely have the opportunity to understand research themes in other disciplines or even to explore the relevance of their own research to other fields. Since the Department of Scientific Computing (DSC) lies at the intersection of applied mathematics, applied science, computer science and engineering, it has the unique opportunity to train students in areas which cut across disciplines.

\subsubsection*{Ph.D Program Tracks}
The goal of the Ph.D. program in Computational Science is to train graduate students to have extensive knowledge in computational science and to allow the student to acquire expertise in one or more areas of the sciences, mathematics, or engineering. Students may choose to follow the major track in computational science which allows them to specialize in aspects of:
\begin{itemize}
    \item computational mathematics
    \item computer science
\end{itemize}
relevant to computational science or choose to complete one of the following specialized tracks:
\begin{itemize}
    \item Ph.D. in Computational Science with a Specialization in \textbf{Atmospheric Science}
    \item with a Specialization in \textbf{Biochemistry}
    \item with a Specialization in \textbf{Biological Science}
    \item with a Specialization in \textbf{Geological Science}
    \item with a Specialization in \textbf{Materials Science}
    \item with a Specialization in \textbf{Physics}
\end{itemize}

\subsubsection*{Coursework \& credits}
Since computational science is an interdisciplinary program, students' programs of study can be quite varied. Consequently, we have built in a lot of flexibility into the course requirements for the Ph.D. in Computational Science. The commonality in the coursework is that each student must take the same four computational science courses (Group A courses), which cut across disciplines, as well as a minimum of 9 credit hours in other computational science courses (Group B courses) plus 9 additional credit hours selected from existing departmental courses in computer science, engineering, mathematics or an applied science (Group C courses). Each student must also take six seminar hours (details are listed in the Graduate Handbook, section 5.2.2).

There are a total of 29 credit hours of coursework that are specified plus a minimum of 24 credits of dissertation hours. A specialization is obtained by completing a minimum of 9 credit hours from courses in the discipline approved by the student's supervisory committee.

The \textbf{required core courses (Group A)} consist of:
\begin{itemize}
    \item Introduction to Scientific Programming (3 credits)
    \item Applied Computational Science I (4 credits)
    \item Applied Computational Science II (4 credits)
    \item Parallel Programming, Algorithms and Architectures (3 credits)
\end{itemize}

The \textbf{elective core courses (Group B)} consist of courses such as:
\begin{itemize}
    \item Monte Carlo/Markov Chain Simulations
    \item Survey of Numerical PDEs
    \item Computational Space Physics
    \item Programming Skills for Computational Biology and Bioinformatics
    \item Computational Evolutionary Biology
    \item Introduction to Bioinformatics
    \item Computational Finite Element Methods
    \item Numerical Methods for Stochastic Differential Equations
    \item Numerical Methods for Earth and Environmental Sciences
    \item Molecular Dynamics
    \item Numerical Linear Algebra
    \item Data mining
    \item Visualization
    \item Verification and Validation in Computational Science
\end{itemize}
\textbf{Note:} Student should select a minimum of three courses from Group B, which are approved by his/her supervisory committee.

\subsubsection*{Application and Information}
Students applying to this program should have earned a bachelor's or master's degree in an applied science, mathematics, computer science or engineering and possess a keen interest in computational science. The DSC expect that graduates of this program will be prepared to seek employment in academic, industrial, or laboratory settings.

You can apply for the graduate programs online. Please start your application process early. For further information contact \texttt{<email protected>}.

\subsubsection{Specialization in Fire Dynamics}
\texttt{url: https://www.sc.fsu.edu/graduate/phd/fire-dynamics}

\paragraph{PROGRAM OVERVIEW} The program is about the study of fire in nature as a fluid dynamical phenomenon, with complex physical, chemical, and turbulent interactions with the environment. Our program emphasizes basic mathematical and physical concepts, the application of atmospheric dynamical principles, and supports both laboratory and field experimental inquiry. The program in fire dynamics may be of interest to: physical science and mathematically prepared students who are interested in the environment and natural systems; meteorology students interested in the role of aerosols, particulates, and gases emitted by forest fires and prescribed burning; physics or engineering students desiring to apply their knowledge to combustion in a natural environment; wildland fire experts who desire to further their academic career; computationally oriented students who desire to solve a problem of direct importance to society; and management and agency personnel who deal with the impact of wild land fires.

\paragraph{FACILITIES} Geophysical Fluid Dynamics Institute facilities include a large modern laboratory for hydrodynamics experiments, a colloquium room and reading room, a photographic and illustrations laboratory, a large modern machine shop, a precision instrument-makers laboratory, and faculty and student offices. Institute facilities also include several precision rotating turntables, a six-meter water channel, convection tanks, temperature controlling systems, general and digital photographic systems, multi-channel data acquisition systems, laser facilities, various machine tools, and other electronic equipment. The institute houses a facility for measuring ocean turbulence as well.

\paragraph{Attachments:}
\begin{center}
\begin{tabular}{ l l l }
\hline
\textbf{File} & \textbf{Description} & \textbf{File size} \\
\hline
GRADUATE PROGRAM IN FIRE DYNAMICS & Feb 2020 & 66 kB \\
\hline
\end{tabular}
\end{center}

\subsubsection{Specialization in Geophysical Fluid Dynamics}
\texttt{url: https://www.sc.fsu.edu/graduate/phd/gfd}

\paragraph{PROGRAM OVERVIEW} The Geophysical Fluid Dynamics (GFD) Program leads to a degree in Computational Sciences with a major in either GFD or Fire Dynamics. It is an interdisciplinary field of study whose primary goal is an improvement in our basic understanding of fluid flows that occur naturally, including such topics as climate and paleoclimate, biogeochemical processes, hydrology and Karst dynamics, air-sea interaction, wild fire dynamics, double diffusive processes, and hurricane dynamics with strong links to the Applied Mathematics Program. The approach to this understanding is through quantitative analysis of observational records and theoretical, mathematical, numerical, and experimenting modeling. A geophysical fluid dynamicist must have a firm grasp of the fundamental principles of classical physics, knowledge of the techniques of applied mathematics, and an interest in the natural sciences.

\paragraph{FACILITIES} These facilities include a large modern laboratory for hydrodynamics experiments, a colloquium room and reading room, a photographic and illustrations laboratory, a large modern machine shop, a precision instrument-makers laboratory, and faculty and student offices. Institute facilities also include several precision rotating turntables, a six-meter water channel, convection tanks, temperature controlling systems, general and digital photographic systems, multi-channel data acquisition systems, laser facilities, various machine tools, and other electronic equipment. The institute houses a facility for measuring ocean turbulence as well.

\paragraph{Attachments:}
\begin{center}
\begin{tabular}{ l l l }
\hline
\textbf{File} & \textbf{Description} & \textbf{File size} \\
\hline
GRADUATE PROGRAM IN GEOPHYSICAL FLUID DYNAMICS (GFD) & Feb 2020 & 73 kB \\
\hline
\end{tabular}
\end{center}

\section{Graduate Application and Financial Aid}

\subsection{Application Information}
\texttt{url: https://www.sc.fsu.edu/graduate/application}

Prospective graduate students - Welcome to the Department of Scientific Computing! We offer Master's and Ph.D. programs, open to students holding a bachelor's degree in mathematical and physical sciences and engineering.

\textbf{Data Science} Applications Follow Standard University Registration Deadlines.

\subsubsection*{FALL 2026 ADMISSION DEADLINES}
For \textit{Computational Science} Degrees. All Master of Science (M.S.) students are \textbf{self-pay}.
\begin{itemize}
    \item \textbf{January 15\textsuperscript{th}, 2026} \\ Ph.D. students with complete applications by this date will be considered for the University's Graduate School fellowships and the Department's TA/RA/Tuition waiver awards.
    \item \textbf{April 30\textsuperscript{th}, 2026} \\ Deadline for \textit{international student} admission to the Department. Students with complete applications by this date may also be considered for any possible remaining funding.
    \item \textbf{July 1\textsuperscript{st}, 2026} \\ Deadline for \textit{domestic student} admission to the Department. Students with complete applications by this date may also be considered for any possible remaining funding.
\end{itemize}

\subsubsection*{Application Process}
All prospective graduate students need to submit an application to Florida State University.

\paragraph{FSU Application Procedure:}
\begin{enumerate}
    \item Complete FSU's Office of Admissions on-line application.
    \item Pay FSU application fee of \$30 (check or credit card).
    \item Have \textbf{official GRE scores} sent to admissions (school code for Florida State University is \textbf{5219}. A department code is not required). The official scores must be received directly from the Educational Testing Service by the Office of Admissions.\\ \textbf{NOTE:} Please check the \textit{Test Scores} section of FSU's Graduate Admissions General Information bulliten for \textbf{waiver details} impacting Master's applicants. Test scores are still required for admission to doctoral programs.
    \item Have \textbf{two unofficial transcripts} submitted to the Office of Admissions by every college and university attended.
    \item Please email \texttt{<email protected>} if our "\textit{Department of Scientific Computing Supplemental Application for Admission}" does not appear in your FSU Application Status Checklist under "\textit{Forms}" within 2 business days.
\end{enumerate}

\paragraph{All applicants, please note:} Transcripts are considered official when they are sent from a college or university directly to the Office of Admissions and contain an official seal and signature. Transcripts bearing the statement "Issued to student" or transcripts submitted by the applicant are not considered official.

\subsubsection*{Admission requirements}
All graduate applicants must satisfy the following FSU admission requirements:
\begin{itemize}
    \item Have a baccalaureate degree from an accredited college or university.
    \item Be in good standing at the institution of higher learning last attended.
    \item Present evidence by official transcripts of a satisfactory prior academic record (at least a 3.0 G.P.A. on a 4.0 scale in all coursework attempted while registered as an upper-division undergraduate student) AND have a minimum ranking of 85 percentile (quantitative) and 50 percentile (verbal) on the Graduate Record Examinations General Test (GRE), under the new scoring system.
    \item \textbf{GRE is currently waived for the Master's admission}. Test scores are still required for admission to doctoral programs.
    \item Be a native English speaker OR pass the TOEFL test (international applicants only).
    \item Additional information for incoming international students, including language requirements, can be found at the International Admissions website.
    \item Be approved by the Department of Scientific Computing.
\end{itemize}
\textbf{NOTE:} Students must meet both University and departmental requirements.

\subsection{Financial Aid}
\texttt{url: https://www.sc.fsu.edu/graduate/financial-aid}

Financial aid in the form of a teaching or research assistantship is available to qualified applicants. The FSU application for graduate school serves as the request for assistantship from the department. You will be notified in your acceptance letter from the department.

In addition, there is a University Fellowship program that provides support for outstanding new and continuing graduate students. \textbf{These awards are highly competitive}.

FSU also has its own Office of National Fellowships to assist highly qualified students with information on the fellowships as well as with the application process and with interview preparations.

\section{Student Resources and Committees}

\subsection{Graduate Student Forms}
\texttt{url: https://www.sc.fsu.edu/graduate/forms}

\textbf{** Unless instructed otherwise, please populate all DocuSign "Email:" fields using official @fsu.edu email addresses listed here for faculty Advisor and Committee members **}

\subsubsection*{Common Forms}
\begin{itemize}
    \item Approval Form for Directed Independent Study (DIS) Form
    \item Graduate Student (Annual/Yearly) Progress Report Form (rev 2025)
    \item Individual Development Plan (IDP)
\end{itemize}

\subsubsection*{Master of Science (M.S.) Forms}
\begin{itemize}
    \item M.S. Committee Form
    \item M.S. Program of Study Form
    \item M.S. Result of Project Defense Form
\end{itemize}

\subsubsection*{Doctor of Philosophy (Ph.D.) Forms}
\begin{itemize}
    \item Ph.D. Admission to Candidacy Form
    \item Ph.D. Committee Form (Scientific Computing)
    \item Ph.D. Committee Form (Fire Dynamics / GFD)
    \item Ph.D. Program of Study Form
    \item Ph.D. Student Evaluation by Supervisory Committee
    \item Ph.D. Request to Schedule Preliminary Examination Form
    \item Ph.D. Result of Preliminary Examination Form
    \item Ph.D. Result of Prospectus Defense Form
    \item Ph.D. Manuscript Clearance Portal
    \item Ph.D. Manuscript Clearance Portal Online Forms Summary
\end{itemize}

*\textit{Effective August 2018}* Unless noted otherwise, all forms need to be filled and signed *digitally/electronically*. Printing of PDF forms is frowned upon and only permitted if an advisor explicitly requests a paper copy to review \& sign. All paper forms need to re-digitized and electronically submitted when complete.

\subsection{Graduate Handbook}
\texttt{url: https://www.sc.fsu.edu/graduate/handbook}

\begin{itemize}
    \item For students entering Fall 2025 - Present (revised July 2025)
    \item For students entering Fall 2024 - Spring 2025 (revised Jan 2025)
    \item For students entering Fall 2021 - Spring 2024 (revised April 2021)
    \item For students entering Fall 2019 - Spring 2021
    \item For students entering Fall 2016 - Spring 2019
    \item For students entering Fall 2009 - Spring 2016
    \item For students entering Fall 2007 - Spring 2009
\end{itemize}

\paragraph{Attachments:}
\begin{center}
\begin{tabular}{ l l l }
\hline
\textbf{File} & \textbf{Description} & \textbf{File size} \\
\hline
handbook\_1-7-2025.zip & Source Files & 495 kB \\
\hline
\end{tabular}
\end{center}

\subsection{Graduate Program Committee}
\texttt{url: https://www.sc.fsu.edu/committees/graduate-program}

Please see our by-laws (III, C, 2 - Standing Committees) for committee description and information.

\paragraph{Current committee members:}
\begin{itemize}
    \item Chen Huang [\textit{chair}]
    \item Xiaoqiang Wang
    \item Alan Lemmon
    \item Karey Fowler [\textit{ex officio}]
    \item \texttt{<email protected>}
\end{itemize}
\textbf{** Unless instructed otherwise, please populate all DocuSign "Email:" fields using official @fsu.edu email addresses listed here for Faculty/Student Supervisors and Student Employees **}

\paragraph{Graduate Supervisor Forms}
\begin{itemize}
    \item Graduate Assistant Performance Evaluation
\end{itemize}

\subsection{Undergraduate Program Committee}
\texttt{url: https://www.sc.fsu.edu/committees/undergraduate-program}

Please see our by-laws (III, C, 2 - Standing Committees) for committee description and information.

\paragraph{Current committee members:}
\begin{itemize}
    \item Sachin Shanbhag [\textit{chair}]
    \item Bryan Quaife
    \item Karey Fowler [\textit{ex officio}]
    \item \texttt{<email protected>}
\end{itemize}

\paragraph{Attachments:}
\begin{center}
\begin{tabular}{ l l l }
\hline
\textbf{File} & \textbf{Description} & \textbf{File size} \\
\hline
DSC\_UG\_MeetingMinutes\_20120301\_public.pdf & Committee Meeting Minutes 20120301 & 70 kB \\
DSC\_UG\_MeetingMinutes\_20120126\_public.pdf & Committee Meeting Minutes 20120126 & 70 kB \\
DSC\_UG\_MeetingMinutes\_20120112\_public.pdf & Committee Meeting Minutes 2012012 & 69 kB \\
DSC\_UG\_MeetingMinutes\_20111214\_public.pdf & Committee Meeting Minutes 20111214 & 73 kB \\
DSC\_UG\_MeetingMinutes\_20111116\_public.pdf & Committee Meeting Minutes 20111116 & 71 kB \\
DSC\_UG\_MeetingMinutes\_20111102\_public.pdf & Committee Meeting Minutes 20111102 & 73 kB \\
DSC\_UG\_MeetingMinutes\_20111019\_public.pdf & Committee Meeting Minutes 20111019 & 67 kB \\
DSC\_UG\_MeetingMinutes\_20110907\_public.pdf & Committee Meeting Minutes 20110907 & 84 kB \\
\hline
\end{tabular}
\end{center}

\section{Core Faculty}
\texttt{URL: https://www.sc.fsu.edu/research/faculty}

\subsection{Beerli, Peter}
\textit{Professor \& Chair, Department of Scientific Computing} \\
Email: [Protected]

\textbf{Research Interests:}
\begin{itemize}
    \item Biological Sciences
    \item Population Genetics
    \item Phylogenetics
    \item Bayesian inference
    \item Model selection
\end{itemize}

I started out my science career as a practical evolutionary biologist working with frogs on islands in the mediterranean sea. There was no good analysis software for my data available at that time. Therefore, in 1994, I started a career as a computational biologist developing software (MIGRATE: https://popgen.sc.fsu.edu) for other biologists to analyze large-scale DNA datasets and compare different evolutionary hypotheses. My interests have centered on the coalescent and how we can use that to infer parameters of population genetic models and also compare them statistically.

\subsection{Chipilski, Hristo}
\textit{Assistant Professor} \\
Email: [Protected]

\textbf{Research Interests:}
\begin{itemize}
    \item Data Assimilation
    \item Artificial Intelligence
    \item Numerical Weather Prediction
    \item Atmospheric Dynamics
\end{itemize}

My research encompasses theoretical, computational and applied aspects of data assimilation — the science of optimally combining numerical models and observations of physical systems. In the past, I have utilized data assimilation algorithms to improve the representation of atmospheric convection through the incorporation of ground-based remote sensors. More recently, my focus has shifted to the development of new data assimilation methods which capitalize on the ongoing AI revolution. Beyond data assimilation, my interests also extend to numerical weather prediction, atmospheric dynamics, and various topics within the data sciences.

\subsection{Dexter, Nicholas}
\textit{Assistant Professor} \\
Email: [Protected]

\textbf{Research Interests:}
\begin{itemize}
    \item Data Science
    \item Scientific Computing
    \item Deep Learning
    \item Compressed Sensing
    \item Mathematical Optimization
    \item Uncertainty Quantification
    \item Numerical Analysis
    \item Inverse problem
    \item Computational Epidemiology
    \item Computational Genomics
    \item Harmonic Analysis
\end{itemize}

Before joining the Department of Scientific Computing at FSU as an Assistant Professor in August 2022, I was previously a Pacific Institute for the Mathematical Sciences Postdoctoral Fellow working with Professors Ben Adcock, Maxwell Libbrecht, and Leonid Chindelevitch at Simon Fraser University. I studied Mathematics at the University of Tennessee under Professor Clayton Webster, and worked in the Computational and Applied Mathematics Group at Oak Ridge National Laboratory.

\subsection{Erlebacher, Gordon}
\textit{Professor, Department of Scientific Computing} \\
\textit{Program Director, Interdisciplinary Data Science} \\
Email: [Protected]

\textbf{Research Interests:}
\begin{itemize}
    \item Data Science
    \item Artificial Intelligence
\end{itemize}

I have been engaged in the development of artificial intelligence since 2014, beginning with the advent of word2vec. Over the years, my research has evolved alongside the field, encompassing work on autoencoders, pruning techniques, graph neural networks, and topic modeling, eventually leading to modern transformer-based architectures. More recently, my focus has shifted toward the application of large language models (LLMs) in education, where I am developing tools to enhance classroom engagement and learning outcomes using frontier AI capabilities. I am also exploring the integration of agentic systems to streamline operations within the Department of Scientific Computing, with the dual goals of reducing administrative workload and improving visibility to prospective students.

\subsection{Huang, Chen}
\textit{Associate Professor} \\
Email: [Protected]

\textbf{Research Interests:}
\begin{itemize}
    \item Computational Materials Science
    \item Scientific Computing
\end{itemize}

I am interested in computational material science. Two main research topics in my group are: (1) developing new multiphysics methods to achieve high accuracy in materials simulations (such as predicting novel electronic structures at oxide interfaces) and (2) developing new orbital-free density functional theory to enable large-scale, accurate simulations of functional materials (such as metal alloys and lithium battery materials).

\subsection{Lemmon, Alan}
\textit{Professor} \\
Email: [Protected]

\textbf{Research Interests:}
\begin{itemize}
    \item Biological Sciences
\end{itemize}

\subsection{Meyer-Baese, Anke}
\textit{Professor} \\
Email: [Protected]

\textbf{Research Interests:}
\begin{itemize}
    \item Medical imaging: pattern recognition techniques applied to breast MRI, computer-aided diagnosis, fMRI data analysis.
    \item Computational biology: dynamical analysis of gene regulatory networks, graph theoretical concepts applied in therapeutics of glioblastoma, stem cells, phosphoproteomics.
    \item Computational neuroscience: brain-based classification techniques, nonlinear stability analysis of cortical systems, graph theory applied to cortical networks.
\end{itemize}

\subsection{Plewa, Tomasz}
\textit{Professor} \\
Email: [Protected]

\textbf{Research Interests:}
\begin{itemize}
    \item Computational Astrophysics
    \item Scientific Computing
    \item fluid dynamics and magnetized flows
    \item reactive flows, flames, and detonations
    \item turbulence and turbulent combustion
    \item adaptive mesh refinement (AMR)
    \item machine learning for subgrid scale modeling
    \item stellar evolution, core collapse and thermonuclear supernovae
    \item laser-driven experiments, high-energy density physics
    \item solution verification and model validation
    \item extreme scale and high performance computing, data analytics
\end{itemize}

\subsection{Quaife, Bryan}
\textit{Associate Professor} \\
Email: [Protected]

\textbf{Research Interests:}
\begin{itemize}
    \item Integral equation methods for complex fluids, in particular, vesicle suspensions
    \item Efficient and high-order methods for solving integral equations
    \item Adaptive and high-order time stepping schemes
    \item Integral equation methods for PDEs on surfaces
    \item Regularizations of Green's functions
    \item Integral equation methods for viscous flow in porous media
    \item Preconditioners for integral equations
\end{itemize}

\subsection{Shanbhag, Sachin}
\textit{Professor} \\
Email: [Protected]

\textbf{Research Interests:}
\begin{itemize}
    \item Polymer Physics
    \item Rheology
    \item Complex Fluids
    \item Modeling for Biological and Materials Applications
\end{itemize}

\subsection{Speer, Kevin}
\textit{Professor} \\
Email: [Protected]

\textbf{Research Interests:}
\begin{itemize}
    \item Geophysical Fluid Dynamics Institute (gfdi.fsu.edu)
\end{itemize}

Kevin has been the director of GFDI since 2011 and has been an active associate for his entire time at FSU. Kevin is a sea-going oceanographer whose research ranges from the global ocean circulation to the dynamics of hydrothermal plumes.

\subsection{Wang, Xiaoqiang}
\textit{Professor} \\
Email: [Protected]

\textbf{Research Interests:}
\begin{itemize}
    \item Numerical analysis and applied partial differential equations
    \item Mathematical biology
    \item Image processing, scientific visualization and data mining
    \item High-performance scientific computing
\end{itemize}

\subsection{Zavala Romero, Olmo}
\textit{Assistant Professor} \\
Email: [Protected]

\textbf{Research Interests:}
\begin{itemize}
    \item Data Science
    \item Scientific Computing
    \item Medical Image Processing
    \item Applied machine learning in medical imaging: prostate and breast cancer detection, personal diagnosis from clinical and image (MRI) data, improve generalization of models.
    \item Applied machine learning in earth sciences: data assimilation in ocean models, nowcasting (precipitation and hail), short-term forecast of air pollution, loop current and eddy detection and analysis.
    \item Climate change: predicting future distribution of invasive insect pests considering climate change projections.
    \item Scientific Machine Learning: Physics Informed Neural Networks (PINNs) to improve parameterizations in Ocean Models, etc.
\end{itemize}

\section{Emeritus Faculty}
\texttt{URL: https://www.sc.fsu.edu/research/faculty}

\subsection{Gunzburger, Max}
\textit{Robert O. Lawton Distinguished Professor / Krafft Professor Emeritus} \\
Email: [Protected]

\textbf{Research Interests:}
\begin{itemize}
    \item Mathematics
    \item Scientific Computing
\end{itemize}

\subsection{Navon, Michael}
\textit{Professor Emeritus} \\
Email: [Protected]

\textbf{Research Interests:}
\begin{itemize}
    \item Reduced order modeling
\end{itemize}

\subsection{Peterson, Janet}
\textit{Professor Emeritus} \\
Email: [Protected]

\textbf{Research Interests:}
\begin{itemize}
    \item Scientific Computing
\end{itemize}

\section{Computing}
\url{https://www.sc.fsu.edu/computing}

The Department of Scientific Computing maintains a large and dynamic computing infrastructure in support of research and education.

Computing resources include the FSU RCC/HPC, a number of clusters and computational servers, a laboratory for scientific visualization, a bioinformatics server, and more. While many of the department's computing resources are dedicated to specific research areas, an increasing number of machines are being made available to the University research community in general through the use of high throughput management tools and job schedulers.

\subsection{Getting Started}
\url{https://www.sc.fsu.edu/computing/getting-started}

For new users getting started in our department the following technical documents should be read first:
\begin{itemize}
    \item Getting an account
    \item Logging in to SC systems
    \item Changing your password or environment
    \item Configuring Email
    \item Personal Website Hosting
    \item Account Policy
\end{itemize}

\subsection{Hardware}
\url{https://www.sc.fsu.edu/computing/hardware}

The Department of Scientific Computing maintains computing hardware in a robust machine room on the fourth floor of Dirac Science Library. Most the hardware in this machine room is owned by DSC and non-DSC research groups and is dedicated to a wide range of research problems including; molecular biophysics, evolutionary biology, network modeling, and Monte-Carlo algorithm development. DSC faculty and students can access the idle cycles on the owner based machines by using a system called \texttt{condor}. The \texttt{condor system} is configured to match single process non-interactive batch submissions to any idle CPUs hosted in the DSC machine room or in DSC desktop machines. Some general access machines are made available to DSC faculty and students to compile software and to run interactive jobs, e.g., MatLab. Also, the DSC manages and supports a general access visualization lab and FSU's shared HPC facility. To learn more about the hardware resources hosted or managed by DSC, click on one of the links below.
\begin{itemize}
    \item General Access
    \item Facilities, Computational Infrastructure
\end{itemize}

\subsection{General Access}
\url{https://www.sc.fsu.edu/computing/general-access}

The Department of Scientific Computing manages and supports a number of general access facilities for DSC faculty and students as well the University computing in general. Click on any of the links below to learn more about these resources.
\begin{itemize}
    \item Interactive
    \item Visualization
    \item High Performance Computing
\end{itemize}

\subsubsection{High Performance Computing (RCC)}
\url{https://www.sc.fsu.edu/computing/general-access/hpc}

\paragraph{The Research Computing Center (RCC)} - formerly known as the FSU shared-HPC facility - enables multidisciplinary computing for the FSU research community. To learn more about this facility please see the RCC web site.
\begin{itemize}
    \item RCC Homepage
    \item High-performance Computing
\end{itemize}

\subsubsection{Interactive}
\url{https://www.sc.fsu.edu/computing/general-access/interactive}

DSC students, faculty, and students taking Computational Science courses can interactively access the computers located in DSL 152 and the hallway via \texttt{ssh}. This allows you to request interactive resources and it will randomly instantiate a session on a workstation. What follows is a quick start for the DSC environment.

\paragraph{How to Run an Interactive Job}
Open an X11 terminal and type:
\begin{itemize}
    \item \texttt{qlogin}
    \item OR
    \item \texttt{hallway} (to access hallway nodes only)
    \item \texttt{classroom} (to access classroom nodes only)
\end{itemize}

After authenticating, type the name of the program you want to run, e.g., \texttt{matlab}.

If \texttt{qlogin} is not available on your machine, you must run the above commands from a DSC desktop. SSH to a DSC desktop or hallway machine:
\texttt{ssh -X hallway-a.sc.fsu.edu}

Then follow the above instructions.

\paragraph{Access Policy}
\begin{itemize}
    \item 1 job per user
\end{itemize}

\paragraph{Availability}
Classroom queues are available with the exception of Mon - Fri 7:45 AM till 6:00 PM.
Hallway queues are available 24/7.
To guarantee which queue you land in please use either \texttt{qlogin -q classroon} or \texttt{qlogin -q hallway}.

\textbf{NOTE:} All remote classroom sessions will be terminated at 7:45 AM M-F. The purpose of the classroom and hallway machines is primarily for students to do their assignments - not as a production engine. Use FSU's HPC system (\url{http://rcc.fsu.edu}) for production batch schedule jobs.

\subsubsection{Visualization}
\url{https://www.sc.fsu.edu/computing/general-access/visualization}

It is said that "A picture is worth a thousand words". How true. The ultimate goal of scientific visualization is to provide scientists with tools that permit them to analyze their data, extract information/features, correlate that information, and display their data in a meaningful way. More importantly, visualization encourages the scientists to ask questions. The more interactive an application, the more questions a scientist can pose. Application areas with large scale datasets (steady or time-dependent) include oceanography, study of hurricanes, data assimilation techniques, phylogenetic tree evolution, geophysics, fluid dynamics, and more.

The visualization laboratory is a DSC facility available for the use of researchers across the entire Florida State University Campus. Use is divided into the following categories:
\begin{itemize}
    \item Visualization Research
    \item Visualization Education
    \item Application of Visualization
\end{itemize}

\paragraph{Applications}
The laboratory is open to all researchers of Florida State University who have a need to visualize and analyze their simulation or other data. Be sure to check out the tutorial section of this web site. This access can take several forms:
\begin{itemize}
    \item \textbf{Work within the visualization lab}, making use of dual large screens for maximum flexibility during the generation and rendering of large scale imagery and the generation of sophisticated animations. User data resides on our local 12 TByte storage facility.
    \item \textbf{Remote access to the visualization laboratory from desktop computers or workstations}. Typically this involves remote SSH connections and X-Window tunneling to run applications on the visualization machines from another location--this does not fully support hardware based graphics. During the coming months the Vis Lab will be testing innovative new software from HP, which allows users to execute any software that runs locally from a remote desktop Linux or windows machine in full graphics mode. As in the first case, data would be stored on our local file system for maximum efficiency.
\end{itemize}

\paragraph{Education}
Perhaps most importantly, visualization is an integral component of the workflow adopted by researchers around the world. Visualization is necessary to:
\begin{itemize}
    \item construct computational grids that discretize the physical domain,
    \item check data for accuracy, correctness
    \item analyze simulation data
    \item extract features from simulation data
    \item ask questions of the data
\end{itemize}

\paragraph{Research}
Scientists are encourage to use the laboratory for advanced research in the area of scientific visualization. Areas of interest include:
\begin{itemize}
    \item Real time rendering using graphics hardware
    \item Remote rendering
    \item Collaborative visualization
    \item Pattern recognition
    \item Feature extraction
    \item Information visualization
    \item Visual realism
    \item Visualization of uncertainty
    \item Visualization of stochastic multi-dimensional processes
    \item Visualization of higher dimensional data (four and above)
    \item and more ...
    \item Computer Vision
\end{itemize}

\subsection{Remote Access}
\url{https://www.sc.fsu.edu/computing/remote-access}

For Computational Science students taking classes needing remote resources for their coursework and research. Please see the following recommended resources:

\paragraph{Scientific Computing Resources}
[some restrictions apply]
\begin{itemize}
    \item Students can use secure shell client to connect to \texttt{pamd.sc.fsu.edu}, from there run \texttt{qlogin} or \texttt{classroom}.
    \item SSH to \texttt{pamd.sc.fsu.edu}
    \item Interactive Shell using \texttt{qlogin} or \texttt{classroom}
\end{itemize}

\paragraph{Remote Desktop Connections}
Students can use secure shell client to tunnel to \texttt{pamd.sc.fsu.edu}, from there you can reference "localhost" to tunnel to a protected remote desktop resource.

\paragraph{myFSUVLab}
[24x7 Access]
\url{myfsuvlab.its.fsu.edu}
Click Here for myFSUVLab's Service Details.

\paragraph{MATLAB Online}
[24x7 Access]
\url{matlab.mathworks.com}
Click Here for MATLAB Online's General Limitations.

\paragraph{Research Computing (RCC/HPC)}
[24x7 Access]
Getting Started $>$ \url{its.fsu.edu/research/rcc-user-accounts}
From the "Select a Faculty Sponsor" dropdown you can choose from the following DSC facuilty members: Gordon Erlebacher, Peter Beerli, Sachin Shanbhag, Xiaoqiang Wang.

\subsubsection{Remote Desktop at DSC}
\url{https://www.sc.fsu.edu/computing/remote-access/desktop}

To facilitate GUI access to classroom and hallway systems, we have opened up remote desktop connections for DSC faculty and students with valid departmental account. Please email This email address is being protected from spambots. You need JavaScript enabled to view it. if you have any difficulty.

\begin{enumerate}
    \item Install "Microsoft Remote Desktop" for Mac; Windows users already have this app by default.
    \item Open terminal for Mac; command prompt for Windows
    \item Run the following command to create a tunnel to hostname (classXX or hallway-X where XX = 02 through 19; X = a through e)
    
    (Mac)
    
    \texttt{ssh -L:3389:hostname.sc.fsu.edu:3389 your\_fsuid@pamd.sc.fsu.edu}
    
    (Windows)
    
    \texttt{ssh -L:13389:hostname.sc.fsu.edu:3389 your\_fsuid@pamd.sc.fsu.edu}
    
    \item Login and keep this tunnel for the next step
    \item Start Microsoft Remote Desktop (Mac) / Add PC
    \begin{itemize}
        \item PC name: \texttt{localhost}
        \item Friendly name: My tunnel to classroom (whatever name you prefer)
    \end{itemize}
    \item Open this newly created "My tunnel to classroom" and login to the pre-selected hostname
    
    \textbf{NOTE:} You can safely ignore certificates warning by clicking on "Continue"
    
    \item Start Remote Desktop Connection (Windows) / Show Options
    \begin{itemize}
        \item Computer: "\texttt{localhost:13389}"
        \item User name: "your\_fsuid"
    \end{itemize}
    \item Click on "Connect" and login to the pre-selected hostname
    
    \textbf{NOTE:} You can safely ignore certificates warning by clicking on "Yes"
\end{enumerate}

\subsubsection{SSH Access}
\url{https://www.sc.fsu.edu/computing/remote-access/ssh}

Most of the department computing resources can be accessed from \texttt{pamd.sc.fsu.edu}. Use the \texttt{hallway} or \texttt{classroom} or \texttt{cil} command to connect to our application servers. For example:
\begin{verbatim}
pamd% hallway
pamd2% classroom
\end{verbatim}
The \texttt{qlogin} command is also available on our desktop workstations (e.g. hostname = \texttt{dskscsXXX.sc.fsu.edu} or \texttt{hallway-[a-e].sc.fsu.edu}). Students can also connect to their departmental desktops directly from pamd or pamd2 as follows:
\begin{verbatim}
pamd% ssh dskscsXXX.sc.fsu.edu
\end{verbatim}
\textbf{Never} use the \texttt{pamd.sc.fsu.edu} servers to run processor intensive jobs. Intensive processes will be terminated to ensure SSH access to \texttt{pamd.sc.fsu.edu} is available to all users.

\subsection{Software}
\url{https://www.sc.fsu.edu/computing/software}

The following software categories are supported by the department's technical support group. Support for any additional submission tools needs to be approved by the Technology Committee.
\begin{itemize}
    \item Supported Applications
    \item Supported Libraries \& Compilers
    \item Supported Submission Tools
\end{itemize}

\subsubsection{Supported Applications}
\url{https://www.sc.fsu.edu/computing/software/applications}

The following software applications are supported by the department's technical support group. Support for any additional applications needs to be approved by the Technology Committee.
\begin{itemize}
    \item Acrobat Reader
    \item Mathematica
    \item R
    \item MATLAB
    \item LaTeX
    \item ImageMagick
    \item Gnuplot
\end{itemize}

\subsubsection{Supported Libraries \& Compilers}
\url{https://www.sc.fsu.edu/computing/software/libraries-compilers}

The following libraries and compilers are supported by the department's technical support group. Support for any libraries and compilers needs to be approved by the Technology Committee.
\begin{itemize}
    \item GROMACS
    \item Super LU
    \item LAPACK
    \item GSL
    \item FFTW
    \item BLAS
    \item ARPACK
    \item g95
    \item g77
    \item g++
    \item GNU Compiler Collection (GCC)
\end{itemize}

\subsubsection{Supported Submission Tools}
\url{https://www.sc.fsu.edu/computing/software/submission-tools}

The following submission tools are supported by the department's technical support group. Support for any additional submission tools needs to be approved by the Technology Committee.
\begin{itemize}
    \item Interactive
\end{itemize}

\subsection{Tech Docs}
\subsubsection{Version Control Systems}
\url{https://www.sc.fsu.edu/computing/96-tech-docs/225-svn-information}

Will your code be for free / open source software?

If \textbf{YES}, then you should first try one of these free source code repositories:
\paragraph{github}
"GitHub is how people build software. With a community of more than 15 million people, developers can discover, use, and contribute to over 38 million projects using a powerful collaborative development workflow." \textit{source}
\paragraph{SourceForge.net}
"SourceForge.net is the world's largest open source software development web site. We provide free services that help people build cool stuff and share it with a global audience." \textit{source}

If \textbf{NO}, then we recommend using \texttt{git}.

\paragraph{Quick start for Git}
Creating and commiting on \texttt{pamd.sc.fsu.edu}
\begin{verbatim}
$ cd (project-dir)
$ git init
$ (add some files)
$ git add .
$ git commit -m 'Initial commit'
\end{verbatim}
Cloning and Creating a Patch
\begin{verbatim}
$ git clone ssh://{mylogin}@pamd.sc.fsu.edu/panfs/\
   panasas1/research/{mydir}/{mygitproj}
$ cd {mygitproj}
$ (edit files)
$ git add (files)
$ git commit -m 'explain what I chanded'
$ git format-patch origin/master
\end{verbatim}
For more information about how to use git see the Git Community Book. The section on Distributed Workflows is especially useful.

\textbf{Note:} Once your code is presentation ready we recommend sharing the URL and details of your project for posting to our Department software collection here. You can email This email address is being protected from spambots. You need JavaScript enabled to view it. with the details.

\section{HELP! - Technical Support}
\url{https://www.sc.fsu.edu/computing/help}

Need help? If you have a technology question or other issue, then send us an email, and your request will be directed to someone who will help.

This email address is being protected from spambots. You need JavaScript enabled to view it.

\section{Software}
\url{https://www.sc.fsu.edu/software}

\subsection{Morphometrics Lab Software}

\begin{itemize}
    \item \textbf{Morpheus}: a cross-platform, general purpose software package for morphometric analysis. More information is available here.
    
    \item \textbf{forAge}: a program for age-at-death estimation using 3D laser scans of the adult human pubic symphysis. More information is available here.

    \item \textbf{GPSA}: software implementing the Generalized Procrustes Surface Analysis (GPSA). More information is available here.
\end{itemize}

\subsection{Migrate}
is a program to estimate population sizes and asymmetrical migration rates between populations and is authored by This email address is being protected from spambots. You need JavaScript enabled to view it.. Migrate uses maximum likelihood and Bayesian approaches to analyze sequences (DNA, RNA), single nucleotide polymorphisms (SNP), microsatellites, and electrophoretic markers. Both approaches utilize the Metropolis-Hastings algorithm to integrate over all possible genealogical relationships. Future developments of Migrate include; speed improvements (via code optimization and additional parallelization) and the relaxation of some population genetic assumptions (e.g., the constancy of size and migration rates through time). Migrate runs on any modern computer platform and also runs on clusters using the Message Passing Interface. The C source code is distributed free of charge here.

\subsection{MrBayes}
is a program for the Bayesian estimation of phylogeny and evolution authored by John Huelsenbeck (University of California, Berkeley) and This email address is being protected from spambots. You need JavaScript enabled to view it. (Florida State University). It handles a wide variety of stochastic evolutionary models for molecular, morphological and other types of discrete data. MrBayes uses Metropolis-coupled Markov chain Monte Carlo techniques to estimate the posterior probability distribution. The program runs on all common platforms and is also available in an MPI-enabled version for cluster computers and shared-memory machines. The source code and executables are distributed free of charge from here.

\subsection{PAUP}
infers phylogenetic trees from DNA and protein sequences and morphological data according to parsimony, distance, and maximum likelihood optimality criteria. In addition to tree-searching, PAUP provides capabilities for data-format conversion, post-tree analyses such as consensus trees and ancestral-state reconstruction, and graphical output. PAUP executables are available to run on Macintosh, Windows, Linux and several other Unix operating systems. The most recent release of PAUP supports multithreading on multiprocessor machines and a version under development supports parallel computation on clusters and supercomputers, using MPI or PVM to distribute tree evaluations across processors. More information is available here.

The following technical documents are here to help our users with the various software and hardware resources available to them. Please send an email to \href{mailto:sysops@sc.fsu.edu}{sysops@sc.fsu.edu} if there is something that needs additional documentation.

\section{Facilities, Computational Infrastructure}
\url{https://www.sc.fsu.edu/computing/tech-docs/278-facilities-computational-infrastructure}

Feel free to copy relevant portions of this text into your research proposals.

\subsection*{The Department of Scientific Computing (DSC)}
The Department of Scientific Computing (DSC) plays a major role in the support of FSU's cyberinfrastructure by providing facilities and technical expertise in the support of scientific computing. The DSC manages a dedicated computing facility located on the main FSU campus in Dirac Science Library. The DSC facility provides a highly flexible computing environment designed to support specialized and experimental hardware and software systems.

\subsection*{DSC Computing Facility}
The DSC facility supports an assortment of computer architectures, interconnects, and operating systems. Systems hosted in the DSC facility are owned by DSC and are dedicated to a wide range of research problems including; machine learning, neuroscience, molecular biophysics, evolutionary biology, network modeling, and Monte-Carlo algorithm development. The DSC facility is equipped with two 40-ton HVAC cooling units, 1000 ft$^2$ of raised floor, an extensive power distribution system, UPS battery backup systems, and a 550 KVA diesel-powered backup generator, which provides backup power to all of the hardware and HVACs in this server room. The DSC network is built on a 10 Gbps backbone, providing connectivity to a switching infrastructure and to key servers and storage. The DSC network connects via 100 Gbps to the FSU campus backbone, which in turn connects to the Florida/National LamdaRail.

\subsection*{FSU RCC (formerly the Shared-HPC)}
Please \href{https://its.fsu.edu/help/it-support/researchers#grants}{click here to access the latest RCC facilities statement}.

\subsection*{Scientific Visualization}
The DSC supports a general access laboratory for scientific visualization and computational intelligence. The laboratory is located in the center of the main FSU campus on the fourth floor of the Dirac Science Library (DSL). The Visualization Laboratory hosts several high-end visualization workstations each equipped with GPU video cards that are compatible with the CUDA SDK. All workstations have access to a multi-terabyte shared high-performance storage.

The DSC's visualization resources also include a high-resolution laser projection system to support multidisciplinary scientific visualization. The system is located in our main seminar room adjacent to the Visualization Lab. A cutting edge 4K Enhancement Technology rear-mounted projector illuminates an 18' x 8' screen. The system supports numerous input devices via a simple to use touch panel screen.

\subsection*{General DSC Infrastructure}
The DSC provides office space to DSC faculty, postdocs, graduate students, and other DSC associated support personnel. Designated visitor offices are also available. All offices are equipped with a desktop computer and network connections; wireless is available throughout campus. The department of Scientific Computing supports a cutting edge classroom facility on the campus of FSU to support scientific programming curriculum. The classroom was funded in part by a Student Technology fee award for instructional technology enhancements. The room is equipped with 19 Intel-based workstations running LINUX or Windows and is used primarily for classes taught by DSC faculty. In addition to the computer classroom, a large seminar room is located on the fourth floor with a capacity for 80 people and is equipped with a 4K Enhanced rear-mounted laser projection system. Also, two conference rooms equipped with large high-definition displays can facilitate smaller groups.

\subsection*{DSC Classroom}
The Department of Scientific Computing (DSC) takes pride in its cutting-edge classroom facility designed to cater to modern computational and scientific needs. Situated within the FSU campus, the classroom has been recently updated to include the latest technology, thereby providing a conducive environment for effective teaching and learning.

The classroom is equipped with 20 new systems, each powered by an Intel® Core™ i9-10940X processor, boasting 3.30GHz speed and 14 cores. These machines are fitted with a robust 64GB of RAM, ensuring rapid data processing and multitasking capabilities. The GPU in these systems is the NVIDIA RTX A5000, designed to handle intense graphics and data-intensive tasks, making them ideal for research in machine learning, data science, and scientific computing.

The DSC Classroom is a testament to FSU's commitment to providing state-of-the-art facilities for research and education in scientific computing. With this advanced setup, students and faculty have the resources they need for in-depth exploration and innovative problem-solving in various domains.
\hfill \textit{Written by Michael McDonald}

\section{File Transfers and Sharing}
\url{https://www.sc.fsu.edu/computing/tech-docs/1661-file-transfers-and-sharing}

\subsection*{FSU Noles File Transfer (NiFTy)}
Are you having trouble to share large documents/attachments? FSU NiFTy allows you to dropoff large files for anyone in the world with email access and vice versa. Please give this a try.
\begin{center}
    \href{http://NiFTy.FSU.edu}{NiFTy.FSU.edu}
\end{center}
\hfill \textit{Written by Michael McDonald}

\section{Download files from DSC computers}
\url{https://www.sc.fsu.edu/computing/tech-docs/1669-download-files-from-dsc-computers}

From off campus, you are able to download files or folders from DSC computers to personal laptops.

\subsection*{Mac}
Open a terminal window in Mac, and first create the tunnel of transfer by providing your fsuid and remote computer name:
\begin{verbatim}
ssh -f -N -L 2222:remote_computer_name.sc.fsu.edu:22 your_fsuid@pamd.sc.fsu.edu
\end{verbatim}
Once the tunnel is established by providing credentials, you can start to download or upload as follows:
\begin{verbatim}
rsync -avz -e 'ssh -p 2222' fsuid@localhost:/remote_directory_to_be_downloaded /local_destination
rsync -avz -e 'ssh -p 2222' /local_upload_folder fsuid@localhost:/remote_destination_directory
\end{verbatim}

\subsection*{Windows 10}
Open a command Window, and establish the tunnel (NOTE the option -R):
\begin{verbatim}
ssh -f -N -R 2222:remote_computer_name.sc.fsu.edu your_fsuid@pamd.sc.fsu.edu
\end{verbatim}
Now you can download or upload files and/or folders as follows (NOTE the option -P):
\begin{verbatim}
scp -r -P 2222 fsuid@localhost:/remote_directory_to_be_downloaded /local_destination
scp -r -P 2222 /local_source fsuid@localhost:/remote_destination
\end{verbatim}
fsuid password may be required.
\hfill \textit{Written by Xiaoguang Li}

\section{Changing your password or environment}
\url{https://www.sc.fsu.edu/computing/tech-docs/177-changing-your-password-or-environment}

\subsection*{Changing passwords}
The department uses FSUIDs/LDAP for a centralized authentication system. This allow for one common password to be used across all of the desktops, webpages, and clusters that we support. However many systems don't understand how to change passwords correctly in LDAP, so if you use a local utility to change your password it probably won't work correctly.

To change your password please follow the link bellow.
\begin{center}
    \href{https://my.fsu.edu/Account-Help/Change-Reset-my-FSUID-Password}{Change/Reset my FSUID Password}
\end{center}

\subsection*{Changing environment}
The department supports three default login shells:
\begin{itemize}
    \item \texttt{csh}
    \item \texttt{tcsh}
    \item \texttt{bash}
\end{itemize}
To change your shell, please email \href{mailto:sysops@sc.fsu.edu}{sysops@sc.fsu.edu}.
\hfill \textit{Written by Administrator}

\section{Configuring Email}
\url{https://www.sc.fsu.edu/computing/tech-docs/178-email}

The Department of Scientific Computing has transferred its email and list services to Florida State University. Your primary email account is now your \texttt{fsuid@my.fsu.edu}. Any mailing lists you are subscribed to need to be modified to use your \texttt{fsuid@my.fsu.edu}.

Here are several methods to access \texttt{fsuid@my.fsu.edu}.

\subsection*{Webmail}
\url{https://webmail.fsu.edu/}

\subsection*{Outlook}
Follow \href{https://its.fsu.edu/help/it-support/email-calendars/student-email-accounts/configure-outlook-your-student-email}{these instructions for outlook access}.

After the setup, you can refer to \href{https://www.sc.fsu.edu/computing/tech-docs/234-outlook-faq}{Outlook FAQ}.

\subsection*{Thunderbird}
If you are current Thunderbird users, you already have accounts setup. Adding FSU account is similar. To include \texttt{fsuid@my.fsu.edu} to Thunderbird, \href{https://support.mozilla.org/en-US/kb/automatic-account-configuration}{follow these instructions}.

\href{https://www.sc.fsu.edu/computing/tech-docs/233-thunderbird-faq}{Thunderbird FAQ}

\subsection*{Other Client Programs}
We don't encourage the use of other email clients than listed above. If you have to use Mac Mail, Entourage, Eudora, or Outlook Express, you can find the information at \href{https://its.fsu.edu/help}{FSU Helpdesks}.

\subsection*{Preferred FSU Email Address}
You may want to change the preferred email address associated with your FSUID if it is \texttt{fsuid@cs.fsu.edu} or \texttt{fsuid@scs.fsu.edu}. Here is how.

Create a case with ITS to change your preferred email address, \url{https://servicecenter.fsu.edu}
\hfill \textit{Written by Xiaoguang Li}

\section{CVS information}
\url{https://www.sc.fsu.edu/computing/tech-docs/226-cvs-information}

CVS usage is generally depreciated in favor of git or SVN. More information regarding version control systems can be found \href{https://www.sc.fsu.edu/computing/tech-docs/225-svn-information}{here}.

What follows is a general tutorial on how to use CVS if you already have a repository hosted somewhere.

From any unix host, such as pamd, ensure all your files for your repository are under the same directory tree.

Set environmental variables by editing your \texttt{.profile}.
\begin{description}
    \item[bash:]
    \begin{verbatim}
CVSROOT={mylogin}@cvs.server.org:/data/cvs
CVS_RSH=ssh
export CVSROOT
export CVS_RSH
    \end{verbatim}
    \item[csh, tsch:]
    \begin{verbatim}
setenv CVSROOT {mylogin}@cvs.server.org:/data/cvs
setenv CVS_RSH ssh
    \end{verbatim}
\end{description}

Import to your repository on the CVS server.
\begin{verbatim}
cvs import -m "{message}" {myproject} {myname} {start}
\end{verbatim}
To create a working copy of your directory:
\begin{verbatim}
cvs checkout {myproject}
\end{verbatim}
Edit a file and check the differences:
\begin{verbatim}
cvs diff
\end{verbatim}
To commit your changes:
\begin{verbatim}
cvs commit
\end{verbatim}

\section{Acrobat Reader}
\url{https://www.sc.fsu.edu/computing/tech-docs/331-acrobat}

\href{https://acrobat.adobe.com/us/en/acrobat/pdf-reader.html}{Acrobat Reader} is a commercial PDF file viewer. To use Acrobat Reader, type \texttt{acroread} at the X11 command prompt. It is available on most desktop machines.
\hfill \textit{Written by Michael McDonald} \\
\hfill \textit{Category: Supported Applications}

\section{ARPACK}
\url{https://www.sc.fsu.edu/computing/tech-docs/338-arpack}

\href{http://www.caam.rice.edu/software/ARPACK/}{ARPACK} is a collection of Fortran77 subroutines designed to solve large scale eigenvalue problems.
\hfill \textit{Written by Michael McDonald} \\
\hfill \textit{Category: Supported Libraries \& Compilers}

\section{BLAS}
\url{https://www.sc.fsu.edu/computing/tech-docs/339-blas}

The Basic Linear Algebra Subprograms (\href{http://www.netlib.org/blas/}{BLAS}) are routines that provide standard building blocks for performing basic vector and matrix operations. The Level 1 BLAS perform scalar, vector and vector-vector operations, the Level 2 BLAS perform matrix-vector operations, and the Level 3 BLAS perform matrix-matrix operations. Because the BLAS are efficient, portable, and widely available, they are commonly used in the development of high quality linear algebra software, \href{http://www.netlib.org/lapack/}{LAPACK} for example. The libraries are typically linked into your application. For example,if the libraries are stored on \texttt{/usr/local/lib}, use \texttt{-L/usr/local/lib -lf77blas} in your program linker step to link to these libraries. Detailed lists of functions can be found on the links above.

For more information see the \href{http://www.netlib.org/blas/}{Basic Linear Algebra Subprogram homepage} or \href{http://www.netlib.org/blas/faq.html}{BLAS FAQ}.
\hfill \textit{Written by Michael McDonald} \\
\hfill \textit{Category: Supported Libraries \& Compilers}

\section{FFTW}
\url{https://www.sc.fsu.edu/computing/tech-docs/340-fftw3}

The "Fastest Fourier Transform in the West" (FFTW) is a C library for computing discrete Fourier transforms (DFT). Note that there are two main versions of FFTW, version 2 and version 3 which are not compatible. The basic usage of FFTW to compute a one-dimensional DFT of size N is simple, and it typically looks something like this code:
\begin{verbatim}
#include <fftw.h>
...
{
    fftw_complex *in, *out;
    fftw_plan p;
    ...
    in = (fftw_complex*) fftw_malloc(sizeof(fftw_complex) * N);
    out = (fftw_complex*) fftw_malloc(sizeof(fftw_complex) * N);
    p = fftw_plan_dft_1d(N, in, out, FFTW_FORWARD, FFTW_ESTIMATE);
    ...
    fftw_execute(p); /* repeat as needed */
    ...
    fftw_destroy_plan(p);
    fftw_free(in);
    fftw_free(out);
}
\end{verbatim}
Other examples can be found in the \href{http://www.fftw.org/fftw3_doc/index.html}{FFTW documentation}.
\hfill \textit{Written by Michael McDonald} \\
\hfill \textit{Category: Supported Libraries \& Compilers}

\section{CHARMM}
\url{https://www.sc.fsu.edu/computing/tech-docs/354-charmm}

Chemistry at HARvard Macromolecular Mechanics (\href{http://www.charmm.org/}{CHARMM}) is a program for macromolecular simulations, including energy minimization, molecular dynamics and Monte Carlo simulations. The primary concern of CHARMM has been to create a research instrument for theoretical studies of the properties and biological function of molecules. Consequently, emphasis has been placed on the versatility of the program and it is being continually modified as work proceeds into new areas.

\section{Electronic Board}
\url{https://www.sc.fsu.edu/computing/tech-docs/375-panaboard}

The Panasonic Panaboard (UB-2815C) is a Color Scanning System that preserves writings on the board to attached devices: USB Flash Memory/SD Memory Card/printers/computers. Four boards are located in DSL 152, 411, 416 and 499.

\subsection*{Useful Links}
\begin{itemize}
    \item \href{https://panasonic.net/cns/doc/pcc/support/panaboard/ub-2815c/index.html}{Device Information}
    \item \href{https://panasonic.net/cns/doc/pcc/support/panaboard/ub-2815c/index.html#download}{Downloads}
    \item \href{https://panasonic.net/cns/doc/pcc/support/panaboard/ub-2815c/index.html#supply}{Supplies}
\end{itemize}

\subsection*{Save Image to USB Flash Drive}
\begin{enumerate}
    \item Set the power switch to on.
    \item Connect the USB flash memory device to “MEMORY” slot.
    \item Devices formatted using FAT (FAT16) will be recognized by the unit faster. Drives with security feature or larger than 2GB in size may experience difficulty.
    \item Press the Mode Key (return or half moon key) to select the scanning mode.
    \item Press the Start/Stop Key (green button).
    \item Do not remove the USB device until the preview is shown.
    \item Press the Multi-Copy Key (right triangle) to zoom in on the image or to return to the preview.
    \item Press the Mode Key (return or half moon key) again to return to standby mode.
    \item Only remove the USB flash memory device once you have returned to standby mode.
\end{enumerate}
Files are saved in \texttt{ub-2815c} folder in PDF format by default. Can choose JPEG/TIFF if needed.
Time stamp on the scan can be turned off if needed.

\subsection*{Attachments}
\begin{description}
    \item[\href{https://www.sc.fsu.edu/computing/tech-docs/attachments/375-panaboard/Panaboard_Tutorial.pdf}{Tutorial Slides}] Panaboard Tutorial (467 kB)
    \item[\href{https://www.sc.fsu.edu/computing/tech-docs/attachments/375-panaboard/User_Manual.pdf}{Operating Instructions}] User Manual (4695 kB)
    \item[\href{https://www.sc.fsu.edu/computing/tech-docs/attachments/375-panaboard/Panaboard_Info.pdf}{Panaboard Info}] Capabilities and specifications (4277 kB)
\end{description}
\hfill \textit{Written by Xiaoguang Li}

\section{Account Policy}
\url{https://www.sc.fsu.edu/computing/tech-docs/531-account-policy}

The policy is intended to make clear when DSC users are required to change their password, what sponsors are required to do to keep a sponsored account, and to prevent sponsored accounts from becoming "orphaned" if their sponsor leaves the DSC.
(Note: Only DSC faculty and Staff are allowed to sponsor accounts)

\subsection*{Password expiration}
\begin{itemize}
    \item \textbf{180 days} without a password change, user gets a warning requiring the change of password.
    \item \textbf{200 days} without a password change, user account is locked out and the sponsor is emailed about this action. The account owner must contact a member of \href{mailto:sysops@sc.fsu.edu}{sysops@sc.fsu.edu} to have account access restored.
    \item \textbf{350 days} without a password change, the account owner and sponsor get a message warning that the account data will be deleted in 10 days if the password is not changed by contacting a member of the TSG.
    \item \textbf{360 days} without a password change, user account and associated data are deleted from DSC file system.
\end{itemize}

\subsection*{Sponsored account verification}
\begin{itemize}
    \item \textbf{180 days} without sponsor verification, the sponsor gets a warning requiring the verification of continued sponsorship.
    \item \textbf{200 days} without sponsor verification, the sponsored account is locked so that the account owner can no longer log in and the sponsor is informed of this action via email. The sponsor must contact a member of \href{mailto:sysops@sc.fsu.edu}{sysops@sc.fsu.edu} to have account access restored.
    \item \textbf{350 days} without a sponsor verification, the account owner and sponsor get a message warning them that the account and associated data will be deleted from the DSC file systems.
    \item \textbf{360 days} without a sponsor verification, the account and associated data is deleted from DSC file systems.
\end{itemize}

\subsection*{Sponsoring account removal}
\begin{itemize}
    \item \textbf{220 days} without a sponsor password change, the sponsoring and sponsored accounts get a warning
    \item \textbf{290 days} without a sponsor password change, the sponsoring and sponsored accounts are locked
    \item \textbf{350 days} without a sponsor password change, the sponsoring and sponsored accounts get a message warning that the accounts and associated data will be deleted in 10 days.
    \item \textbf{360 days} without a sponsor password change, the sponsoring and sponsored accounts and associated data are deleted from the DSC file systems.
\end{itemize}
\hfill \textit{Written by Xiaoguang Li}

\section{Redhat Software Collections}
\url{https://www.sc.fsu.edu/computing/tech-docs/1017-redhat-software-collections}

Sometimes more recent versions of programming tools are needed. To accomodate that without replacing the base operating system tools, Redhat has implemented software collections. This allows users to run gcc 4.9.x, python 2.7.x, etc. on a per application basis while maintaining well-tested operating system integrity.

The \texttt{scl} utility makes the switching simple.

The current gcc version is 4.4. To use gcc 4.9 in bash shell, run the following commands:
\begin{verbatim}
scl enable devtoolset-4 bash
gcc -v
gfortran -v
\end{verbatim}
Likewise to use python 3.6, run the following:
\begin{verbatim}
scl enable rh-python36 bash
python -V
\end{verbatim}
By exiting that bash shell, users will return back to the default system tools.

Other useful commands:
\begin{verbatim}
cat my_script | scl enable rh-python36 -
scl -l devtoolset-4
\end{verbatim}
The first command runs \texttt{my\_script} using python3.6 from the current shell (note the dash in the end). The second lists the packages included in collection \texttt{devtoolset-4}.

For more information, refer to \href{https://access.redhat.com/documentation/en-us/red_hat_software_collections/2/html/user_guide/index}{Red Hat Software Collections 2.0}
\hfill \textit{Written by Xiaoguang Li}

\section{Jupyter Notebook}
\url{https://www.sc.fsu.edu/computing/tech-docs/1018-ipython-notebook}

"Jupyter Notebook, formerly known as iPython, is an interactive computational environment, in which you can combine code execution, rich text, mathematics, plots and rich media..." Currently IPython Notebook is available on classroom and hallway computers.

To use this tool on hallway computers from user desktop, run the following commands:
\begin{verbatim}
qlogin
scl enable rh-python36 bash
cd directory_with_notebook_files
jupyter-notebook
\end{verbatim}
The second command is not needed if notebook desn't require python 3.6.

For more information, refers to \href{http://jupyter.org/}{IPython Notebook}
\hfill \textit{Written by Xiaoguang Li}

\section{Introduction to CPython}
\url{https://www.sc.fsu.edu/computing/tech-docs/1163-cpython}

This tutorial will run through using cpython on DSC machines.
Click on any image to view it in a lightbox.

\subsection*{The Basics}
For starters, let's get python to run the eponymous hello world program. The first thing to do is navigate to your preferred source code storage location, and write the hello world program.
Then, start the command prompt.
And then navigate to your program location.
Then, run the python interpreter.

\subsection*{Other Features}
At some point you may build up a code library that you would like to reference. Python contains options to ease the reuse of old code. In order to explore these options, lets compute the factorial of five. First, go to your prefered source directory, and write the program. The program is split up into two files: one that contains a factorial algorithm inside a module, and one that makes use of that module. The two files are split into disparate directories.

Start up the command prompt and navigate to the folder containing the main program. Now, whenever python runs, it searches for code libraries based on the environment variables \texttt{PYTHON\_HOME} and \texttt{PYTHON\_PATH}. In order for python to be able to find the factorial function, its folder must be on the \texttt{PYTHON\_PATH}. So, set the variable (on Windows, setting a variable in the command prompt only affects that command prompt).

Now just run the main program, and you should find that 5! = 120.
\hfill \textit{Written by Michael McDonald}

\section{Linux Modules}
\url{https://www.sc.fsu.edu/computing/tech-docs/1177-linux-modules}

On our Linux systems, most software resources are made available via modules. Each modulefile contains the information needed to configure the shell for an application. Once the Modules package is initialized, the environment can be modified on a per-module basis using the \texttt{module} command which interprets modulefiles. Typically modulefiles instruct the module command to alter or set shell environment variables such as \texttt{PATH}, \texttt{MANPATH}, etc. modulefiles may be shared by many users on a system and users may have their own collection to supplement or replace the shared modulefiles.

This allows us to offer a large array of software to many users without the packages interfering with each other.

\subsection*{Using built-in modules}
We provide many built-in modules for loading software packages and compilers. You can see a list of these by running
\begin{verbatim}
$ ls /etc/modulefiles/*
\end{verbatim}
OR
\begin{verbatim}
$ module avail
\end{verbatim}
on any Desktop, Hallway, or Classroom workstation.

Let's say, for example, you wish to use MPI. You can use the \texttt{/etc/modulefiles/mpi/openmpi-x86\_64} to make openmpi available to your shell.

To do this, simply run:
\begin{verbatim}
$ module load mpi/openmpi-x86_64
\end{verbatim}
To show the current list of loaded modules, run:
\begin{verbatim}
$ module list
\end{verbatim}
If you want to unload a module, run:
\begin{verbatim}
$ module unload mpi/openmpi-x86_64
\end{verbatim}

\subsection*{Creating custom modules}
You may wish to customize the Linux environment for your jobs on your own. You can create your own custom module to accomplish this.

First, create a directory for your modules:
\begin{verbatim}
$ mkdir ~/modules
\end{verbatim}
Then, add the module path to the end of your \texttt{\textasciitilde{}/.bashrc} file:
\begin{verbatim}
export MODULEPATH=${MODULEPATH}:${HOME}/modules
\end{verbatim}
or add the following to your \texttt{\textasciitilde{}/.cshrc} or \texttt{\textasciitilde{}/.tcshrc} if you use the tcsh shell:
\begin{verbatim}
setenv MODULEPATH ${MODULEPATH}:${HOME}/modules
\end{verbatim}
A module can be as simple as setting a few environment variables (such as \texttt{PATH} and \texttt{LD\_LIBRARY\_PATH}) or can be complicated Tcl scripts.

You can refer to any of our existing modules in the \texttt{/etc/modulefiles} directory as templates for creating your own. For example, the \texttt{/etc/modulefiles/orca} module includes the following directives:

Provide basic information about what the module does:
\begin{verbatim}
#%Module 1.0
#
#  Orca module
#
module-whatis  "Set path for orca."
\end{verbatim}
Add the ORCA executable directory to the path, so that ORCA runs when the 'orca' command is typed:
\begin{verbatim}
prepend-path   PATH /panfs/storage.local/opt/orca_3_0_0_linux_x86-64
\end{verbatim}
Set an environment variable necessary for ORCA to run:
\begin{verbatim}
setenv         RSH_COMMAND ssh
\end{verbatim}
Running \texttt{module load orca} will make those alterations to the environment. running \texttt{module unload orca} will remove them.

For a more complex example, refer to \texttt{/etc/modulefiles/gaussian09}. This module file examines what architecture the module is run and adjusts its paths accordingly.

For a complete reference of module file directives, refer to the modulefile man page (\texttt{man modulefile}).
\hfill \textit{Written by Michael McDonald}

\section{VirtualBox Issues}
\url{https://www.sc.fsu.edu/computing/tech-docs/1185-virtualbox-issues}

If you have trouble accessing VMs provided, please email \href{mailto:sysops@sc.fsu.edu}{sysops@sc.fsu.edu} for help or search "virtualbox" from the site.

\subsection*{Inaccessible Disk / Existing UUID}
If you should encounter an error saying that the virtual machine disk is inaccessible or that a disk with this UUID already exists, this can be fixed by removing the disk and adding it back.
\begin{enumerate}
    \item Right click on the virtual machine in the left pane of the VirtualBox Manager and click Settings. Navigate to the Storage section.
    \item Right click the .vdi image and click Remove Attachment.
    \item Right click "Controller: SATA" and click "Add Hard Disk." Click "Choose existing disk."
    \item Navigate to the virtual machine .vdi file and press "Open."
\end{enumerate}
\hfill \textit{Written by Xiaoguang Li}

\section{VirtualBox File Management}
\url{https://www.sc.fsu.edu/computing/tech-docs/1186-virtualbox-file-management}

When using Windows virtual machines provided on Linux Workstations, you are strongly recommended to save data per session.
\textbf{All data will be lost} when the virtual machine is rebuilt as needed without notice. There are several ways through which you can save files:
\begin{itemize}
    \item Your home directory
    \item Dropbox
    \item Google Drive
    \item One Drive
    \item USB
\end{itemize}
To save works in your home directory, you need to set it up \textbf{before} starting the virtual machine.
\begin{enumerate}
    \item Start "Oralce VM VirtualBox"
    \item Select VM to be used, open “Settings”, go to Shared Folders
    \item Click on the icon with blue folder and plus sign
    \item Folder Path: \texttt{/home/username/...}
    \item Folder Name: MyDescription
    \item Check "Make Permanent"
    \item Click on "OK" to finish or add other shared folders if needed
\end{enumerate}
Now you are ready to start Windows VM and save files in the home directory: under "Network / VBOXSVR / MyDescription", the UNC equivalent: \texttt{\textbackslash{}\textbackslash{}VBOXSVR\textbackslash{}MyDescription}.

To save files on USB, your can use "Shared Folders" similar to home directory.
\hfill \textit{Written by Xiaoguang Li}

\section{Linux Commands Cheat Sheet}
\url{https://www.sc.fsu.edu/computing/tech-docs/1302-linux-commands-cheatsheet}

Below are a number of commonly used commands in unix. For almost every command, there are additional flags that can be used to modify the command's behavior. For example, typing \texttt{ls} in the terminal displays the files contained within the current working directory, but \texttt{ls -l} displays them in a list format. Additionally, \texttt{ls -lt} displays them in a list, but also in order of modification time. Information on what flags can be used, along with general explanations about the commands, can be found by typing \texttt{man} before the command. Ex: \texttt{man ls}.

\subsection*{Navigation}
Basic navigation is performed using a few commands:
\begin{description}
    \item[\texttt{cd}:] Short for 'change directory', changes the current working directory.
    \item[\texttt{ls}:] Short for 'list', lists all of the files in the current working directory.
    \item[\texttt{cwd}:] Short for 'current working directory', lists the absolute path to the directory the user is in.
    \item[\texttt{clear}:] Clears the terminal of all text, although the previous text can still be seen by scrolling up.
\end{description}
In navigating a Unix file structure, a number of shorthand notations should be mentioned. Example usage of the above commands are shown here as well:
\begin{description}
    \item[\texttt{\textasciitilde{}}:] The location of a user's home directory. When logging into a unix machine, you are likely going to begin in your home directory, where your personal files will be located. Typing \texttt{ls} while in the home directory will display folders like \texttt{Desktop}, \texttt{Pictures}, \texttt{Downloads}, and \texttt{Documents}. To move into one of these folders, one can type, e.g., \texttt{cd Downloads}. Additionally, to display the contents of the \texttt{Downloads} folder without navigating to it, type \texttt{ls Downloads}.
    \item[\texttt{/}:] The machine's root directory. This is the highest level directory that contains all other files and folders in the machine. Programming libraries, system files, etc are found within the folders here. Typing \texttt{cwd} in any directory will display the path from the root directory to the current working directory. For example, if the user's name was 'tom', typing 'cwd' while in the 'Downloads' folder will display \texttt{/home/tom/Downloads}.
    \item[\texttt{.}:] The current directory is represented with a simple period. Typing \texttt{cd .} moves you to the current working directory, i.e. to where you are now.
    \item[\texttt{..}:] The directory above the current working directory. Ex: If you're in the directory \texttt{\textasciitilde{}/Downloads/}, typing \texttt{cd ..} will bring you to the home directory.
    \item[\texttt{-}:] This represents the last directory you were in. Ex: If you were in \texttt{\textasciitilde{}/Downloads} and moved to \texttt{/lib64/}, typing \texttt{cd -} would move you back to \texttt{\textasciitilde{}/Downloads} and \texttt{ls -} would display the files within \texttt{\textasciitilde{}/Downloads}.
\end{description}

\subsection*{Creating, deleting, moving, renaming files/folders}
\begin{description}
    \item[\texttt{mkdir}:] Short for 'make directory', creates a directory with a given name. Ex: \texttt{mkdir \textasciitilde{}/Desktop/DSC} creates a folder 'DSC' in the 'Desktop' folder in your home directory.
    \item[\texttt{rmdir}:] Short for 'remove directory', deletes a directory. Note that there are no backups, so be careful when using this command.
    \item[\texttt{touch}:] Generally used to create a file with a given name. Ex: \texttt{touch \textasciitilde{}/Downloads/example.txt} creates an empty file named 'example.txt' in the Downloads folder and closes it. This command was originally used to update the time a file was accessed/modified -- 'touching' it by opening it if it exists, not modifying anything, and closing it.
    \item[\texttt{cp}:] Short for 'copy', simply copies a file. Ex: If a file named 'example.txt' were in the Downloads folder and the Downloads folder is the current working directory, we could copy it into to the Desktop by using \texttt{cp example.txt \textasciitilde{}/Desktop/}. We could also rename the copy something else by using \texttt{cp example.txt \textasciitilde{}/Desktop/example2.txt}. To copy a folder, the \texttt{-rf} flag is required, as it needs to recursively copy all the files within the folder if they are present.
    \item[\texttt{mv}:] Short for 'move', moves a file from one place to another, and is often used to rename a file. This command effectively does the same thing as 'cp', only that the original file is removed.
    \item[\texttt{rm}:] Short for 'remove', deletes a file. Note that there are no backups of these files, so be careful when using this command. When deleting a directory, use \texttt{rm -rf} or \texttt{rmdir}.
    \item[\texttt{locate}:] Searches the entirety of the file structure for a file/directory with a given name.
\end{description}

\subsection*{Additional tips}
\begin{description}
    \item[Tab-completion:] When typing in commands, especially when trying to navigate file structures, using the tab key can auto-complete the names of files/folders if they are present. If nothing auto-completes after typing tab once, type it again to see all possible options. Similar to our example above under the 'cp' explanation, if we were to type 'ls \textasciitilde{}/Desktop/example' and hit tab twice, it would show 'example.txt' and 'example2.txt'. Typing in '.' would then let it auto-complete the word 'example.txt' after hitting tab again.
    \item[Arrows:] Use the up arrow to access previous commands that have been typed in. Using the down arrow after using the up arrow will go back down the list of commands.
    \item[Permissions:] A fundamental part of the unix file system is permissions (which cannot be fully explained here) and is generally accessed by using 'ls -l', i.e. displaying a folder's contents in list format. If you cannot read or modify a file, check the permissions. The permissions are on the left and dictate who can and cannot read (r), write (w), or execute (x) files in Unix (called access modes). Ex: If one has read access, the 'r' will be displayed. If one does not have read access, the 'r' column will be replaced with a dash '-'. Further, these are split into three sets. The first set of access modes are the permissions of the file's owner (the first name beside the permissions). The second set are the permissions of the file's group (the second name beside the permissions). The third set is the permissions for everyone else. Ex: If a file has only read access, the permissions would read '-r--r--r--'. This means that the user, the user's group, and anyone else can open and read the contents of the file. If it had read, write, and execute access for only the user, it would read '-rwxr--r--'. Here the user can open the file to read its contents, save edits to it, and execute it (if it's a program or a script), but everyone else can only read its contents. If it further had execute access for the user's group, it would read '-rwxr-xr--'. Finally, notice that all of these examples have a leading '-'. This is replaced with a 'd' only if it is a directory. In that case, the number beside the permissions would display how many files are in the folder.
\end{description}
A final example for a folder named 'DSC' with 3 files in it owned by a user named 'tom' in the group 'users' with read, write, and execute permissions for the user 'tom', read and execute permissions for the group 'users', and read permissions for everyone else, with a byte size of 1000 and last modified on August 1st, 2017, would be:
\begin{verbatim}
  drwxr-xr-- 3 tom users 1000 Aug 1 2017 DSC
\end{verbatim}
\hfill \textit{Written by Xiaoguang Li}

\section{Shared Computer Policy}
\url{https://www.sc.fsu.edu/computing/tech-docs/1333-shared-computer-policy}

DSC offers hallway and classroom computers as shared resources to all users. For fair use of the public systems, hallway computers should be used interactively and remote sessions should not include long running (8 hours or more) or CPU intensive jobs; classroom computers may be used for remote jobs during non-business hours (7pm - 7am). Technical Support Group reserves the rights to terminate jobs on shared systems if they are deemed as abusive.

If you require more resources, please consider \href{https://rcc.fsu.edu/}{Research Computing Center} or other dedicated systems through faculty members.
\hfill \textit{Written by Xiaoguang Li}

\section{Python Jungle}
\url{https://www.sc.fsu.edu/computing/tech-docs/1416-python-jungle}

Python 2 will soon be obsolete. Python 3 can be a challenge to navigate. It is strongly recommended to use virtual environment to manage python projects. Don't use system python for your projects.

Here is an example to start a project named isc3313. This approach will isolate python issues to the specific environment without impacting anything else.
\begin{verbatim}
scl enable rh-python36 bash
virtualenv isc3313
source isc3313/bin/activate
\end{verbatim}
Manage anything needed for this project
\begin{verbatim}
pip install tensorflow...
\end{verbatim}
To exit the environment, run
\begin{verbatim}
deactivate
\end{verbatim}
To determine if you are running a particular Software Collection:
\begin{verbatim}
echo $X_SCLS
\end{verbatim}
To test a Software Collection state, use "\texttt{scl\_enabled rh-python36}" exit code.

To run script using python3.6 through Redhat Software Collection, use the following script template
\begin{verbatim}
#!/usr/bin/scl enable rh-python36 -- python3
import sys
version = "Python %d.%d" % (sys.version_info.major, sys.version_info.minor)
print("You are running Python",version)
\end{verbatim}
For more details, refers to \href{https://developers.redhat.com/products/python/hello-world}{Python 3 on RHEL}
\hfill \textit{Written by Xiaoguang Li}

\section{PGI Community Edition}
\url{https://www.sc.fsu.edu/computing/tech-docs/1418-pgi-community-edition}

pgcc is available on classroom and hallway computers. To get it for your desktop computers, please request it at \href{mailto:sysops@sc.fsu.edu}{sysops@sc.fsu.edu} including your desktop hostname.

To access Portland Group compiler, use
\begin{verbatim}
module load PrgEnv-pgi/19.4
pgcc myprog.c
\end{verbatim}
Example files are found at \texttt{/usr/common/pgi2019\_194/linux86-64-llvm/2019/examples}. For "module" command information, please visit \href{https://www.sc.fsu.edu/computing/tech-docs/1177-linux-modules}{Linux Modules}.
\hfill \textit{Written by Xiaoguang Li}

\section{Upload files to DSC computers}
\url{https://www.sc.fsu.edu/computing/tech-docs/1653-upload-files-to-dsc-computers}

From off campus, you are able to upload files from personal computers or laptops to DSC computers.

\subsection*{Mac}
Open a terminal window in Mac, and first create the tunnel of transfer by providing your fsuid and remote computer name:
\begin{verbatim}
ssh -f -N -L 2222:remote_computer_name.sc.fsu.edu:22 your_fsuid@pamd.sc.fsu.edu
\end{verbatim}
Once the tunnel is established by providing credentials, you can start to upload or download as follows:
\begin{verbatim}
rsync -avz -e 'ssh -p 2222' /local_upload_folder fsuid@localhost:/remote_destination_directory
rsync -avz -e 'ssh -p 2222' fsuid@localhost:/remote_directory_to_be_downloaded /local_destination
\end{verbatim}

\subsection*{Windows 10}
Open a command Window, and establish the tunnel (NOTE the option -L):
\begin{verbatim}
ssh -f -N -L 2222:remote_computer_name.sc.fsu.edu your_fsuid@pamd.sc.fsu.edu
\end{verbatim}
Now you can upload or download files and/or folders as follows:
\begin{verbatim}
scp -r -P 2222 /local_source fsuid@localhost:/remote_destination
scp -r -P 2222 fsuid@localhost:/remote_directory_to_be_downloaded /local_destination
\end{verbatim}
fsuid password may be required.
\hfill \textit{Written by Xiaoguang Li} \\
\hfill \textit{Category: Computing}

\section{OneDrive for Ubuntu}
\url{https://www.sc.fsu.edu/computing/tech-docs/1672-onedrive-for-ubuntu}

If you want to synchronize or download files in OneDrive to DSC desktops running Ubuntu, you can follow the following guidelines on Ubuntu client computers.

\textbf{NOTE:} If you have a large OneDrive quota, you are strongly encourage to synchronize ONLY the folder / directory you need. Fail to do so may result in your DSC account reach its storage quota. That will freeze your account!

To setup link to your cloud storage in OneDrive, run the following command and follow instructions on the screen:
\begin{verbatim}
onedrive --monitor
\end{verbatim}
Press Ctrl + C if you don't want to synchromize everything in the Cloud space.
A sample run is as follows:
\begin{verbatim}
onedrive --monitor
Configuring Global Azure AD Endpoints
Authorize this app visiting:

https://login.microsoftonline.com/common/oauth2/v2.0/authorize?client_id=...

Enter the response uri: https://login.microsoftonline.com/common/oauth2/nativeclient?code=...
Initializing the Synchronization Engine ...
Initializing monitor ...
OneDrive monitor interval (seconds): 300
Creating local directory: ...
......
\end{verbatim}

To synchronize only one folder "dsc\_files" from the Cloud / OneDrive
\begin{verbatim}
onedrive --synchronize --single-directory 'dsc_files'
\end{verbatim}
To download files / folders only:
\begin{verbatim}
onedrive --synchronize --single-directory 'dsc_files' --download-only
\end{verbatim}
To upload to OneDrive without deleting files in cloud that are missing from local directories
\begin{verbatim}
onedrive --synchronize --upload-only --no-remote-delete
\end{verbatim}
To show OneDrive setting:
\begin{verbatim}
onedrive --display-config
\end{verbatim}
\hfill \textit{Written by Xiaoguang Li}

\section{Logging in to SC systems}
\url{https://www.sc.fsu.edu/computing/tech-docs/174-logging-in-to-sc-systems}

Computing resources can be accessed both through the SSH gateway servers (\texttt{pamd.sc.fsu.edu}) and through the FSU VPN (\texttt{vpn.fsu.edu}).

\subsection*{SSH connection to pamd.sc.fsu.edu}
Accessing the SSH gateway server is simple. Use your workstation's SSH client to directly log in to \texttt{pamd.sc.fsu.edu}. For example:
\begin{verbatim}
% ssh pamd.sc.fsu.edu
\end{verbatim}
You will be prompted for a username and password. If the login information you entered is correct, you will have access to a shell on \texttt{pamd.sc.fsu.edu}. From this shell on \texttt{pamd.sc.fsu.edu}, users can access public computing resources by running command "hallway" or "classroom".

\textbf{Never} use the \texttt{pamd.sc.fsu.edu} servers to run processor intensive jobs.

More information on SSH access is available \href{https://www.sc.fsu.edu/computing/tech-docs/345-using-ssh}{here}.
\hfill \textit{Written by Administrator}

\section{Storage, Backup and Recovery}
\url{https://www.sc.fsu.edu/computing/tech-docs/175-storage-backup-and-recovery}

Only DSC accounts with valid FSU accounts are backed up - temporary accounts, eg: \texttt{wkshop01}, are not backed up.
SC provides centralized file storage via \href{http://www.sun.com/software/solaris/zfs.jsp}{ZFS} file system. The system provides a RAID-based storage to ensure against the loss of two disk drives.

Share file system provides home directories and research space for all desktops and pamd.
\begin{itemize}
    \item Users have a fixed disk quota (see below) on their home directory, which is based on our current backup capacity. As our backup capacity increases, so will user quotas on home directories. Disk quotas are not increased otherwise.
    \item Users have other options if more disk space is needed than is available in their home directory.
    \item Request a research partition by sending a message to \href{mailto:sysops@sc.fsu.edu}{sysops@sc.fsu.edu}. A research partition with a maximum capacity of 900 GB will be given to SC faculty members upon request.
    \item Research partitions are not backed up.
    \item Inactive research partitions will be reclaimed following confirmation from the faculty member.
    \item Files on individual desktop systems are not backed up.
    \item Files over 500 MB are not backed up.
    \item Users wishing to protect local files can save these files to the shared file system by mounting their user volume locally (via Samba, sshfs, scp, rsync or other mechanisms) and copying the important files to this directory.
\end{itemize}

\subsection*{Recovery}
To recover lost files from the last three weeks, please send a request to \href{mailto:sysops@sc.fsu.edu}{sysops@sc.fsu.edu} stating the date and files or directories to be restored.
No archival data beyond 21 days is maintained.

\subsection*{Backup Technology}
The backup system consists of a 10 TB iSCSI configured in Raid 5 with a hot spare with a \href{http://www.dell.com/us/en/k-12/servers/pedge_1950/pd.aspx?refid=pedge_1950&cs=22&s=k12}{Dell 1950} acting as the hosting node. Backups are accomplished with \href{http://samba.anu.edu.au/rsync/}{rsync}.
\hfill \textit{Written by Xiaoguang Li} \\
\hfill \textit{Category: Data Storage}

\section{Joomla Web Page Tips and Tricks}
\url{https://www.sc.fsu.edu/computing/tech-docs/176-web-pages}

\subsection*{Creating New FAQs}
The FAQ section is reserved for Administrative frequently asked questions. New FAQ entries must contain the following HTML code as its template:
\begin{verbatim}
<p><strong>Q:</strong> ?</p>
<p><strong>A:</strong> </p>
\end{verbatim}

\subsection*{Populating Metadata Information}
Every article under Joomla can have metadata associated with it. This metadata is used to display Related Articles below the current article. It is important to \textbf{NOT} use generic keywords in the metadata. This will result in unrelated articles being displayed instead of only the relevant ones. To circumvent this problem, only use keywords which are unique to the article you are working on. Refrain from using commonly used words (\textit{the, and, who, what, when, etc.}).
\href{http://www.sc.fsu.edu/images/stories/techdocs/joomla/metadata.png}{Click here to see an example}.

\subsection*{Adding the Lightbox Effect}
To add a lightbox effect to images the following class tag must be added.
\paragraph{Example Code}
\begin{verbatim}
<a class="modal" href="/images/joomla_logo_black.jpg">
<img src="/images/joomla_logo_black.jpg" 
width="100" height="26" /></a>
\end{verbatim}
\paragraph{Working Example}
\href{/images/joomla_logo_black.jpg}{Joomla Logo}

\subsection*{Adding Slider Panes}
To add slider panes to web pages the following code template should be used. An example of this feature is used in this current page.
\begin{verbatim}
{slider=Heading 1}
content for heading 1
{slider=Heading 2}
content for heading 2
{/sliders}
\end{verbatim}

\subsection*{Showing Code in Articles}
To show code in articles you must use the \texttt{"pre xml:html"} tags. This activates the Code Hightlighter (GeSHi) which nicely displays your code with syntax highlighting. Replace the starting and ending quotes in the below example with less than and greater than characters, respectively.
\begin{verbatim}
<pre xml:html>
  ...
</pre>
\end{verbatim}
\hfill \textit{Written by Michael McDonald}

\section{Printing}
\url{https://www.sc.fsu.edu/computing/tech-docs/179-printing}

To use the department's printers you must be connected to the local network. This mean that if you are working off campus or using the campus wireless network you must first connect to the department's VPN (see VpnSetup for more details). Assume you have network access and permission to the printers, you may \href{https://www.sc.fsu.edu/computing/tech-docs/206-network-printer-setup}{set them up as described here}.

Note: The department \href{http://printers.sc.fsu.edu/printers/}{print server} uses the "CUPS" software. Consolidation of printing services will change this.

\subsection*{Poster Printing}
Posters are only printed for Department of Scientific Computing faculty and students.
Requests to print posters must be made/endorsed by an SC faculty member and are approved by the SC Director.

\subsection*{Printer etiquette}
\begin{itemize}
    \item When you submit a job to a printer, please retrieve your output as soon as possible. Output which is left in the copy room over 24 hours will be thrown away.
    \item When you do pick up your output, and it is located within a large stack of pages, please do so carefully so as to not disturb the output of other users. Don't pull out pages while the printer is printer and then try to put them back. This leads to "shuffling" the pages and leads to a lot of confusion. Please dispose of unnecessary pages in the recycling boxes located on the floor beneath the counters, and in the trays designated for this purpose nearby all of the printers.
    \item Please respect the resourses. Do NOT try to clear jams, or correct problems with the printers, even if on weekends. Our printers are not covered under warranty, and repairs are very expensive. Send email to sysops or to Dana Lutton.
\end{itemize}

\subsection*{Known Printer Problems}
\begin{itemize}
    \item Acrobat files may not print from Windows to the pr483 QMS 860 printer. This printer is a Level 2 PostScript and most current Acrobat files contain Level 3 PostScript commands in them.
    \item The pr473 HP printer is also a Level 2 PostScript printer and will not print Level 3 PostScript files. You will need to set your application to print Level 2 PostScript only.
    \item It is recommended that you not print manual paperfeed jobs from UNIX where the printer must pause and wait for the manual feed tray to be loaded.
    \item Level 3 PostScript commands primarily deal with the "color" definitions and commands, and are the primary cause of printing problems on Level 2 PostScript printers.
    \item Problems printing PhotoShop files. Some Adobe PhotoShop files contain binary information prior to the \%!PS, and following the \%EOF. To properly print these files you must remove the binary information from the begin and the end of the file. This especially applies to people who must incorporate them into (La)TeX documents using dvips.
    \item All PostScript files must contain \%!PS as the first 4 characters. If your file doesn't print, this is a likely cause.
    \item PostScript files received by email may contain aberrant characters, at the end of line, or beginning of new lines. Binary portions of files received by email may be truncated, and will not print.
    \item Problems with lpr and EPS files. EPS files are "encapsulated" and contain no "printer specific" information. To print an EPS file using the lpr file.eps command you need to have a "utility wrapper" surrounding this code.
    \item Web pages and PDF files containing graphics are a whole "other" issue -- many of them cannot be printed because they were incorrectly made to begin with.
    \item If a PDF was made with special fonts that are not installed on a printer the file will not print correctly. Authors should be encouraged to embed all fonts into their pdf document to avoid this problem.
\end{itemize}
\hfill \textit{Written by Xiaoguang Li}

\section{vscode Proxy through SSH Tunnel}
\url{https://www.sc.fsu.edu/computing/tech-docs/1831-vscode}

If you want to use vscode to connect to \texttt{targetHost.sc.fsu.edu} from a computer at home through proxy \texttt{pamd.sc.fsu.edu}, please use the following steps.
Feel free to email \href{mailto:sysops@sc.fsu.edu}{sysops@sc.fsu.edu} if you have any difficulty.

\subsection*{Configure SSH for ProxyJump}
You'll want to add a configuration for \texttt{targetHost.sc.fsu.edu} in your SSH config file to use \texttt{pamd.sc.fsu.edu} as an intermediate (jump) host.
Open or create the SSH configuration file using whatever text editor you prefer, we use nano as an example:
\begin{verbatim}
nano ~/.ssh/config
\end{verbatim}
Add the following settings to the SSH configuration file \texttt{\textasciitilde{}/.ssh/config}:
\begin{verbatim}
Host pamd
    HostName pamd.sc.fsu.edu
    User fsuid
Host targetHost
    HostName targetHost.sc.fsu.edu
    User fsuid
    IdentityFile ~/.ssh/<private_key_for_targetHost>
    ProxyJump pamd
\end{verbatim}
Make sure the \texttt{IdentityFile} points to the correct SSH private key for each host.

\subsection*{Set Up the SSH Extension in VS Code}
Install the \href{https://marketplace.visualstudio.com/items?itemName=ms-vscode-remote.remote-ssh}{Remote - SSH} extension for VS Code if you haven't already:
\begin{enumerate}
    \item Go to Extensions and search for "Remote - SSH."
    \item Install the extension.
    \item Open the Command Palette in VS Code and search for "Remote-SSH: Connect to Host..."
    \item Type \texttt{targetHost.sc.fsu.edu} (or the alias name you set in \texttt{\textasciitilde{}/.ssh/config} for targetHost) and select it.
\end{enumerate}
VS Code will automatically use the SSH config you defined, routing the connection through \texttt{pamd} to reach \texttt{targetHost}.

\subsection*{Verify Connection}
Once connected, VS Code should open a remote session on targetHost. You can confirm the connection by checking the VS Code terminal, which should indicate it's connected to targetHost. Now you can open files, run commands, and work as if you were directly connected to targetHost.
\hfill \textit{Written by Xiaoguang Li}

\section{VPN Access}
\url{https://www.sc.fsu.edu/computing/tech-docs/204-vpn-access}

A VPN connection will allow you to join our network even when you are working off campus. After you establish a connection, your network communication is securely routed through the VPN sever when communicating with FSU endpoints.

\subsection*{FSU VPN Service Documentation}
\begin{enumerate}
    \item Login to the FSU-DSC VPN here, \url{https://vpn.fsu.edu/dsc}
    \item Use your FSUID and FSUID-password
    \item Click the \texttt{AnyConnect} link in the left-hand menu
    \item Click the \texttt{Start AnyConnect} link in the middle frame
    \item The VPN client will automatically try and detect your operating system
    \item Download \& Install the VPN client provided by link
    \item Use the address \texttt{vpn.fsu.edu/dsc} within the client
    \item Enter your FSUID (Username) and FSUID-password (Password) in the VPN client prompt
    \item Your FSUID is saved for convenience
    \item Passwords will *never* be saved
    \item Done
\end{enumerate}
\hfill \textit{Written by Michael McDonald} \\
\hfill \textit{Category: VPN}

\section{Network Printer Setup}
\url{https://www.sc.fsu.edu/computing/tech-docs/206-network-printer-setup}

If the information provided here is not enough, please contact \href{mailto:sysops@sc.fsu.edu}{sysops@sc.fsu.edu}.

\textbf{Note:} Due to the print quota scheme, print jobs will simply be dropped if your login name is not found in the LDAP directory. One solution is to use -U option of lpr command, for example, \texttt{lpr -U username file2Print.ps}. The best solution is to implement LDAP authentication or change the login name to SC uid.

\subsection*{Add a Network Printer for Windows 10/7}
To determine the make and model of the printer you wish to add, go to \url{http://printers.sc.fsu.edu/printers}.
For this example, we will use Ricoh MP 2555, mfp473
\begin{enumerate}
    \item Go to 'Printers and Faxes
    \item Click on 'Add a Printer'
    \item Click Next through the Add a Printer Wizard
    \item Click the button for 'A Network Printer....'
    \item In the 'Specify' a Printer box, click on the button for 'Connect to a printer on the internet'
    \item In the space provided, type: \url{http://printers.sc.fsu.edu:631/printers/mfp473}
    \item Click 'Next'
    \item Select your printer make and model from the list to install the correct print driver. See the end of the page for Windows drivers.
    \item Complete the Add a Printer Wizard
\end{enumerate}
\textbf{NOTE:} In this case, and a few others the exact driver is not listed in the Windows Driver Menu. Select the closest driver to your printer model as possible. For the 4300 Series use the 'HP Laser Jet 4100 series PS' driver. Additionally, the current version of the CUPS Printing Software support both PostScript and PCL Drivers.

\subsection*{Add a Network Printer for MAC OS X}
To determine the make and model of the printer you wish to add, go to: \url{http://printers.sc.fsu.edu/printers}.
\begin{enumerate}
    \item Open System Preferences and click "Print \& Fax".
    \item Click Printing and then click the "+" sign to open the "Printer Browser" box.
    \item For "Protocol", select LPD
    \item For "Address" enter printers.sc.fsu.edu
    \item For "Queue" enter the printer name (EX. mfp420, mfp473, mfp150v)
    \item For "Printer Name" , type the name of the printer (Should be the same name as in "Queue")
    \item For "Location", you can fill in or leave blank.
    \item For "Print Using", Scroll through the list of printer manufacturers and select the maker of the printer.
    \item Next, Scroll through the list of printer models and select your model to install the correct print driver.
    \item Finally, click on "Add" to include the printer on your list of available printers.
\end{enumerate}
\textbf{NOTE:} When the exact driver is not listed in the Driver Menu, select the closest driver to your printer model as possible. Additionally, the current version of the CUPS Printing Software only support PostScript printing.

\subsection*{Network Printers for UNIX}
In most cases, network printers are configured and ready to go as they are. If you need to setup them up for some reason, here are the steps.
DSC uses the Common Unix Printing System (CUPS) to manage printing jobs. If you already have CUPS installed on your system, place the following into your \texttt{client.conf} file (usually \texttt{/etc/cups/client.conf})
\begin{verbatim}
ServerName printers.sc.fsu.edu
\end{verbatim}
Remember to restart cupsd after saving the changes to the file.
You can print to any printer through most of the application printer interface. To print or manage print jobs using command line:
\begin{verbatim}
lpr -P mfp420 filename
lpq -P mfp420
lprm -P mfp420 jobID
\end{verbatim}

\subsection*{Drivers Download}
\begin{itemize}
    \item \href{https://www.ricoh-usa.com/en/support-and-download?keyword=MP%202555}{Ricoh MP 2555 Drivers}
    \item \href{https://www.ricoh-usa.com/en/support-and-download?keyword=IM%20C3000}{Ricoh IM C3000 Drivers}
    \item \href{https://www.ricoh-usa.com/en/support-and-download}{Ricoh Support and Download}
\end{itemize}
\hfill \textit{Written by Xiaoguang Li}

\section{Version Control Systems}
\url{https://www.sc.fsu.edu/computing/tech-docs/225-svn-information}

Will your code be for free / open source software?
If \textbf{YES}, then you should first try one of these free source code repositories:
\begin{description}
    \item[\href{https://github.com/}{github}] "GitHub is how people build software. With a community of more than 15 million people, developers can discover, use, and contribute to over 38 million projects using a powerful collaborative development workflow." (\href{https://github.com/about}{source})
    \item[\href{https://sourceforge.net/}{SourceForge.net}] "SourceForge.net is the world's largest open source software development web site. We provide free services that help people build cool stuff and share it with a global audience." (\href{https://sourceforge.net/about}{source})
\end{description}
If \textbf{NO}, then we recommend using \textbf{git}.

\subsection*{Quick start for Git}
\paragraph{Creating and commiting on pamd.sc.fsu.edu}
\begin{verbatim}
$ cd (project-dir)
$ git init
$ (add some files)
$ git add .
$ git commit -m 'Initial commit'
\end{verbatim}

\paragraph{Cloning and Creating a Patch}
\begin{verbatim}
$ git clone ssh://{mylogin}@pamd.sc.fsu.edu/panfs/\
   panasas1/research/{mydir}/{mygitproj}
$ cd {mygitproj}
$ (edit files)
$ git add (files)
$ git commit -m 'explain what I chanded'
$ git format-patch origin/master
\end{verbatim}
For more information about how to use git see the \href{https://git-scm.com/book/en/v2}{Git Community Book}. The section on \href{https://git-scm.com/book/en/v2/Distributed-Git-Distributed-Workflows}{Distributed Workflows} is especially useful.

Note: Once your code is \textit{presentation ready} we recommend sharing the URL and details of your project for posting to our \href{http://www.sc.fsu.edu/research/departmental-software}{Department software collection here}. You can email \href{mailto:sysops@sc.fsu.edu}{sysops@sc.fsu.edu} with the details.
\hfill \textit{Written by Marcelina Lapuz Nagales}

\section{Thunderbird FAQ}
\url{https://www.sc.fsu.edu/computing/tech-docs/233-thunderbird-faq}

Thunderbird works largely the same on all platforms: Linux, Windows and Macintosh. For users working on more than one platform, Thunderbird can be a significant benefit comparing with other email clients. The solutions given here may vary slightly in naming and layout depending on the platforms. If you need to setup Thunderbird, consult \href{https://www.sc.fsu.edu/computing/tech-docs/178-email}{this page} first.

\subsection*{How to turn on/off email forwarding from fsu.edu?}
\begin{enumerate}
    \item Login \url{http://webmail.fsu.edu}
    \item From top menus, click Options/Mail/Settings
    \item Check/uncheck "Enable forwarding"
\end{enumerate}
Be careful not to create forwading loops with other email accounts!

\subsection*{How to display the size and number of messages in a folder?}
\begin{enumerate}
    \item Go to "Preferences / Advanced / General"
    \item Check "Show expanded colums in the folder pane"
    \item In the folder pane, click on the icon next to "Name" column and select Total and/or Size
    \item Likewise each message size can be displayed by clicking on the icon in the email pane
\end{enumerate}

\subsection*{How to setup reply above/before the original message?}
\begin{enumerate}
    \item Go to "Account Settings... / Composition \& Addressing" of the mailbox in question
    \item Check "Automatically quote the original message when replying"
    \item Then, "Start my reply above the quote"
    \item and place my signature "below my reply (above the quote)" if preferred
\end{enumerate}

\subsection*{How to setup FSU address book?}
\begin{enumerate}
    \item Go to "Preferences / Composition / Addressing"
    \item Check "Directory Server"
    \item Click on "Edit Directories..." and "Add" buttons
    \item Enter the fields as shown (Hostname: addressbook.fsu.edu) below and make sure to use your own "Bind DN: uid=fsuid,ou=people,dc=fsu,dc=edu".
    \item After saving the settings, go to "Offline" tab and click on "Download Now", which should help with adddres completion performance.
    \item Under the "Directory Server" drop-down menu, select the address book just created.
    \item Now you should be able to get automatic address completion for FSU email addresses.
\end{enumerate}

\subsection*{How to keep local copies of IMAP messages?}
\begin{enumerate}
    \item Go to "Account Settings... / Offline \& Disk Space" of the mailbox in question
    \item Check "When I create new folders, select them for offline use"
    \item Click on "Select folders for offline use..." button and check folders you want to keep local copies.
\end{enumerate}
This is helpful on laptops so you can still read old messages when network connections are not available. If you wonder where the messages are saved, go to "Accoutn Settings / Server Settings", "Local directory" should display the folder.

\subsection*{How to change the original message indicator from the colored bars to the traditional >?}
Go to "Account Settings... / Composition \& Addressing"
Uncheck "Compose messages in HTML format"
This is equivalent to "\texttt{mail.identity.default.compose\_html = false}"

\subsection*{How to setup calendar in Thunderbird?}
Download \href{https://addons.thunderbird.net/en-us/thunderbird/addon/lightning/}{Lightning plugin} for Thunderbird, which also provide "Tasks" service
One main drawback of the Tasks is the inability to save tasks and to transport them. If you have Gmail account, you are encouraged to use \href{https://www.google.com/calendar}{Google Calendar} and its Tasks service instead.

Lightening plugin serves as a portal for various calendars you may have: FSU calendars, Google calendars and others. If you prefer a separate application altogether, use \href{http://www.mozilla.org/projects/calendar/sunbird.html}{Sunbird}. It is better to use Lightning plugin to only view the various calendars since changing calendar events may require complicated authentications. Its reminders may not be working properly each time you reload the calendars. SyncCGP plugin may help where Lightning is lacking, but we will try to keep things simple for now.

Install Lightning as follows
\begin{enumerate}
    \item In Thunderbird, go to "Tools / Add-ons"
    \item Click "Install" button, and select the file you downloaded and click "OK". Restart Thunderbird.
    \item Under Calendar, go to "Calendar / New Calendar..."
    \item Select "On the network" and click "Next".
    \item For Google calendars, select "iCalendar" and enter in the "Location" with the private URL provided by Google unless your calendar is publicly available.
    \item For FSU calendars, select "Sun Java System Calendar Server (WCAP)" with "Location: \url{https://webmail.fsu.edu:83}" as shown below.
\end{enumerate}
FSU calendars can be configured via \url{http://webmail.fsu.edu}. Refers to \href{https://www.sc.fsu.edu/computing/tech-docs/243-fsu-calendar}{this}.
\hfill \textit{Written by Xiaoguang Li}

\section{Outlook FAQ}
\url{https://www.sc.fsu.edu/computing/tech-docs/234-outlook-faq}

Make sure you turn off forwarding from fsu.edu to scs.fsu.edu and turn on forwarding from scs.fsu.edu to fsu.edu. The former is to prevent forward loops, the latter is to prevent duplicates when moving and synchronizing the two mailboxes or folders. Here is a brief summary of these two steps.

\subsection*{Turn off forwarding from fsu.edu}
\begin{enumerate}
    \item Login \url{http://webmail.fsu.edu}
    \item From top menus, click Options/Mail/Settings
    \item Uncheck "Enable forwarding" or at least remove \texttt{fsuid@scs.fsu.edu} and other CSIT and SCRI accounts from "Mail Forwarding list:" if any
\end{enumerate}

\subsection*{Turn on forwarding from scs.fsu.edu}
\begin{enumerate}
    \item Login \url{http://mail.scs.fsu.edu}
    \item Click on Options/Filters from the left menu and Forward from the top menu
    \item Type your \texttt{fsuid@my.fsu.edu} in "Address(es) to forward to:" field
\end{enumerate}
\hfill \textit{Written by Xiaoguang Li}

\section{Integrated Technology Plan}
\url{https://www.sc.fsu.edu/computing/tech-docs/242-it-plan}

\section{Multifunction Printers}
\url{https://www.sc.fsu.edu/computing/tech-docs/268-mfps}

\subsection*{Ricoh MP 2555 / IM C3000}
No code required for printing and scanning. To make copies at the console, enter your code.
\subsubsection*{Scan documents to email address}
At the copier console, press "Scanner (Classic)" / "Search Dest..." / "LDAP" / "Email Address"
Now enter a few characters, press "Start Search" and select the desired entries from list.
For more details, check \href{http://support.ricoh.com/bb_v1oi/html/oi/rc3/model/mp25/manual/e-mail.htm}{Basic Procedure for Sending Scan Files by E-mail}
\subsubsection*{Make photo copy of business / ID cards}
At the copier console, press "Copier (Classic)" / "1 Sided->Combine 1 Side". Put business / ID card on glass pane at upper left corner. This will copy both sides of cards onto one page.
\subsubsection*{Print confidential documents}
You can use "Locked Print" feature to print a document, then go to the console, press "Printer (Classic)" and enter the passcode to receive it so that no one else will take/view it by mistake.
\href{http://support.ricoh.com/bb_v1oi/html/oi/r-c3/model/mpc300/manual/printer/locked.htm}{How to use the Locked Print function}

\subsection*{Bizhub C364}
We have two color Bizhub C364 muti-function printers: mfp420 and mfp150v queues. Black/white only Bizhub 363 is the counterpart of C364: mfp473 queues. Please contact David Amwake for your copy/scan code.
\subsubsection*{Console access at the printers}
At the logon touch screen, press keyboard icon and enter the code.
"OK" / "Login" to access the system
Press "Access" on keypad to logout when finished
\subsubsection*{How to scan documents in PDF and send to email address?}
\textbf{NOTE:} Use \texttt{Compact PDF} under "File Type" to minimize file size. Otherwise you may lose the scan in email if its size is over 20MB.
\begin{enumerate}
    \item Login if necessary
    \item Press "Fax / Scan" button, select "Addr. Search" tab on touch screen
    \item (If you want to enter email address manually, selecct "Direct Input" tab)
    \item "Search" / type in few letters of your name / "Start Search"
    \item Touch the address or addresses to select or deselect / "OK" (touch once to select, twice to deselect)
    \item If you need to scan multiple sets from the glass or automatic document feeder (ADF), touch "Applications" / "Separate Scan" / "ON"
    \item If necessary, touch "Applications" / "Simplex/Duplex" and/or "Resolution" and/or "File Type" / "OK" to customize
    \item Press "Start" on keypad to begin scanning
    \item Press "Access" on keypad to logout when finished
\end{enumerate}
Addressees should receive the scanned documents in a few minutes. Note network condition and email server policies can affect the delivery. You may need to check with the network and email administrators if you are dealing with large files and non-PDF format. Otherwise use scan to USB instead.
\subsubsection*{How to scan unto USB key?}
\begin{enumerate}
    \item Login if necessary
    \item Plug in USB drive, connector is located on the right side in the back
    \item Press "User Box" button
    \item Touch "Save Document" / "System User Box" tab / "External Memory" / "OK" ("External Memory" will not be present without USB drive plugged in)
    \item If customized document name is preferred, touch "Document Name" / press "C" on keypad to clear default name / type in the new name / "OK"
    \item If necessary, touch "Scan Settings" / "Simplex/Duplex" and/or "Resolution" and/or "File Type" / "OK" to customize
    \item The default is to scan PDF document as single-sided in one continuous process (no pause or changing/feeding another set of papers)
    \item If you need to scan multiple pages from the glass or manual feed ADF, touch "Scan Settings" / "Separate Scan" (toggle) / "OK"
    \item Press "Start" on keypad to begin scanning
    \item Press "Access" on keypad to logout when finished
\end{enumerate}
\subsubsection*{How to print to them?}
Setup printers on your computer: mfpxxx, mfpxxx\_color, mfpxxx\_duplex. The numbers xxx can be 420, 473, or 150v. Refer to \href{https://www.sc.fsu.edu/computing/tech-docs/206-network-printer-setup}{this guide} if needed.
Color printing will be accounted for by users. All printing are logged, but B/W printing are not accounted.
If you need help, please email \href{mailto:sysops@sc.fsu.edu}{sysops@sc.fsu.edu}.
\subsubsection*{Terms}
\begin{description}
    \item[MFP] = Multifunction Printer
    \item[ADF] = Automatic Document Feeder
    \item[USB] = External Memory
    \item[User Box] = 
    \item[Toggle] = 
    \item[Separate Scan] = 
    \item[Legal] = 8.5 x 14
    \item[Ledger] = 11 x 17
\end{description}
\hfill \textit{Written by Xiaoguang Li}

\section{Hosting Policy}
\url{https://www.sc.fsu.edu/computing/tech-docs/287-hosting-policy}

\subsection*{Support for non-DSC HPC systems}
The DSC plays a major rule in the support of FSU's computational infrastructure by providing expert technical support in advanced computing. In addition to the management of FSU's shared multidisciplinary HPC (see \url{http://hpc.fsu.edu}), the DSC has also invested a large sum of funds to develop a computing facility located in the center of the main FSU campus. This facility is currently equipped with 1000 square feet of raised floor, battery backup power, high-speed network connectivity to the campus network backbone and the Florida Lambda Rail, and 80 tons of cooling.

As part of it's service to University, the DSC hosts computational servers owned by other FSU academic units on a first-come, first-served basis until 25\% of the DSC machine room cooling, electrical, or floor capacity is reached. Once the capacity of the room is reached, the DSC will not support additional machines.

\subsection*{Eligibility}
All units within the College of Arts and Sciences are eligible to have machines hosted in the DSC computer room. The DSC's Local Systems Committee will evaluate each request on a case-by-case basis and will agree to host machines provided the capacity allocated for non-DSC systems has not been reached and the requirements listed below are met.

\subsection*{Requirements}
Each of the requirements enumerated below is described in greater detail in the following pages.
\begin{enumerate}
    \item The DSC is not responsible for incidental costs associated with the installation or maintenance of computer systems hosted in the DSC computer room.
    \item Requests to house computer systems in the DSC computer room must include a detailed hardware proposal.
    \item Requests to house computer systems in the DSC computer room must include a proposal describing how the computer systems will be managed.
    \item Computer systems hosted in DSC computer room are required to run the DSC load distribution system.
    \item Requests to house computer systems in the DSC computer room must include a designated faculty liaison from within the unit making the request.
\end{enumerate}
For questions or information regarding availability contact Jim Wilgenbusch at 645-0307 or \href{mailto:sysops@sc.fsu.edu}{sysops@sc.fsu.edu}.

\subsubsection*{1. Incidental costs}
In addition to the substantial costs of providing a cold room with reliable power, there are a number of "incidental" costs associated with hosting computer systems. These costs include server racks, power distribution units (PDUs or power strips), and network switches. The DSC is not responsible for covering incidental costs. At a minimum, the DSC will provide a single outlet for each electrical circuit required to power the computer system and a single 10/100/1000 network uplink in the vicinity of the computer system. The cost of incidentals (e.g., racks, PDUs, and network connectivity) can either be included in the price of a fully integrated system or the DSC will provide these items at cost. In order to avoid half empty racks, the later option may be required. For example, if a group plans to purchase a small number of servers, then the DSC would prefer to use an existing rack, PDU, and network switch. In this case, costs will be assessed according to the portion of the resource being used.

\subsubsection*{2. Hardware proposal}
Requests to house computer systems in the DSC computer room must include a detailed hardware proposal. An agreement to house a computer system in the DSC computer room is limited to the hardware described in the proposal. This requirement is intended to safeguard against someone purchasing hardware that is in some way incompatible with the DSC machine room infrastructure. In addition, the hardware proposal is needed to estimate the power, space, and cooling requirements of the computer system so that we do not exceed the capacity of the DSC computer room. The DSC has a wealth of experience with numerous hardware vendors and can create the hardware proposal on the behalf of unit making the request.

Hardware proposals must include the following information:
\begin{itemize}
    \item Server rack dimensions *
    \item Switch type and size *
    \item PDU type and quantity *
    \item Chassis type per unit
    \item CPU type and quantity per unit
    \item Motherboard type per unit
    \item Ethernet or network adapter type and quantity per unit
    \item Hard drive type and quantity per unit
    \item Memory type and quantity per unit
    \item Type and quantity of all peripheral devises that will be part of the computer system (e.g. disk arrays, CD/DVD drives, etc.)
\end{itemize}
* if not using equipment provided by the DSC

\subsubsection*{3. Management proposal}
Requests to house computer systems in the DSC computer room must include a proposal describing how the computer system will be managed. Two general administrative options are available; 1. the DSC will manage the computer system, and 2. the unit making the request will manage the computer system.

\paragraph{Option 1: DSC manages the computer system}
Option 1 gives a group access to a computational resource without the hassle of having to administer it. The DSC will agree to manage a computer system only if the DSC is given discretion to choose the hardware, the operating system, and the tools used to update and distribute software on the computer system. The DSC systems manager determines who is granted access to privileged commands. Access to privileged commands will be controlled by the "sudo" command; root logins are not permitted. Only members of the DSC systems group will know the root password. The DSC systems manager will determine the update cycle of the core operating system and system software.

\paragraph{Option 2: Non-DSC unit manages the computer system}
If a unit chooses to manage their own computer system, then a qualified systems administrator must be designated to manage the resource. The required management proposal must address in detail how the designated systems administrator will tend the following tasks: account management, disk management, network management, system installation, software management, security, systems management, user support, hardware support, and administration of purchase and warranty information. Systems not managed by the DSC will be relegated to a separate secure network. A firewall will monitor incoming and outgoing network traffic and the DSC systems manager will determine the stringency of the firewall. Access the machine room will be limited to the designated systems administrator. Members of the DSC systems group will have root access to all systems housed in the DSC computer room.

\subsubsection*{4. Cycle sharing}
In order to maximize throughput on DSC and non-DSC computational servers and cluster nodes, the DSC uses a job scheduling and queuing system called Condor (\url{http://www.cs.wisc.edu/condor/}). All of the computer systems housed in the DSC computer room must participate in this system. The Condor system allows fine-grain configuration of policies to support distributively owned computing resources. For example, specific nodes can be configured to "prefer" jobs submitted by specific users. This way, a resource owner will have limited access to all of the DSC computational resources and guaranteed access to his or her machines. This system ensures that cycles are not wasted if the primary resource owner is not using them.

While the Condor system provides some important advantages to managing and distributing computationally intensive jobs, a resource owner is not obligated to use it. For example, a resource owner can run jobs interactively on any of his or her computer systems and his or her jobs will preempt jobs managed by the Condor system. Circumventing the Condor system in this way is to the disadvantage of the resource owner, however. By submitting jobs through the Condor system, a resource owner will not only have guaranteed access to his or her machines, but the Condor system will also provide access to the other machines housed in the DSC machine room.

\subsubsection*{5. Faculty Liaison}
Requests to house computer systems in the DSC computer room must include a designated faculty liaison from within the unit making the request. The liaison will be responsible for conveying information to the users of the computer system housed in the DSC computer room. If the DSC agrees to manage a computer system, the faculty liaison is responsible for approving account requests, requesting changes to a user's priority, and requesting software installations. The authority to make these requests can be transferred to a designate person (e.g., graduate student) at the written request of the faculty liaison.
\hfill \textit{Written by Michael McDonald}

\section{Firefox/Thunderbird Troubleshooting}
\url{https://www.sc.fsu.edu/computing/tech-docs/290-firefox-troubleshooting}

If Firefox will not start it is probably because the lockfiles weren't cleaned up
\begin{verbatim}
# killall -9 firefox
# /panfs/panasas1/system/sysfiles/usrcommon/scripts/cleanFirefox.sh
# /panfs/panasas1/system/sysfiles/usrcommon/scripts/cleanThunderbird.sh
\end{verbatim}
Useful firefox options
\begin{verbatim}
firefox -P
firefox -no-remote
firefox -safe-mode
\end{verbatim}

\section{Stereographic Visualization}
\url{https://www.sc.fsu.edu/computing/tech-docs/306-stereographic-visualization}

The Department of Scientific Computing hosts a high resolution stereographic projection system designed to support multidisciplinary scientific visualization in its seminar room located on the 4th floor of Dirac Science Library. The equipment was acquired in large part thanks to a Major Research Instrumentation grant from the \href{https://www.nsf.gov/}{National Science Foundation}.

The room can seat approximately 100 people.

\subsection*{Equipment}
The SCS seminar room is equipped with a presentation console where a variety of A/V devises can be controlled by using a simple touch panel controller. The following devises/connections are provided:
\begin{itemize}
    \item one windows XP workstation for DVI and stereo displays (full screen resolution is 2480 x 1050)
    \item two connections for laptops (one cat5 cable, plus the campus wireless is available)
    \item one document camera
    \item one DVD/VHS player
    \item limited cable TV
    \item stereo audio output from the workstation, DVD/VHS player, and cable TV
    \item wireless microphone
\end{itemize}
The following is a picture of the touch panel used to access the available input devises.

A view of the four-high resolution projectors used to illuminate the 17.5' x 7.5' screen. The bottom two projectors are used in 2d mode, while all four are used when the system is in 3d mode.

\subsection*{Use}
Coming soon.

\subsection*{Reservations}
Subject to availability, the seminar room may be booked by non-DSC faculty and staff.

\subsection*{Old Documentation}
\begin{itemize}
    \item \href{http://www.sc.fsu.edu/SciVis/VizWall/VizWall_FAQ.php}{FSU VizWall FAQ}
    \item \href{http://www.sc.fsu.edu/SciVis/Stereo_How-To/html/index.html}{Stereo Workstation Usage Guide}
    \item \href{http://www.sc.fsu.edu/SciVis/Amira/html/index.html}{Amira Tutorial for Creating Stereo (3D) Movies}
\end{itemize}

\subsection*{Attachments}
\begin{description}
    \item[\href{https://www.sc.fsu.edu/computing/tech-docs/attachments/306-stereographic-visualization/xpo_user_manual.pdf}{xpo\_user\_manual.pdf}] XPO user manual (660 kB)
    \item[\href{https://www.sc.fsu.edu/computing/tech-docs/attachments/306-stereographic-visualization/StereoWallIntro.pdf}{StereoWallIntro.pdf}] Introduction to the StereoWall (9774 kB)
    \item[\href{https://www.sc.fsu.edu/computing/tech-docs/attachments/306-stereographic-visualization/Control_Interface.jpg}{Control Interface}] Control Interface (85 kB)
\end{description}

\section{Gnuplot}
\url{https://www.sc.fsu.edu/computing/tech-docs/322-gnuplot}

Gnuplot is a portable command-line driven interactive data and function plotting utility for UNIX, IBM OS/2, MS Windows, DOS, Macintosh, VMS, Atari and many other platforms. The software is copyrighted but freely distributed (i.e., you don't have to pay for it). It was originally intended as to allow scientists and students to visualize mathematical functions and data. It does this job pretty well, but has grown to support many non-interactive uses, including web scripting and integration as a plotting engine for third-party applications like Octave. Gnuplot has been supported and under development since 1986. For more information see the \href{http://www.gnuplot.info/}{Gnuplot homepage}.
\hfill \textit{Written by Michael McDonald} \\
\hfill \textit{Category: Supported Applications}

\section{ImageMagick}
\url{https://www.sc.fsu.edu/computing/tech-docs/323-imagemagick}

ImageMagick is a software suite to create, edit, and compose bitmap images. It can read, convert and write images in a variety of formats. For more information, see the \href{https://imagemagick.org/}{ImageMagick website}.
\hfill \textit{Written by Michael McDonald} \\
\hfill \textit{Category: Supported Applications}

\section{LaTeX}
\url{https://www.sc.fsu.edu/computing/tech-docs/324-latex}

LaTeX is a document preparation system for high-quality typesetting.
It is most often used for medium-to-large technical or scientific
documents but it can be used for almost any form of publishing.
\begin{description}
    \item[Home Page:] \url{http://www.latex-project.org/}
    \item[Distribution Specific Information:] \url{http://www.tug.org/tetex/}
    \item[General Tutorials:] \url{http://www.latex-project.org/guides/}
\end{description}

\subsection*{Use}
Open an X11 terminal (if you do not use an X windows enabled terminal Maple will run in the command-line mode).
TeXMaker is the LaTeX frontend.
LaTeX is available from on almost every computer managed by DSC, to execute it type
\begin{verbatim}
latex filename.tex
pdflatex filename.tex
pslatex filename.tex
\end{verbatim}

\subsection*{Missing Style Files}
You may request TSG to install standard packages by emailing \href{mailto:sysops@sc.fsu.edu}{sysops@sc.fsu.edu}, or download and put them in \texttt{\textasciitilde{}/texmf/}. The following commands can be helpful:
\begin{verbatim}
kpsewhich filename.sty
kpsewhich -var-value=TEXMFHOME
\end{verbatim}
If you have \texttt{filename.ins}, \texttt{filename.sty} can be generated by running:
\begin{verbatim}
latex filename.ins
\end{verbatim}
\hfill \textit{Written by Michael McDonald} \\
\hfill \textit{Category: Supported Applications}

\section{MATLAB}
\url{https://www.sc.fsu.edu/computing/tech-docs/325-matlab}

MATLAB, the MATrix LABratory, is a high-level technical computing language and interactive environment for algorithm development, data visualization, data analysis, and numeric computation. Using MATLAB, you can solve technical computing problems faster than with traditional programming languages, such as C, C++, and Fortran. For more see: \href{https://www.mathworks.com/products/matlab.html}{MathWorks MATLAB website} and \href{https://www.mathworks.com/company/newsletters/articles/the-origins-of-matlab.html}{Origins of MATLAB}.

The primary benefits of developing a project with MATLAB include: efficient, concise array manipulations; availability of many already-implemented numerical routines (via the standard distribution, add-on toolboxes, or online software repositories); and the large community of users available for support. One particularly active MATLAB community and software repository is MATLAB Central. For more see: \href{https://www.mathworks.com/matlabcentral/}{MATLAB Central}.

\subsection*{Toolboxes}
Toolboxes are add-ons to the standard MATLAB distribution. They provide additional functionality within a targeted application area. DSC has the following toolboxes available (current as of Spring 2009):
\begin{itemize}
    \item Control System Toolbox
    \item Fixed-Point Toolbox
    \item Image Processing Toolbox
    \item Mapping Toolbox
    \item Neural Network Toolbox
    \item Optimization Toolbox
    \item Partial Differential Equation Toolbox
    \item Signal Processing Toolbox
    \item Statistics Toolbox
    \item System Identification Toolbox
    \item Wavelet Toolbox
\end{itemize}
For more see: \href{https://www.mathworks.com/products.html}{MathWorks Product Listing}.

\subsection*{High-Performance Computing with MATLAB}
Beginning with MATLAB R2007b, new features have been implemented in the official MathWorks MATLAB distribution that allow for parallel and distributed computing. The MATLAB Distributed Computing Engine, used in conjunction with Distributed Computing Toolbox, enables MATLAB and Simulink users to execute applications on a computer cluster. For more see: \href{https://www.mathworks.com/company/newsletters/articles/parallel-matlab-multiple-processors-and-multiple-cores.html}{Cleve's Corner: Parallel MATLAB: Multiple Processors and Multiple Cores}.

MatlabMPI is set of MATLAB scripts that implement a subset of MPI and allow any MATLAB program to be run on a parallel computer. The key innovation of MatlabMPI is that it implements the widely used MPI look and feel on top of standard MATLAB file i/o, resulting in a pure MATLAB implementation that is exceedingly small (\textasciitilde{}300 lines of code). Thus, MatlabMPI will run on any combination of computers that MATLAB supports. For more see: \href{http://www.ll.mit.edu/mission/isr/matlabmpi.html}{MATLAB MPI}.

\subsection*{Running MATLAB}
With the exception of the classroom and hallway computers, MATLAB access is available only on the \href{https://www.sc.fsu.edu/systems/general-purpose-cluster.php}{General Purpose (GP) Cluster through SGE}. Unless initiated through an X11-capable terminal (i.e. Gnome Terminal, xterm, rxvt, etc.), MATLAB will start in a text-only command-line mode.
\textbf{NOTE:} To force MATLAB to run in command-line mode use the following flags: \texttt{matlab -nodesktop -nojvm -nosplash}
\begin{description}
    \item[-nodesktop] Start MATLAB without its desktop.
    \item[-nojvm] Start MATLAB without the Java virtual machine (JVM).
    \item[-nosplash] Start MATLAB but does not display the splash screen.
\end{description}
\subsubsection*{DSC General Access}
MATLAB must be run remotely using SGE
\begin{enumerate}
    \item Open an X11 terminal and type: \texttt{qlogin}
    \item SGE will locate the least-loaded classroom node and initiate a connection.
    \item After authenticating, type: \texttt{matlab}
\end{enumerate}
If \texttt{qlogin} is not available on your machine, you must run the above commands from the PAMD cluster.
\begin{enumerate}
    \item SSH to pamd: \texttt{ssh -X pamd.sc.fsu.edu}
    \item Then follow the above instructions.
\end{enumerate}
\subsubsection*{Classroom and Hallway Computers}
MATLAB is installed locally on each machine.
Open an X11 terminal and type: \texttt{matlab}
\subsubsection*{In Batch Mode}
There are circumstances where a user would want to run MATLAB without its graphical user interface (GUI). For example, a user may want call a MATLAB script from a Makefile, or submit multiple batch jobs to a queuing system.
\paragraph{From a Makefile}
Use the following syntax:
\texttt{cd WORKING\_DIRECTORY \&\& matlab -nosplash -nojvm -r PATH\_TO\_MATLAB\_SCRIPT}
where \texttt{WORKING\_DIRECTORY} is the directory that MATLAB will work in.
\texttt{PATH\_TO\_MATLAB\_SCRIPT} is the path to the MATLAB m-file you wish to run

\subsection*{Learning MATLAB}
There are a plethora of beginning MATLAB tutorials on the internet. A few are listed here:
\begin{itemize}
    \item \href{http://www.math.unm.edu/~wz/hpced/matlab.pdf}{Numerical Computing with MATLAB}
    \item \url{http://www.math.utah.edu/lab/ms/matlab/matlab.html}
    \item \url{http://www.math.mtu.edu/~msgocken/intro/intro.html}
\end{itemize}
One of the most overwhelming parts of learning MATLAB is becoming familiar with all of its functions. A useful resource is \href{https://www.mathworks.com/help/matlab/}{MATLAB Help}.
\hfill \textit{Written by Michael McDonald} \\
\hfill \textit{Category: Supported Applications}

\section{R}
\url{https://www.sc.fsu.edu/computing/tech-docs/326-r}

R is a language and environment for
statistical computing and graphics. It is a \href{http://www.gnu.org/}{GNU} project
which is similar to the S language and environment which
was developed at Bell Laboratories (formerly AT\&T, now Lucent
Technologies) by John Chambers and colleagues. R can be considered as
a different implementation of S. There are some important differences,
but much code written for S runs unaltered under R.

R provides a wide variety of statistical (linear and nonlinear
modelling, classical statistical tests, time-series analysis,
classification, clustering, ...) and graphical techniques, and is
highly extensible. The S language is often the vehicle of choice for
research in statistical methodology, and R provides an Open Source
route to participation in that activity.

One of R's strengths is the ease with which well-designed
publication-quality plots can be produced, including mathematical
symbols and formulae where needed. Great care has been taken over the
defaults for the minor design choices in graphics, but the user
retains full control.

R is available as Free Software under the terms of the \href{http://www.fsf.org/}{Free Software Foundation}'s
GNU General Public License in source code
form. It compiles and runs on a wide variety of UNIX platforms and
similar systems (including FreeBSD and Linux), Windows and MacOS.

For more information see the R \href{https://www.r-project.org/}{website}.

\subsection*{Running R}
R access is available only on the General Purpose (GP) Cluster through SGE.
Unless initiated through an X11-capable terminal (i.e. Gnome Terminal, xterm, rxvt, etc.), R will start in a text-only command-line mode.
\begin{verbatim}
% ssh pamd
% qlogin -l r
% R
\end{verbatim}
\hfill \textit{Written by Michael McDonald} \\
\hfill \textit{Category: Supported Applications}

\section{Mathematica}
\url{https://www.sc.fsu.edu/computing/tech-docs/330-mathematica}

Mathematica is a general purpose computer algebra system, that can be used to solve mathematical problems and produce high-quality technical graphics. Mathematica uses a high-level programming language, which allows for numerical and symbolic solutions and gives users a way to define their own procedures. it also has some built in packages, which may be loaded to do work in group theory, linear algebra, and statistics, as well as in other fields. It can be used interactively or in batch mode.
\begin{description}
    \item[Home Page:] \url{http://www.wolfram.com/products/mathematica/index.html}
\end{description}
\subsection*{General Links}
\begin{itemize}
    \item \url{http://www.wolfram.com/broadcast/screencasts/}
    \item \url{http://reference.wolfram.com/mathematica/guide/Mathematica.html}
    \item \url{http://support.wolfram.com/}
\end{itemize}

\subsection*{Use}
Current floating network license allows Mathematica to be installed on any client belong to the Department. Use \texttt{license.sc.fsu.edu} as License Manager (LM) server in such deployment. Mathematica is currently installed on classroom and hallway computers as well as GP cluster. Other installations may be requested through \href{mailto:sysops@sc.fsu.edu}{sysops@sc.fsu.edu}.

Open an X11 terminal (X-Win32 for Windows users). If you do not use an X windows enabled terminal Mathematica will return error like this:
\texttt{X connection to localhost:... broken (explicit kill or server shutdown).}

From an DSC managed desktop (e.g., hostname is \texttt{dskscs***.sc.fsu.edu}) type:
\begin{verbatim}
qlogin -l mathematica
\end{verbatim}
From all other machines first log in to \texttt{pamd.sc.fsu.edu}
\begin{verbatim}
ssh -Y pamd.sc.fsu.edu
\end{verbatim}
then type:
\begin{verbatim}
qlogin -l mathematica
\end{verbatim}
This command logs you onto the least loaded GP node. At GP prompt, type one of the following commands:
\begin{verbatim}
mathematica
mathematica -cleanStart
mathematica -mesa
mathematica -defaultvisual
unset LD_LIBRARY_PATH; mathematica
\end{verbatim}
To use the non-graphical version of Mathematica type one of these:
\begin{verbatim}
math
math -noprompt < batchfile
math -lmverbose
math -noinit
\end{verbatim}
\hfill \textit{Written by Xiaoguang Li} \\
\hfill \textit{Category: Supported Applications}

\section{GNU Compiler Collection (GCC)}
\url{https://www.sc.fsu.edu/computing/tech-docs/334-gcc}

The GNU Complier Collection (GCC) is an integrated compiler that can compile programs written in C, C++, and Fortran. GCC is both the most general name for the compiler suite, and the name used when the emphasis is on compiling C programs (as the abbreviation formerly stood for "GNU C Compiler"). For more information see the \href{https://gcc.gnu.org/}{GNU GCC website}.
\hfill \textit{Written by Michael McDonald} \\
\hfill \textit{Category: Supported Libraries \& Compilers}

\section{g++}
\url{https://www.sc.fsu.edu/computing/tech-docs/335-g}

The command \texttt{g++} invokes the C++ compiler from the GNU Compiler Collection (GCC). C++ source files conventionally use one of the suffixes \texttt{.C}, \texttt{.cc}, \texttt{.cpp}, \texttt{.c++}, \texttt{.cp}, or \texttt{.cxx}; preprocessed C++ files use the suffix \texttt{.ii}. GCC recognizes files with these names and compiles them as C++ even when invoked as \texttt{gcc}. However, C++ programs often require additional class libraries, and under some circumstances you might want to compile programs from standard input, or otherwise without a suffix that flags them as C++ programs. \texttt{g++} is a program that calls GCC with the default language set to C++, and automatically specifies linking against the C++ library.

When you compile C++ programs, you may specify many of the same command-line options that you use for compiling programs in any language; or command-line options meaningful for C and related languages; or options that are meaningful only for C++ programs. These options may be found in the \href{https://gcc.gnu.org/onlinedocs/}{GCC documentation} or on the \href{https://gcc.gnu.org/}{GCC web site}.
\hfill \textit{Written by Michael McDonald} \\
\hfill \textit{Category: Supported Libraries \& Compilers}

\section{g77}
\url{https://www.sc.fsu.edu/computing/tech-docs/336-g77}

g77 is a free Fortran 77 compiler. For info on G77, now integrated into the GNU Compiler Collection (GCC), see the online \href{https://gcc.gnu.org/onlinedocs/gcc-3.4.6/g77/index.html}{g77 documentation}.
\hfill \textit{Written by Michael McDonald} \\
\hfill \textit{Category: Supported Libraries \& Compilers}

\section{g95}
\url{https://www.sc.fsu.edu/computing/tech-docs/337-g95}

GNU FORTRAN 95.
\hfill \textit{Written by Michael McDonald} \\
\hfill \textit{Category: Supported Libraries \& Compilers}

\section{GSL}
\url{https://www.sc.fsu.edu/computing/tech-docs/341-gsl}

The GNU Scientific Library (GSL) is a numerical library for C and C++ programmers. It is free software under the GNU General Public License. The library provides a wide range of mathematical routines such as random number generators, special functions and least-squares fitting. There are over 1000 functions in total with an extensive test suite. For more information, see the \href{https://www.gnu.org/software/gsl/}{GSL website} or \href{https://www.gnu.org/software/gsl/doc/html/index.html}{reference manual}.
\hfill \textit{Written by Michael McDonald} \\
\hfill \textit{Category: Supported Libraries \& Compilers}

\section{LAPACK}
\url{https://www.sc.fsu.edu/computing/tech-docs/342-lapack}

LAPACK, a Linear Algebra PACKage, is written in Fortran90 and provides routines for solving systems of simultaneous linear equations, least-squares solutions of linear systems of equations, eigenvalue problems, and singular value problems. The associated matrix factorizations (LU, Cholesky, QR, SVD, Schur, generalized Schur) are also provided, as are related computations such as reordering of the Schur factorizations and estimating condition numbers. Dense and banded matrices are handled, but not general sparse matrices. In all areas, similar functionality is provided for real and complex matrices, in both single and double precision.

LAPACK provides a library of routines that can be linked to your executables. To link your executable to the LAPACK library, use the option \texttt{-llapack}. Note that you may also need to link to the BLAS library. See the \href{http://www.netlib.org/lapack/}{LAPACK website} for additional information.
\hfill \textit{Written by Michael McDonald} \\
\hfill \textit{Category: Supported Libraries \& Compilers}

\section{Super LU}
\url{https://www.sc.fsu.edu/computing/tech-docs/343-superlu}

Super LU is a general purpose library for the direct solution of large,
sparse, nonsymmetric systems of linear equations on high performance
machines. The library is written in C and is callable from either C or
Fortran. The library routines will perform an LU decomposition with
partial pivoting and triangular system solves through forward and back
substitution. The LU factorization routines can handle non-square
matrices but the triangular solves are performed only for square
matrices. The matrix columns may be preordered (before factorization)
either through library or user supplied routines. This preordering for
sparsity is completely separate from the factorization. Working
precision iterative refinement subroutines are provided for improved
backward stability. Routines are also provided to equilibrate the
system, estimate the condition number, calculate the relative backward
error, and estimate error bounds for the refined solutions.

For more information see the Super LU \href{https://portal.nersc.gov/project/sparse/superlu/}{website} or its \href{https://portal.nersc.gov/project/sparse/superlu/superlu_ug.pdf}{users' guide}.
\hfill \textit{Written by Michael McDonald} \\
\hfill \textit{Category: Supported Libraries \& Compilers}

\section{GROMACS}
\url{https://www.sc.fsu.edu/computing/tech-docs/344-gromacs}

GROMACS is a versatile package to perform molecular dynamics, i.e. simulate the Newtonian equations of motion for systems with hundreds to millions of particles. It is primarily designed for biochemical molecules like proteins and lipids that have a lot of complicated bonded interactions, but since GROMACS is extremely fast at calculating the nonbonded interactions (that usually dominate simulations) many groups are also using it for research on non-biological systems, e.g. polymers.

GROMACS supports all the usual algorithms you expect from a modern molecular dynamics implementation.

For more information, check the \href{http://www.gromacs.org/}{GROMACS website} or the \href{http://www.gromacs.org/Documentation/Tutorials}{GROMACS tutorials}.
\hfill \textit{Written by Michael McDonald} \\
\hfill \textit{Category: Supported Libraries \& Compilers}

\section{Using SSH}
\url{https://www.sc.fsu.edu/computing/tech-docs/345-using-ssh}

\subsection*{Passwordless authentication}
\textbf{Note:} Passwordless authentication has been disabled on the \texttt{pamd} servers. However, once inside our network you can setup passwordless ssh to other hosts normally.

Below you'll find the "barebones" steps needed to connect to a remote machine without having to enter your password. This is NOT a comprehensive explanation of how ssh or passwordless authentication works, rather it is simply some notes that I've kept to get ssh to work for my purposes. You should also know that setting up passwordless authentication for your account and machine will expose you to at least one security vulnerability. For example, if someone logged into your account on your machine then they can access any other machine that has been configured to accept passwordless authentication. That said, here's how to do it.

There are two versions of ssh and each version has a slightly
different way of authenticating. I'll break things down into multiple
parts to cover all of the ways you could connect to the different
clients. To find out what version off ssh you are using type:
\begin{verbatim}
ssh -V
\end{verbatim}
You'll either get something that has the string OpenSSH in it or something like this, \texttt{SSH Secure Shell 3.1.0 non-commercial version}. I'll refer to the former as OpenSSH and to the later as ssh.com

\subsubsection*{OpenSSH to OpenSSH}
Most SCS systems are configured to use OpenSSH, which will make your life considerably more simple. The most notable exception to this is the IBM SP machines. The IBMs use the ssh.com version of ssh. It is still possible to connect to machines using differenct implementations of SSH, but it adds an extra layer of work that I will talk about later. In this section I'll discuss how to setup passwordless authentication between a host and a server running openSSH.

First, you need to create a public key using the following command:
\begin{verbatim}
ssh-keygen -t dsa
\end{verbatim}
You'll be prompted for a passphrase and you might be prompted to name the identity files. Leave them blank and hit return.
\begin{verbatim}
Generating public/private dsa key pair.
Enter file in which to save the key (~/.ssh/id_dsa):
Enter passphrase (empty for no passphrase):
Enter same passphrase again:
Your identification has been saved in id_dsa.
Your public key has been saved in id_dsa.pub.
The key fingerprint is:
af:02:20:6e:05:96:84:5a:cc:2c:0a:1e:db:bb:1c:8b
\end{verbatim}
In this step, two files were generated; one is the local identity file (\texttt{id\_dsa}) and the other is the public key that will go into a specific file on the remote host. If the remote host mounts your SC home directory then this step is simple. Change directories to thehidden ssh configuration directory:
\begin{verbatim}
cd ~/.ssh
\end{verbatim}
Copy the contents of \texttt{id\_dsa.pub} into the \texttt{authorized\_keys2} file.
\begin{verbatim}
cat id_dsa.pub >> authorized_keys2
\end{verbatim}
Notice that I've used the append redirect (\texttt{>>}) because you can have multiple host identity keys in the same \texttt{authorized\_keys2} file. The important thing here is that each key must occupy its own line. If you don't yet have an \texttt{authorized\_keys2} file then you can do something like this:
\begin{verbatim}
cat id_dsa.pub > authorized_keys2
\end{verbatim}
That's it. You should now be able to logon to a remote host that is mounting your SCS home directory without having to type in your password. If you are attempting to connect to a remote server that doesn't mount the the CSIT user file system then you will need to copy the public key to the sever and paste or redirect the file into the authorized\_keys2 file. For example:
\begin{verbatim}
scp ./id_dsa.pub faroffserver.bingo.org:.ssh/.
ssh remoteHost.bingo.org
remoteHost> cd .ssh
remoteHost> cat id_dsa.pub >> authorized_keys2
\end{verbatim}
note! if the file doesn't exist you'll need to do this:
\begin{verbatim}
remoteHost> cat id_dsa.pub > authorized_keys2
\end{verbatim}

\subsubsection*{OpenSSH to ssh.com}
Now you want to connect to a remote server using the ssh.com implementation of ssh from a host that's using OpenSSH. This would be the case if you wanted to connect to one of SCS's IBM machines from most of the other SCS machines. Start by generating an ssh.com (SECSH) formatted public key from the OpenSSH public key already in your \texttt{.ssh} directory and put the new key in the right place. Here's what you'll do:
\begin{verbatim}
cd ~/.ssh
ssh-keygen -e -f id_dsa.pub > id_dsa_secsh.pub
\end{verbatim}
SSH.com stores it's configuration files in a hidden directory called \texttt{.ssh2}. If you are trying to connect to a remote machine that mounts the SCS user file system then you'll need to create the \texttt{.ssh2} directory if doesn't already exist.
\begin{verbatim}
mkdir ~/.ssh2
cd ~/.ssh2
\end{verbatim}
Next you will need to copy the new file to the \texttt{.ssh2} directory and configure the authorization file so that it knows where to look to find the key.
\begin{verbatim}
cp ~/.ssh/id_dsa_secsh.pub ~/.ssh2/.
echo Key id_dsa_secsh.pub >> authorization
\end{verbatim}
Again, if the authorization file doesn't exist then you'll need to do:
\begin{verbatim}
echo Key id_dsa_secsh.pub > authorization
\end{verbatim}
That's it. If the remote server to which you are trying to connect does not mount the DSC file system then you will copy the key file to the remote machine and then logon to the machine and make the changes described above.

\subsubsection*{SSH.com to OpenSSH}
Much of what is needed in the section is covered more verbosely
above so I'm going to stick to the commands needed to configure a local
host running ssh.com (local) to connect to a remote host running
OpenSSH (remote).
\begin{verbatim}
local> ssh-keygen -t dsa
local> echo IdKey id_dsa_1024_a > identification
local> scp id_dsa_1024_a.pub remote.fsu.edu:.ssh/.
remote> cd .ssh
remote> ssh-keygen -f id_dsa_1024_a.pub -i > newkey.pub
remote> cat newkey.pub >> authorized_keys2
\end{verbatim}
That's it.
\hfill \textit{Written by Michael McDonald}

\section{Printing FAQ}
\url{https://www.sc.fsu.edu/computing/tech-docs/352-printerfaq}

DSC has consolidated all of its printing and copying options under one contract with three Konica Minolta multifunction printers. As part of this contract, the department is now charged 6 cents per color page, 1 cent per grayscale page. For this reason we have implemented a print accounting systems based on Pykota on top of CUPS. If you have any problems with printing please consult the FAQ below. If you do not find your answer here please contact TSG with your job number and a description of the issue.

\begin{description}
    \item[Q:] What is my quota?
    \item[A:] The default quotas were determined by the department to be \$30 per semester (6months) for faculty and \$10 per semester (6 months) for students. Color pages cost 6 cents per page, grayscale 1 cent per page.
    
    \item[Q:] How do I find out the current balance of my quota?
    \item[A:] We have made an interface for you to review the quota and print jobs that can be found from the print quota link on the main page (note you must be logged in to see this).
    
    \item[Q:] Why don't I have a quota set?
    \item[A:] If you have no quota set this means you aren't listed as being part of the department, if this is a mistake please contact \href{mailto:sysops@sc.fsu.edu}{sysops@sc.fsu.edu} and we will correct it.
    
    \item[Q:] Can I still print if I have no quota or I have used it all up?
    \item[A:] Yes you should still be able to print if you add more credit to the account.
    
    \item[Q:] How do I increase my quota if I need it?
    \item[A:] You can contact Michele Locke and arrange for additional money to be placed on your account that will allow you to print additional pages.
    
    \item[Q:] I am printing from my personal laptop, why do none of my jobs get printed?
    \item[A:] When printing from your personal computer it is important to make sure that the username of your own machine matches your SC/FSU username. Each print job needs to correspond to a user for accountability. If printing with command \texttt{lpr}, this can be done with the -U flag. It is also important to put your personal computer on the proper network via VPN or otherwise.
    
    \item[Q:] My print job seems to sit around in the queue for a long time. Why?
    \item[A:] There can be a lot of reasons. The most important of which is that these are multifunction printers, if someone is copying or scanning with them, they will be unavailable for printing. Check and see if the printer is busy. Also only one print job can be sent to the printer at a time, this means that if someone else prints a job to a queue that is set for the same printer, one job will have to wait for the other to complete first. Email \href{mailto:sysops@sc.fsu.edu}{sysops@sc.fsu.edu} if attention is needed.
\end{description}

\section{NetBeans}
\url{https://www.sc.fsu.edu/computing/tech-docs/355-netbeans}

NetBeans is an ide for developing Java. More information can be found at the \href{https://netbeans.apache.org/}{NetBeans website} as well as a \href{https://netbeans.apache.org/kb/docs/java/quickstart.html}{tutorial introduction}.

\subsection*{Use}
Open an X11 terminal (if you do not use an X windows enabled terminal this package will not work).
If NetBeans is installed locally on your machine (i.e., in the VisLab), type:
\begin{verbatim}
netbeans
\end{verbatim}
Otherwise, from an DSC managed desktop (e.g., hostname is \texttt{dsksc***.sc.fsu.edu}) type:
\begin{verbatim}
qlogin -l netbeans
\end{verbatim}
From all other machines first log in to \texttt{pamd.sc.fsu.edu}
\begin{verbatim}
ssh -Y pamd.scs.fsu.edu
\end{verbatim}
Then type:
\begin{verbatim}
qlogin -l netbeans
\end{verbatim}
This command logs you onto the least loaded GP node. Type:
\begin{verbatim}
netbeans
\end{verbatim}

\section{PETSc}
\url{https://www.sc.fsu.edu/computing/tech-docs/356-petsc}

PETSc, pronounced PET-see (the S is silent), is a suite of data structures and routines for the scalable (parallel) solution of scientific applications modeled by partial differential equations. It employs the MPI standard for all message-passing communication. See the \href{https://www.mcs.anl.gov/petsc/}{PETSC web pages} for additional information.

\section{Totalview}
\url{https://www.sc.fsu.edu/computing/tech-docs/357-totalview}

TotalView is a source-level and machine-level debugger with support for Fortran, C, C++, OpenMP, MPI, and threads. The Totalview website can be found \href{https://www.roguewave.com/products-services/totalview}{here} and a general tutorial is available \href{https://www.roguewave.com/sites/rw/files/resource-assets/TotalView_Tutorial.pdf}{here}.

\section{VMD}
\url{https://www.sc.fsu.edu/computing/tech-docs/358-vmd}

Visual Molecular Dynamics (VMD) is a molecular visualization program for displaying, animating, and analyzing large biomolecular systems using 3-D graphics and built-in scripting. For more information, see the \href{http://www.ks.uiuc.edu/Research/vmd/}{VMD home page} or its \href{http://www.ks.uiuc.edu/Research/vmd/current/docs.html}{documentation and tutorials}.

\subsection*{Use}
Open an X11 terminal.
If VMD is installed locally on your machines (i.e., in the VisLab), type:
\begin{verbatim}
vmd
\end{verbatim}
Otherwise, from an DSC managed desktop (e.g., hostname is \texttt{dsksc***.sc.fsu.edu}) type:
\begin{verbatim}
qlogin -l vmd
\end{verbatim}
From all other machines first log in to \texttt{pamd.sc.fsu.edu}
\begin{verbatim}
ssh -Y pamd.sc.fsu.edu
\end{verbatim}
then type:
\begin{verbatim}
qlogin -l vmd
\end{verbatim}
This command logs you onto the least loaded GP node. Then type:
\begin{verbatim}
vmd
\end{verbatim}

\section{Portland Group Compliers}
\url{https://www.sc.fsu.edu/computing/tech-docs/359-portland-group-compliers}

The Portland Group, Inc. (PGI) provides high-performance parallelizing/optimizing Fortran, C and C++ compilers for 32-bit x86 and 64-bit x64 processor-based Linux and Windows workstations, servers and clusters. For more information, see the \href{https://www.pgroup.com/}{PGI home page}.
\textbf{Note:} The PGI compilers are not available at DSC at this time, but are available at the FSU HPC.

\section{Intel Compliers}
\url{https://www.sc.fsu.edu/computing/tech-docs/360-intel-compliers}

The Intel compilers provide improved performance over the GNU C/C++ compilers. This compiler product automatically optimizes and parallelizes software to take best advantage of multi-core Intel processors. For more information about the Intel C/C++ and Fortran compilers, look \href{https://software.intel.com/en-us/fortran-compilers}{here}.

\section{MPI}
\url{https://www.sc.fsu.edu/computing/tech-docs/361-mpi}

Message Passing Interface (MPI) is a specification for an API that allows many computers to communicate with one another. The MPI standard includes point-to-point message-passing, collective communications, group and communicator concepts, process topologies, environmental management, process creation and management, one-sided communications, extended collective operations, external interfaces, I/O, some miscellaneous topics, and a profiling interface. Language bindings for C, C++ and Fortran are defined. OpenMPI and MPICH2 are two popular free MPI implementations.

\textbf{Note:} At DSC, we currently support OpenMPI and MPICH2 on owner-based clusters only.

\subsection*{OpenMPI}
The Open MPI Project is an open source MPI-2 implementation that is developed and maintained by a consortium of academic, research, and industry partners. Open MPI is a project combining technologies and resources from several other projects (LAM/MPI, FT-MPI, LA-MPI, and PACX-MPI) in order to build the best MPI library available. OpenMPI programs must be compiled and linked with the appropriate MPI compiler. The user creates a special environment for jobs to run in.

\subsection*{MPICH2}
MPICH2 is a freely available, portable implementation of MPI, the Standard for message-passing libraries. It implements both MPI-1 and MPI-2. MPICH2 is supposed to be a compatible replacement for any MPICH version 1 implementation. MPICH2 requires users to build and link against the set of provided routines then run in a special environment. See the \href{https://www.mpich.org/documentation/guides/}{MPICH2 documentation} for additional specifications.

\section{Short Technical Courses}
\url{https://www.sc.fsu.edu/computing/tech-docs/365-short-technical-courses}

The DSC Technical Workshop Series consists of one hour labs designed to
give students and faculty hands on experience with a variety of
applications used in computational science. The labs are led by SCS
graduate students and are held in the SCS computer classroom (DSL 152)
or Seminar room (DSL 499).
\begin{itemize}
    \item Overview of computing resources in the DSC
    \item Introduction to FSU's HPC facility
    \item Introduction to Unix
    \item Introduction to Digital Video and Video Encoding
    \item Introduction to Software Versioning, SVN
    \item Introduction to shell programming
\end{itemize}

\section{Owner-Based FAQs}
\url{https://www.sc.fsu.edu/computing/tech-docs/383-owner-based-faqs}

\subsection*{How do I get an account on a cluster?}
\begin{itemize}
    \item If you already have an DSC account, access can be requested through \href{https://sc.fsu.edu/members/profile}{SCUM}.
    \item These requests will be sent to the cluster owner for approval.
    \item If you do not already have an DSC account, you must first request a DSC sponsored account.
    \item To request a DSC sponsored account, go to \href{https://sc.fsu.edu/members/profile}{SCUM}, click on "Request a DSC Account", select "Sponsored Account", then select the DSC faculty member that will act as your account sponsor.
    \item Once the account sponsor approves your request, you can request cluster access.
\end{itemize}

\subsection*{How do I log into the cluster?}
\begin{itemize}
    \item Cluster resources (compute nodes, etc.) can only be accessed through the cluster headnode.
    \item Inside the DSC network, cluster headnodes can be accessed using \texttt{ssh}.
    \item For example, to log into a cluster called \texttt{phoenix}, type: \texttt{ssh phoenix}
    \item If you are outside the DSC network, you must first log into PAMD to get past our firewall:
    \begin{verbatim}
ssh pamd.sc.fsu.edu
ssh phoenix
    \end{verbatim}
\end{itemize}

\subsection*{How do I compile programs?}
\begin{itemize}
    \item Use the cluster headnode to compile programs intended to run on that cluster.
    \item On the headnode, your default path will be set to allow access to all compilers available on that system.
    \item For certain proprietary compilers, licensing issues will limit the number of simultaneous compilations allowed.
    \item If a compilation job fails because there are no available licenses, be patient and wait for other users' compilation processes to complete, then attempt your compilation again.
\end{itemize}

\subsection*{How do I run programs?}
\begin{itemize}
    \item The headnode must be available to provide essential services to other cluster nodes, so it is imperitive that resource-intensive programs are never run directly on the headnode.
    \item Instead, all programs should be submitted via the cluster's queuing system.
    \item Although \texttt{ssh} may be allowed to compute nodes, please refrain from using this ability to run jobs, because this prevents the scheduler from optimally managing the cluster's resources.
\end{itemize}

\subsection*{How do I log into a specific node?}
In the event that it is necessary to direcly access a specific cluster node interactively, please use the cluster's queuing system.
\begin{verbatim}
qlogin -q [queue-name]
\end{verbatim}
you will be given a login on a node of that type.
If you need to log into a specific node you can use the form
\begin{verbatim}
qlogin -q [queue-name]@[machine-name]
\end{verbatim}

\subsection*{What are the differences between the MPI implementations?}
\begin{itemize}
    \item In general, most MPI implementations provide the same functionality, but in slightly different ways.
    \item OpenMPI provides many utilities that make it more portable and dynamic than the others, making it the suggested platform for new software development.
    \item Occasionally programs require features that are specific to another MPI implementation.
    \item Other MPI implementations are available on certain clusters, but often are often less scheduler-friendly, and have been known to leave processes running if a job dies unexpectedly.
\end{itemize}

\section{Macintosh System Configurations}
\url{https://www.sc.fsu.edu/computing/tech-docs/433-macintosh-setup}

Topics cover the basic system configurations for SC Macintosh clients running Leopard (OS X 10.5.x). Configurations for Tiger may vary somewhat. They allow users without local accounts to access the computer via LDAP. User files can also be saved on the network in a transparent manner. This is ideal for public Macintosh systems. Individual Macs can be configured as such with a slight twist: Mobile accounts. It provides the consistency and convenience of tapping into department resources and yet allows the freedom to logon when off campus.

\subsection*{Configure the System to Use LDAP Authentication}
It may be necessary to reboot when LDAP settings have changed.
\begin{enumerate}
    \item Start /Applications/Utilities/Directory Utility or /System/Library/CoreServices/Directory, unlock and show advanced settings if needed. You need administrator access to make any changes.
    \item Under "Services" tab, edit LDAPv3
    \item Click on "New..." button
    \item Server Name or IP Address: \texttt{ldap-vm0.sc.fsu.edu}. Make sure "Encrypt using SSL" is unchecked, otherwise authentication will fail. As of this writing, this is a bug in Leopard. We can correct this setting later.
    \item Continue and choose template: RFC 2307 (Unix) and enter searchbase: \texttt{ou=people,dc=sc,dc=fsu,dc=edu}
    \item Save is as "LDAP" under "Configuration Name", click on "OK" to finish
    \item Earlier than Leopard, secure LDAP connection is not mandated. Modify the mandate by editing \texttt{/etc/openldap/ldap.conf}
    \texttt{TLS\_REQCERT demand ==> TLS\_REQCERT allow}
\end{enumerate}

\subsection*{Configure the System to Use LDAP SSL to Authenticate}
Assume LDAP authentication is configured already and you need administrator privilege
\begin{enumerate}
    \item Directory Utility
    \item Services
    \item LDAPv3
    \item Encrypt using SSL
    \item Append the following to file \texttt{/usr/share/curl/curl-ca-bundle.crt}
    \begin{verbatim}
Go Daddy Class 2 CA
===================
-----BEGIN CERTIFICATE-----
... (certificate data) ...
-----END CERTIFICATE-----
    \end{verbatim}
    \item Append the following line
    \texttt{TLS\_CACERT /usr/share/curl/curl-ca-bundle.crt}
    to file \texttt{/etc/openldap/ldap.conf} and make sure to restore allow to demand (default for Leopard)
    \texttt{TLS\_REQCERT allow ==> TLS\_REQCERT demand}
\end{enumerate}

\subsection*{Automount Configuration for Remote Homedirectory}
Start "Directory Utility" as before, select "Mounts" tab
\begin{description}
    \item[Remote NFS URL:] nfs://pan-nfs.sc.fsu.edu/
    \item[Mount location:] /panfs/panasas1
\end{description}
\textbf{NOTE:} The trailing slash in the URL is critical. The absence of trailing slash in the mount location is also important.

\subsection*{Configure System to Use CUPS Printers}
If users prefer more options and better control with printers, refer to \href{https://www.sc.fsu.edu/computing/tech-docs/206-network-printer-setup}{network printer setup} and install them with native drivers. Using CUPS printers, the administrative overhead is minimal.
\begin{enumerate}
    \item Login as administrator and create or edit \texttt{/etc/cups/client.conf}
    \texttt{ServerName cups.sc.fsu.edu}
    \item Now users should have direct access to all CUPS printers (reboot if necessasry). All jobs will be printed as black and white by default unlike in most other cases where color is the default.
\end{enumerate}

\subsection*{How to print in color via CUPS printers?}
Select "Color Matching" under Options and then "In Printer"
\hfill \textit{Written by Xiaoguang Li}

\section{Macintosh Frontpage}
\url{https://www.sc.fsu.edu/computing/tech-docs/446-macintosh-frontpage}

This is the first page to consult for general questions about managed Macintosh systems.

\subsection*{Configure Macintosh OS X (Leopard)}
TSG managed Macintosh systems are typically configured as follows. They apply to home/mobile users in most cases. If you need additional items to be addressed, please contact \href{mailto:sysops@sc.fsu.edu}{sysops@sc.fsu.edu}.
\begin{itemize}
    \item Install Leopard from scratch or via upgrade using Install DVD
    \item Complete current updates
    \item Setup LDAP (LDAPS to be done saparately)
    \item Configure MFP printers
    \item Office 2008 for Mac and updates
    \item \href{http://www.mozilla.com/en-US/firefox/}{Firefox} and \href{http://www.mozilla.com/en-US/thunderbird/}{Thunderbird} Setup
    \item \href{http://www.tuxera.com/mac/ntfs-3g_faq.html}{ntfs-3g}
    \item \href{http://www.macupdate.com/info.php/id/26105/svnx}{svnX}
    \item \href{http://www.tug.org/mactex/}{MacTeX} (backend) and \href{http://pages.uoregon.edu/koch/texshop/}{TeXShop} (frontend)
    \item Developer Tools and Xcode from Install DVD
    \item Others
\end{itemize}

\subsection*{Windows for Mac Users}
\subsubsection*{Need to run virtual machines}
\href{http://www.virtualbox.org/}{VirtualBox}
\subsubsection*{Need to access external NTFS partitions or hard drives}
Install \href{http://www.tuxera.com/mac/ntfs-3g_faq.html}{ntfs-3g}
\subsubsection*{Need to run Windows OS on Intel Macintosh}
Install Boot Camp and latest updates
For PowerPC users, VMware Fusion, Parallels, and other commercial packages are available
Follow \href{https://www.sc.fsu.edu/computing/tech-docs/447-windows-frontpage}{Windows Frontpage} to setup Windows XP

\subsection*{Want to remotely control Macintosh}
Install and configure \href{http://www.redstonesoftware.com/products/vine_server}{Vine Server} on OS X
Apple Remote Desktop and other commercial solutions are available
Users can access/control OS X from either Windows and Linux computers via VNC clients
\textbf{NOTE:} this approach may not be secure enough over the public networks. To overcome that, use either ssh tunnels or other commercial grade VNC servers for Mac.

\subsection*{How to change the user short name?}
This may be necessary for easy access to our MFP printers if you don't have LDAP authentication configured. \href{http://docs.info.apple.com/article.html?artnum=106915}{Here is how}.
\hfill \textit{Written by Xiaoguang Li}

\section{Google Applications}
\url{https://www.sc.fsu.edu/computing/tech-docs/527-google-applications97}

The department is using Google Calendar for scheduling both individuals and resources. If you already have Gmail account, you may have found other Google applications - Docs and Sites - useful. Although we are not using FSU calendar system as a department, you may have need to collaborate with other people who do. If you have questions about using FSU calendar, please refer to \href{https://www.sc.fsu.edu/computing/tech-docs/243-fsu-calendar}{this link} or email \href{mailto:sysops@sc.fsu.edu}{sysops@sc.fsu.edu}.

\subsection*{Google Calendar}
Login at \url{http://calendar.sc.fsu.edu/}
You need a Google calendar account for this. Please contact \href{mailto:sysops@sc.fsu.edu}{sysops@sc.fsu.edu} for help.
If you prefer video, view \href{https://www.youtube.com/watch?v=E-k-3G0sZ14}{this clip on Youtube}.
\begin{itemize}
    \item Click on "Settings" (upper right hand menu) if you want to customize settings. Also available under "My calendars" or "Other calendars"
    \item To customize four day weather info:
        \begin{itemize}
            \item Under General page
            \item Location: Tallahassee, FL
            \item Show weather based on my location: F
        \end{itemize}
    \item To add/view/share calendars with others:
        \begin{itemize}
            \item Under Calendars page
            \item Click "Create new calendar" button to add other calendars, for example, personal and project related
            \item Click on "Import calendar" link to view others calendar
            \item Click on "Shared: Edit settings" or your calendars link to share them
            \item You can only delete your own calendars, but may unsubscribe others calendars
        \end{itemize}
    \item In Calendar view (vs Settings view), you can add/remove schedules and event:
        \begin{itemize}
            \item Click and drag in the proper slot to create a schedule (default 30 minute increments). "edit event details" if needed: When and/or Repeats, Where, Calendar, Description, Reminder, Privacy.
            \item To delete an event, click on the event link to open the details section and "Delete" button is at the top.
            \item Typical calendar views: by day, week and month.
        \end{itemize}
    \item To schedule a group meeting:
        \begin{itemize}
            \item Create/add a schedule as above
            \item Open details section
            \item Click on "Add Guests"
            \item "Choose from contacts" link will start a popup window
            \item Select "All contacts.." from the drop-down menu or simply type in the calendar ID, for example, \texttt{calendarid@group.calendar.google.com}
            \item Click on the names to toggle the selection on and off
            \item Click on "Save" button when completed
        \end{itemize}
    \item To customize the colors of each calendar:
        \begin{itemize}
            \item Expand "My calendars" or "Other calendars" if needed
            \item Click on the triangle to the right of the calendar in question
            \item Select the desired color from the drop-down menu
        \end{itemize}
    \item Now you can also use Google calendar to manage tasks:
        \begin{itemize}
            \item Click on "Tasks" menu on the left to toggle tasks list on the right
            \item Type the name for your task to create one...
            \item Tasks can also be categorized using different lists...
        \end{itemize}
\end{itemize}

\subsection*{Google Documents}
When multiplatform users are involved and security is not paramount, Google Docs can be a very handy option.
Click on "Documents" link on the upper left menu from Google calendar or via \url{http://docs.google.com/}.

\subsection*{Google Sites}
Refer to \href{https://www.sc.fsu.edu/computing/tech-docs/528-personal-website-hosting}{Personal Website Hosting}.
Click on "Sites" link on the upper left menu or via \url{http://sites.google.com/} or \url{http://pages.google.com/}.
\hfill \textit{Written by Xiaoguang Li}

\section{Personal Website Hosting}
\url{https://www.sc.fsu.edu/computing/tech-docs/528-personal-website-hosting}

There are numerous ways to host your personal website. Here are a few recommended methods:

\subsection*{DSC's \texttt{public\_html} Solution}
\url{https://people.sc.fsu.edu/~FSUID} or \url{https://sc.fsu.edu/~FSUID}
\begin{enumerate}
    \item Change permissions of your home directory
    \begin{verbatim}chmod 0711 ~\end{verbatim}
    \item Create \texttt{public\_html} folder in your home directory
    \begin{verbatim}mkdir ~/public_html\end{verbatim}
    \item Change permissions of your public\_html directory
    \begin{verbatim}chmod 755 ~/public_html\end{verbatim}
    \item Add content to your \texttt{public\_html} (space counts towards your quota)
\end{enumerate}
\textbf{IMPORTANT:} This will likely allow other users in the file system to read files in your home directory. Please manage the permissions accordingly.

\subsection*{Google portals and other web presence hosts}
\begin{itemize}
    \item Google Site (formerly Google Pages)
    \item etc...
\end{itemize}

\subsection*{FSU's myWebDAV Solution}
\href{https://its.fsu.edu/service-catalog/wi-fi-internet-access/mywebdav}{FSU's myWebDAV Service}. Available to all Faculty and Students.
\url{http://myweb.fsu.edu/FSUID}
If your website is not created yet, please submit a support ticket here, \url{help.fsu.edu}.
Use WebDAV clients to manage your website: e.g., My Network Places (under Windows) or Connect to Server (on the Mac).
\hfill \textit{Written by Michael McDonald} \\
\hfill \textit{Category: Web Design}

\section{Windows X11}
\url{https://www.sc.fsu.edu/computing/tech-docs/532-windows-x11}

To access Linux X11 resources - Matlab, Mathematica, Tecplot, Avizo, and COMSOL - from Windows computers, users need to have \href{http://www.sc.fsu.edu/software/win-ssh-client.php}{SSH Secure Shell Client} and X-Win32 properly configured. Users will also need \href{https://www.sc.fsu.edu/computing/tech-docs/204-vpn-access}{VPN client} if connecting from off campus.

\subsection*{How to turn on X11 tunneling from SSH Secure Shell Client?}
Start "SSH Secure Shell Client", go to "Profiles/Edit Profiles".
Select the profiles in question and/or Quick Connect, under "Tunneling" tab, check "Tunnel X11 connections".
If you start X-Win32 before "SSH Secure Shell Client", all X11 applications will tunnel through X-Win32. Make sure you allow that through the firewalls.

\subsection*{How to enable X-Win32 license (for all users of the computer)?}
Ask your system administrator or email \href{mailto:sysops@sc.fsu.edu}{sysops@sc.fsu.edu} for help.
After entering the correct license as username, copy file
\texttt{C:\textbackslash{}Documents and Settings\textbackslash{}username\textbackslash{}Application Data\textbackslash{}StarNet\textbackslash{}X-Win32\textbackslash{}License.config}
to folder
\texttt{C:\textbackslash{}Program Files\textbackslash{}StarNet\textbackslash{}X-Win32 9.4}.
That will enable all users of the computer to share that license.
\hfill \textit{Written by Xiaoguang Li}

\section{Seminar Room Equipment}
\url{https://www.sc.fsu.edu/computing/tech-docs/620-seminar-room-equipment}

\begin{description}
    \item[Problem:] The screen is flickering when I connect my external device.
    \item[Solution:] Open the front door to the left cabinet/rack and power cycle the top two Cyviz boxes (switch both off for 30 seconds and then turn back on).
    
    \item[Problem:] The screen is missing some colors (red/green/blue) when I connect my external device.
    \item[Solution:] There is most likely faulty wiring between your device and the Extron matrix switcher. Please try the other input cable as a temporary solution and submit a help request to \href{mailto:sysops@sc.fsu.edu}{sysops@sc.fsu.edu} for repair.
\end{description}
\textbf{Note:} If any of the above solutions do not work then please seek further assistance from the Technical Support Group, \href{mailto:sysops@sc.fsu.edu}{sysops@sc.fsu.edu}.
\hfill \textit{Written by Michael McDonald}

\section{Sun Analyzer}
\url{https://www.sc.fsu.edu/computing/tech-docs/655-sun-analyzer}

\href{https://www.sc.fsu.edu/computing/tech-docs/attachments/655-sun-analyzer/analyzer.pdf}{Sun Analyzer Report}
\hfill \textit{Written by Xiaoguang Li}

\section{Video Conferencing}
\url{https://www.sc.fsu.edu/computing/tech-docs/661-video-conferencing}

The attached slides were presented by Michael McDonald on November 4, 2010 at the Department of Scientific Computing. The presentation outlined the various video conferencing tools available to faculty and students. Some of the tools/clients that were covered in the presentation included:
\begin{itemize}
    \item Skype
    \item Google Hangouts
    \item iChat
    \item EVO
    \item Access Grid
\end{itemize}

\subsection*{Attachments}
\begin{description}
    \item[\href{https://www.sc.fsu.edu/computing/tech-docs/attachments/661-video-conferencing/Video_Conferencing.pdf}{Video Conferencing Presentation}] Created by Michael McDonald (2207 kB)
\end{description}
\hfill \textit{Written by Michael McDonald} \\
\hfill \textit{Category: Video Conferencing}

\section{RHEL6}
\url{https://www.sc.fsu.edu/computing/tech-docs/685-rhel6}

More reliable. More open. More comprehensive. More fun. More stable. More computing.
With Red Hat Enterprise Linux, you can do more, not tomorrow, but today.
\href{https://www.redhat.com/en/technologies/linux-platforms/enterprise-linux}{Click here to read more.}

\subsection*{Upgrade FAQs}
\subsubsection*{Is Firefox crashing or you can't read your FSU email using Firefox?}
\begin{enumerate}
    \item Backup Bookmarks - (Bookmarks->Organize Bookmarks->Import and Backup->Backup)
    \item close firefox
    \item \texttt{move \textasciitilde{}/.mozilla \textasciitilde{}/.mozilla.backup}
\end{enumerate}

\subsubsection*{Problems with Acroread Firefox plugin?}
\begin{enumerate}
    \item Backup Bookmarks - (Bookmarks->Organize Bookmarks->Import and Backup->Backup)
    \item close firefox
    \item \texttt{move \textasciitilde{}/.mozilla \textasciitilde{}/.mozilla.backup}
    \item if the above two steps fail, then click Edit->Preferences->Applications->pdf and change the document viewer
\end{enumerate}
\hfill \textit{Written by Michael McDonald}

\section{Usage of the HTML pre Tag}
\url{https://www.sc.fsu.edu/computing/tech-docs/688-usage-of-the-html-pre-tag}

Proper usage of the HTML \texttt{<pre>} tag is the following:
\begin{verbatim}
<pre>
code line 1
code line 2
code line 3
</pre>
Text in a pre element
is displayed in a fixed-width
font, and it preserves
both      spaces and
line breaks
\end{verbatim}
\hfill \textit{Written by Michael McDonald}

\section{FSU Virtual Lab (myFSUVLab)}
\url{https://www.sc.fsu.edu/computing/tech-docs/736-myfsuvlab}

Did you know you can access university software without ever stepping foot on campus? Florida State University's virtual computer lab—myFSUVLab—makes it easy to use university software … anytime, anywhere. Use myFSUVLab for quick access to specialty software, such as SPSS or MATLAB, at home or on the go from any computer, tablet or mobile device. And save yourself a trip to campus.
\begin{center}
    \href{https://its.fsu.edu/service-catalog/end-point-computing/myfsuvlab}{Click Here to Learn More About myFSUVLab} \\
    \href{https://myfsuvlab.its.fsu.edu/vpn/index.html}{Click Here to Launch myFSUVLab}
\end{center}
\hfill \textit{Written by Xiaoguang Li}

\section{Network Time Protocol (NTP)}
\url{https://www.sc.fsu.edu/computing/tech-docs/737-ntp}

Due to security issues access to external NTP resources on the DSC network is prohibited. You may use our internal NTP servers for synchronizing your clocks.
\begin{center}
    \texttt{ntp.sc.fsu.edu}
\end{center}
\hfill \textit{Written by Michael McDonald}

\section{Intel Parallel Studio}
\url{https://www.sc.fsu.edu/computing/tech-docs/739-intel-parallel-studio}

Intel's Parallel Studio provides an comprehensive development environment and is available on classroom and hallway nodes.

To get started using Intel(R) VTune(TM) Amplifier XE 2013 Update 2:
\begin{itemize}
    \item To set your environment variables: \texttt{source /opt/intel-studio/vtune\_amplifier\_xe\_2013/amplxe-vars.sh}
    \item To start the graphical user interface: \texttt{amplxe-gui}
    \item To use the command-line interface: \texttt{amplxe-cl}
    \item For more getting started resources: \url{file:/opt/intel-studio/vtune_amplifier_xe_2013/documentation/en/welcomepage/get_started.html}.
\end{itemize}

To get started using Intel(R) Inspector XE 2013 Update 2:
\begin{itemize}
    \item To set your environment variables: \texttt{source /opt/intel-studio/inspector\_xe\_2013/inspxe-vars.sh}
    \item To start the graphical user interface: \texttt{inspxe-gui}
    \item To use the command-line interface: \texttt{inspxe-cl}
    \item For more getting started resources: \url{file:/opt/intel-studio/inspector_xe_2013/documentation/en/welcomepage/get_started.html}.
\end{itemize}

To get started using Intel(R) Advisor XE 2013 Update 1:
\begin{itemize}
    \item To set your environment variables: \texttt{source /opt/intel-studio/advisor\_xe\_2013/advixe-vars.sh}
    \item To start the graphical user interface: \texttt{advixe-gui}
    \item To use the command-line interface: \texttt{advixe-cl}
    \item For more getting started resources: \url{file://opt/intel-studio/advisor_xe_2013/documentation/en/welcomepage/get_started.html}.
\end{itemize}

To get started using Intel(R) Composer XE 2013 Update 1 for Linux*:
\begin{itemize}
    \item Set the environment variables for a terminal window using one of the following (replace "intel64" with "ia32" if you are using a 32-bit platform).
    \item For csh/tcsh: \texttt{source /opt/intel-studio/bin/compilervars.csh intel64}
    \item For bash: \texttt{source /opt/intel-studio/bin/compilervars.sh intel64}
    \item To invoke the installed compilers:
    \begin{itemize}
        \item For C++: \texttt{icpc}
        \item For C: \texttt{icc}
        \item For Fortran: \texttt{ifort}
    \end{itemize}
    \item To get help, append the \texttt{-help} option or precede with the \texttt{man} command.
    \item For more getting started resources:
    \url{file:/opt/intel-studio/composer_xe_2013/Documentation/en_US/get_started_lc.htm}
    \url{file:/opt/intel-studio/composer_xe_2013/Documentation/en_US/get_started_lf.htm}
\end{itemize}
To view movies and additional training, visit \url{http://www.intel.com/software/products}.

\section{FSU Calendar Service}
\url{https://www.sc.fsu.edu/computing/tech-docs/243-fsu-calendar}

If you have a question about FSU calendar in general, please email \href{mailto:help@acns.fsu.edu}{help@acns.fsu.edu}, \href{mailto:sysops@sc.fsu.edu}{sysops@sc.fsu.edu}, or visit \href{http://its.fsu.edu/}{Information Technology Services}.

\subsection*{How to allow someone else to manage your calendar?}
\begin{enumerate}
    \item Login at \url{http://webmail.fsu.edu} and click on Calendar (2nd tab on top menu)
    \item Click on drop-down list next to "Current Calendar" at upper-right corner
    \item Select \texttt{Calendar Actions [Manage Calendars...]}
    \item Click "Edit" an then "Share this calendar with specific users"
    \item Enter username and click on Add button to include this user for calendar access/management
    \item Check any or all access permissions to grant: Accessibility/Read/Invote/Modify/Delete
\end{enumerate}

\subsection*{How to view another FSU calendar user's schedule?}
\begin{enumerate}
    \item Login at \url{http://webmail.fsu.edu} and click on Calendar (2nd tab on top menu)
    \item Click on drop-down list next to "Current Calendar" at upper-right corner
    \item Select \texttt{Calendar Actions [Manage Calendars...]}
    \item Click "Subscribe..." button
    \item Search and select the calendars to view...
\end{enumerate}

\subsection*{How to ...}
\hfill \textit{Written by Xiaoguang Li}

\section{Overview to Computing Resources in the Dept. of Scientific Computing}
\url{https://www.sc.fsu.edu/computing/tech-docs/366-overview-to-computing-resources-in-the-deptartment-of-scientific-computing}

This event provides DSC faculty, students, and staff with a high level overview of the computing resources available in the department. Members of the technical support group will be on hand to answer more specific questions that you might have about the department's computing resources. This event is geared toward users who have wondered where they can run lots of long serial (single processor) batch jobs, run large parallel jobs, compile software, run matlab interactively, or visualize data sets in 3d. Slides and videos from previous presentations are given below.
\begin{itemize}
    \item \href{https://www.sc.fsu.edu/computing/tech-docs/attachments/366-overview-to-computing-resources-in-the-deptartment-of-scientific-computing/computing_overview.pdf}{Slides}
    \item \href{http://www.youtube.com/watch?v=kUizd2owLh8}{Video}
\end{itemize}


\section{Simulation Gallery}
\url{https://www.sc.fsu.edu/research/simulation-gallery}

\href{https://www.youtube.com/channel/UC5fC52_j8y-77hTANNlC-Xw}{Click here} for more videos from our YouTube channel.

\section{Facilities, Computational Infrastructure}
\url{https://www.sc.fsu.edu/facilities}

Feel free to copy relevant portions of this text into your research proposals.

The Department of Scientific Computing (DSC) plays a major role in the support of FSU's cyberinfrastructure by providing facilities and technical expertise in the support of scientific computing. The DSC manages a dedicated computing facility located on the main FSU campus in Dirac Science Library. The DSC facility provides a highly flexible computing environment designed to support specialized and experimental hardware and software systems.

\subsection{DSC Computing Facility}
The DSC facility supports an assortment of computer architectures, interconnects, and operating systems. Systems hosted in the DSC facility are owned by DSC and are dedicated to a wide range of research problems including; machine learning, neuroscience, molecular biophysics, evolutionary biology, network modeling, and Monte-Carlo algorithm development. The DSC facility is equipped with two 40-ton HVAC cooling units, 1000 ft$^2$ of raised floor, an extensive power distribution system, UPS battery backup systems, and a 550 KVA diesel-powered backup generator, which provides backup power to all of the hardware and HVACs in this server room. The DSC network is built on a 10 Gbps backbone, providing connectivity to a switching infrastructure and to key servers and storage. The DSC network connects via 100 Gbps to the FSU campus backbone, which in turn connects to the Florida/National LamdaRail.

\section{FSU RCC (formerly the Shared-HPC)}
\url{https://www.sc.fsu.edu/facilities}

Please click here to access the latest RCC facilities statement (\url{https://its.fsu.edu/help/it-support/researchers#grants}).

\section{Scientific Visualization}
\url{https://www.sc.fsu.edu/facilities}

The DSC supports a general access laboratory for scientific visualization and computational intelligence. The laboratory is located in the center of the main FSU campus on the fourth floor of the Dirac Science Library (DSL). The Visualization Laboratory hosts several high-end visualization workstations each equipped with GPU video cards that are compatible with the CUDA SDK. All workstations have access to a multi-terabyte shared high-performance storage.

\par The DSC's visualization resources also include a high-resolution laser projection system to support multidisciplinary scientific visualization. The system is located in our main seminar room adjacent to the Visualization Lab. A cutting edge 4K Enhancement Technology rear-mounted projector illuminates an 18' x 8' screen. The system supports numerous input devices via a simple to use touch panel screen.

\section{General DSC Infrastructure}
\url{https://www.sc.fsu.edu/facilities}

The DSC provides office space to DSC faculty, postdocs, graduate students, and other DSC associated support personnel. Designated visitor offices are also available. All offices are equipped with a desktop computer and network connections; wireless is available throughout campus. The department of Scientific Computing supports a cutting edge classroom facility on the campus of FSU to support scientific programming curriculum. The classroom was funded in part by a Student Technology fee award for instructional technology enhancements. The room is equipped with 19 Intel-based workstations running LINUX or Windows and is used primarily for classes taught by DSC faculty. In addition to the computer classroom, a large seminar room is located on the fourth floor with a capacity for 80 people and is equipped with a 4K Enhanced rear-mounted laser projection system. Also, two conference rooms equipped with large high-definition displays can facilitate smaller groups.

\section{DSC Classroom}
\url{https://www.sc.fsu.edu/facilities}

The Department of Scientific Computing (DSC) takes pride in its cutting-edge classroom facility designed to cater to modern computational and scientific needs. Situated within the FSU campus, the classroom has been recently updated to include the latest technology, thereby providing a conducive environment for effective teaching and learning.

\par The classroom is equipped with 20 new systems, each powered by an Intel\textsuperscript{\textregistered} Core\texttrademark~i9-10940X processor, boasting 3.30GHz speed and 14 cores. These machines are fitted with a robust 64GB of RAM, ensuring rapid data processing and multitasking capabilities. The GPU in these systems is the NVIDIA RTX A5000, designed to handle intense graphics and data-intensive tasks, making them ideal for research in machine learning, data science, and scientific computing.

\par The DSC Classroom is a testament to FSU's commitment to providing state-of-the-art facilities for research and education in scientific computing. With this advanced setup, students and faculty have the resources they need for in-depth exploration and innovative problem-solving in various domains.

\section{Room Reservations}
\url{https://www.sc.fsu.edu/room-reservations}

The department hosts a state of the art computer classroom, a large seminar room, and several smaller conference rooms. Use of these facilities requires advance reservations and department faculty and staff are generally given priority access.

\subsection*{Available Rooms}
\begin{table}[h!]
\centering
\begin{tabular}{ll}
\hline
\rowcolor[HTML]{EFEFEF} 
\textbf{Title} & \textbf{Published Date} \\ \hline
Seminar Room (DSL 499) & May 12 2008 \\
Computer Classroom (DSL 152) & May 09 2008 \\
Conference Room (DSL 416) & May 09 2008 \\
Executive Conference Room (DSL 401) & May 10 2008 \\
Classroom (DSL 468) & May 09 2008 \\
Interactive Classroom (DSL 422) & August 23 2016 \\ \hline
\end{tabular}
\end{table}

\section{Room Details}

\subsection{Classroom (DSL 468)}
\url{https://www.sc.fsu.edu/room-reservations/classroom}

This classroom is located on the 4th floor of Dirac Science Library. Subject to availability, other uses of the classroom may be allowed.
\begin{description}
    \item[Room Capacity] 12 persons
\end{description}
\begin{itemize}
    \item Show On Floorplan
    \item View Room Calendar
    \item Reserve this room!
    \item Edit Calendar
\end{itemize}

\subsection{Computer Classroom (DSL 152)}
\url{https://www.sc.fsu.edu/room-reservations/computer-classroom}

The room is equipped with high-performance Intel-based workstations running LINUX and Microsoft Windows. It is used primarily for classes taught by the department. Subject to availability, other uses of the classroom may be allowed.

Our Computer Classroom is located in 152 Dirac Science Library (DSL). This is \textbf{not the main entrance} for Dirac Science Library. The entrance is on the Southeast corner of the building, next to the parking garage, at street level.
\begin{description}
    \item[Room Capacity] 18 persons
\end{description}
\begin{itemize}
    \item Show On Floorplan
    \item View Room Calendar
    \item Reserve this room!
    \item Edit Calendar
\end{itemize}

\subsection{Conference Room (DSL 416)}
\url{https://www.sc.fsu.edu/room-reservations/conference-room}

This small conference room holds about 10 people, and is used mainly for department group meetings.
\begin{description}
    \item[Room Capacity] 10 persons
\end{description}
\begin{itemize}
    \item Show On Floorplan
    \item View Room Calendar
    \item Reserve this room!
    \item Edit Calendar
\end{itemize}

\subsection{Executive Conference Room (DSL 401)}
\url{https://www.sc.fsu.edu/room-reservations/exec-conference-room}

The Executive Conference Room (DSL 401) is for the department Chair's use. Subject to availability, it is also for use by department personnel.
\begin{description}
    \item[Room Capacity] 12 persons
\end{description}
\begin{itemize}
    \item Show On Floorplan
    \item View Room Calendar
    \item Reserve this room!
    \item Edit Calendar
\end{itemize}

\subsection{Interactive Classroom (DSL 422)}
\url{https://www.sc.fsu.edu/room-reservations/interactive-classroom}

The Department of Scientific Computing supports an interactive classroom facility on the campus of FSU. The 422 classroom is on the fourth floor of Dirac Science Library. The room is equipped with 28 seats and features an interactive smart projector/whiteboard. It is used primarily for classes taught by department faculty. Subject to availability, other uses of the classroom may be allowed.
\begin{description}
    \item[Room Capacity] 28 persons
\end{description}
\begin{itemize}
    \item Show On Floorplan
    \item View Room Calendar
    \item Reserve this room!
    \item Edit Calendar
\end{itemize}

\subsection{Seminar Room (DSL 499)}
\url{https://www.sc.fsu.edu/room-reservations/seminar-room}

Our large seminar room holds up to 50 persons \& has a state-of-the-art laser rear-projected screen. Subject to availability, the seminar room may be booked by non-department faculty/staff or by \textbf{officially recognized student organizations}.
\begin{description}
    \item[Room Capacity] 50 persons
\end{description}
\begin{itemize}
    \item Show On Floorplan
    \item View Room Calendar
    \item Reserve this room!
    \item Edit Calendar
\end{itemize}

\section{Careers}
%\url{https://www.sc.fsu.edu/links/222-careers}

\subsection{SIAM.org}
\textit{Hits: 6334} \\
\textbf{Internships}, \textbf{Post Docs} \\
Careers in Applied Mathematics.

\subsection{glassdoor.com}
\textit{Hits: 9926} \\
\textbf{Internships} \\
Your Next Career Move Starts Here. Search Millions of Job Listings. See Real Employee Salaries. Read Reviews from Employees.

\subsection{FSU Career Center}
\textit{Hits: 9964} \\
\textbf{Internships} \\
The FSU Career Center provides comprehensive career services to students, alumni, employers, faculty/staff and other members of the FSU community.

\subsection{ThisIsStatistics.org}
\textit{Hits: 23899} \\
Statisticians aren’t who you think they are.

\subsection{MathJobs.Org}
\textit{Hits: 8816} \\
\textbf{Careers} \\
MATHJOBS, a website where you can find job opportunities in academics, industry, and national laboratories.

\subsection{MathPrograms.Org}
\textit{Hits: 6392} \\
\textbf{Careers} \\
MATHPROGRAMS, a website listing REU's, graduate workshops, grant opportunities.

\subsection{SIAM.org Career Center}
\textit{Hits: 5611} \\
\textbf{Careers} \\
SIAM JOBS, a list of job openings for graduates with degrees in applied mathematics and computational science.

\subsection{CRA.org - Computing Research Association}
\textit{Hits: 5206} \\
\textbf{Careers} \\
cra.org is a web site run by "Computing Research Association" which includes an extensive list of jobs in computing. (Not just computer science, but computational science, graphics, computer engineering, and so on).

\subsection{jobs.ACM.org}
\textit{Hits: 6265} \\
\textbf{Careers} \\
jobs.ACM.org is a web site maintained by the Association for Computing Machinery with an international listing of academic, industrial, and lab jobs.

\subsection{OkCupid}
\textit{Hits: 5532} \\
\textbf{Careers} \\
You might want to work here.

\subsection{Walmart}
\textit{Hits: 8036} \\
\textbf{Careers} \\
Data Science and Analytics.

\subsection{FSU Career Center: Landing a Part-time Job}
\textit{Hits: 5373} \\
\textbf{Internships} \\
When employers recruit new college graduates, they look beyond a student’s major for a skillset. Part-time jobs offer unique benefits, such as building your resume. Part-time jobs provide an opportunity to develop new skills and competencies that complement classroom learning and can be highlighted in future job searches or the graduate school application process.

\subsection{Lowes}
\textit{Hits: 12883} \\
\textbf{Careers} \\
Data Scientist

\subsection{Subcategories}
\begin{itemize}
    \item Undergraduate Opportunities
    \item Graduate Opportunities
    \item Postdoc Positions
    \item Funding Opportunities \& Fellowships
\end{itemize}

\section{Graduate Opportunities}
%\url{https://www.sc.fsu.edu/links/222-careers/226-graduate-opportunities}

\subsection{Internships - Blue Waters}
\textit{Hits: 6541} \\
\textbf{Internships} \\
National Computational Science Institute (NCSI) \\
Applications Deadline: Annually/Biannually due by end of January.

\subsection{Internships - SIAM.org}
\textit{Hits: 8290} \\
\textbf{Internships} \\
SIAM INTERNSHIPS, a list of companies and their internship offerings for students in applied mathematics and computational science.

\subsection{Internships - Risk Management Solutions}
\textit{Hits: 5919} \\
\textbf{Internships} \\
Risk Management Solutions (RMS) \\
HWind/RMS-Tallahaassee

\subsection{Internships - Pittsburgh Supercomputing Center}
\textit{Hits: 4966} \\
\textbf{Internships} \\
Pittsburgh Supercomputing Center (PSC)

\subsection{Syntech Systems}
\textit{Hits: 7328} \\
Syntech Systems, Incorporated is a dynamic defense and commercial engineering design and manufacturing firm. We specialize in automated fuel management systems and munitions support equipment. Our operations include engineering design, systems integration, independent research and development, and product manufacturing. We are experienced in designing and manufacturing according to customer concepts, product performance requirements, or item development specifications. \\
Offices located in Tallahassee, FL, United States.

\subsection{Internships - Tall Timbers}
\textit{Hits: 3553} \\
\textbf{Internships} \\
The mission of Tall Timbers is to foster exemplary land stewardship through research, conservation and education. \\
For more information please contact \textbf{Bryan Quaife}.

\subsection{Internships - Department of Navy HBCU/MI Internship Program}
\textit{Hits: 731} \\
\textbf{Internships} \\
Perform cutting edge research with world-renowned Naval Research Laboratory (NRL) scientists and engineers in Washington D.C. to strengthen America's naval defense. \\
Application Closes: January. \\
For more information please contact \textbf{Hristo Chipilski}.

\subsection{Graduate - Naval Research Enterprise Intern Program (NREIP)}
\textit{Hits: 1011} \\
Interested in summer work at NRL? \\
The Naval Research Enterprise Intern Program (NREIP) is a ten-week intern program is designed to provide opportunities for undergraduate and graduate students to participate in research, under the guidance of an appropriate mentor, at a participating Navy laboratory. The Office of Naval Research (ONR) is offering summer appointments at a Navy lab to current sophomores, juniors, seniors and graduate students from participating schools. Saxman One is handling the administration of the application process through a website. Each electronic application from a prospective intern will be sent for evaluation to the point of contact at the Navy Lab identified by the applicant. \\
For more information please contact \textbf{Hristo Chipilski}.

\section{Postdoc Positions}
%\url{https://www.sc.fsu.edu/links/222-careers/227-postdoc-positions}

\subsection{Postdoc - NRC Research Associateship Programs}
\textit{Hits: 4289} \\
Interested in a post-doc at NRL? \\
The NRC Research Associateship Programs (RAP) are prestigious postdoctoral and senior research awards designed to provide promising scientists and engineers with high-quality research opportunities at federal laboratories and affiliated institutions. These programs offer a comprehensive experience, including mentorship, access to state-of-the-art facilities, and opportunities to influence government policy, all geared toward enhancing the research career development of the Research Associates. \\
For more information please contact \textbf{Hristo Chipilski}.

\subsection{Postdoc - NCAR ASP Postdoc Fellowship}
\textit{Hits: 2716} \\
\textbf{Internships} \\
The ASP postdoctoral fellowships provide successful applicants with considerable freedom to pursue their own research interests. Fellows develop research projects in collaboration with NCAR scientists and engineers, but all are expected to choose their own research directions and are responsible for the design and conduct of their projects. \\
Deadline for applications is late October, [annually]. \\
For more information please contact \textbf{Hristo Chipilski}.

\subsection{Postdoc - NRL Postdoctoral Fellowship Program}
\textit{Hits: 767} \\
\textbf{Internships} \\
Interested in a post-doc at NRL? \\
American Society for Engineering Education (ASEE). The NRL Postdoctoral Fellowship Program provides approximately forty (40) new postdoctoral appointments per year. Fellows are competitively selected on the basis of their overall qualifications and technical proposals addressing specific areas defined by the host Navy laboratories. The selected participants will work in a unique Navy laboratory environment, while interacting with senior laboratory scientists and engineers. \\
For more information please contact \textbf{Hristo Chipilski}.

\section{Funding Opportunities \& Fellowships}
%\url{https://www.sc.fsu.edu/links/222-careers/228-funding-and-fellowships}

\subsection{DOE Computational Science Graduate Fellowship}
\textit{Hits: 5492} \\
\textbf{Fellowships} \\
Established in 1991, the Department of Energy Computational Science Graduate Fellowship (DOE CSGF) provides outstanding benefits and opportunities to students pursuing doctoral degrees in fields that use high-performance computing to solve complex science and engineering problems. The program fosters a community of energetic and committed Ph.D. students, alumni, DOE laboratory staff and other scientists who want to have an impact on the nation while advancing their research. Fellows come from diverse scientific and engineering disciplines but share a common interest in using computing in their research.

\subsection{Office of Science Graduate Student Research (SCGSR) Program}
\textit{Hits: 13266} \\
\textbf{Careers} \\
The goal of the Office of Science Graduate Student Research (SCGSR) program is to prepare graduate students for science, technology, engineering, or mathematics (STEM) careers critically important to the DOE Office of Science mission, by providing graduate thesis research opportunities at DOE laboratories. The SCGSR program provides supplemental awards to outstanding U.S. graduate students (US citizens or lawful permanent residents) to pursue part of their graduate thesis research at a DOE laboratory/facility in areas that address scientific challenges central to the Office of Science mission. The research opportunity is expected to advance the graduate students’ overall doctoral thesis while providing access to the expertise, resources, and capabilities available at the DOE laboratories/facilities.

\subsection{Oak Ridge National Laboratory (ORNL) Distinguished Staff Fellowship Program}
\textit{Hits: 5560} \\
\textbf{Fellowships} \\
ORNL’s Distinguished Staff Fellowship program aims to cultivate future scientific leaders by providing dedicated mentors, world-leading scientific resources, and enriching research opportunities at a national laboratory. The program includes the Russell, Weinberg, and Wigner Fellowships. Fellowships are awarded to outstanding early-career scientists and engineers, new to ORNL, who demonstrate success within their academic, professional, and technical areas. \\
Application Deadline: posted annualy.

\subsection{Fellowship - Graduate Visitor Program}
\textit{Hits: 1080} \\
\textbf{Internships} \\
NCAR's Advanced Study Program's Graduate Student (GVP) Fellowship is an excellent way to spend time at NCAR and work on parts of your thesis, or final project equivalent, with guidance from NCAR scientists and engineers. The GVP also is an opportunity to develop research collaborations at NCAR and to participate in professional development workshops and seminars. \\
Deadline for applications is late October, annually. \\
For more information please contact \textbf{Hristo Chipilski}.

\section{Undergraduate Opportunities}
%\url{https://www.sc.fsu.edu/links/222-careers/262-undergraduate-opportunities}

\subsection{Undergraduate - Bruins-In-Genomics (B.I.G.) Summer Undergraduate Research Program}
\textit{Hits: 5840} \\
\textbf{Internships} \\
DO YOU HAVE A PASSION FOR COMPUTING, BIOLOGY, MATH, AND SCIENCE? \\
Bruins-In-Genomics (B.I.G.) Summer Research Program is an 8-week full-time immersion program for undergraduates interested in learning how to read and analyze genes and genomes. Through this program students will have the opportunity to experience graduate-level coursework, and learn the latest cutting-edge research, tools and methods used by leading scientists to solve real-world problems. \\
ROLLING/ANNUAL APPLICATION DEADLINE: JUNE 20 TO AUGUST 11.

\subsection{Undergraduate - Naval Research Enterprise Intern Program (NREIP)}
\textit{Hits: 1032} \\
Interested in summer work at NRL? \\
The Naval Research Enterprise Intern Program (NREIP) is a ten-week intern program is designed to provide opportunities for undergraduate and graduate students to participate in research, under the guidance of an appropriate mentor, at a participating Navy laboratory. The Office of Naval Research (ONR) is offering summer appointments at a Navy lab to current sophomores, juniors, seniors and graduate students from participating schools. Saxman One is handling the administration of the application process through a website. Each electronic application from a prospective intern will be sent for evaluation to the point of contact at the Navy Lab identified by the applicant. \\
For more information please contact \textbf{Hristo Chipilski}.

\subsection{Undergraduate - STEM Student Employment Program (SSEP)}
\textit{Hits: 786} \\
Interested in summer work at NRL? \\
The STEM Student Employment Program (SSEP) is a direct hire authority for undergraduate and graduate degree seeking students enrolled in scientific, technical, engineering, or mathematics majors. Students must be continuously enrolled on at least a half-time basis at a qualifying educational institution and be at least 16 years of age. \\
For more information please contact \textbf{Hristo Chipilski}.

\section{Committees}
\texttt{URL: https://www.sc.fsu.edu/committees}

\subsection*{Articles}
\begin{itemize}
    \item Committee History
    \item Research Committee
    \item Graduate Program Committee
    \item Undergraduate Program Committee
    \item Promotion Committee
    \item Executive Committee
\end{itemize}

\subsection{Executive Committee}
\texttt{URL: https://www.sc.fsu.edu/committees/executive}

Please see our by-laws (III, C, 2 - Standing Committees) for committee description and information.

\textbf{Current committee members:}
\begin{itemize}
    \item Anke Meyer-Baese [- Fall 2023 voted]
    \item Sachin Shanbhag [Fall 2022 - Fall 2024 voted]
    \item Xiaoqiang Wang [Fall 2023, appointed]
    \item Alan Lemmon [Fall 2023, appointed]
\end{itemize}

\textit{This email address is being protected from spambots. You need JavaScript enabled to view it.}

\vspace{1em}
\textbf{Attachments:}
\begin{table}[h!]
\centering
\begin{tabular}{|l|l|l|}
\hline
\textbf{File} & \textbf{Description} & \textbf{File size} \\ \hline
2007.03.14    & Meeting Minutes    & 1 kB               \\ \hline
\end{tabular}
\end{table}

\subsection{Promotion Committee}
\texttt{URL: https://www.sc.fsu.edu/committees/promotion}

Please see our by-laws (III, C, 2 - Standing Committees) for committee description and information.

\textbf{Current committee members:}

\vspace{1em}
\textbf{For cases that involve promotion to Full Professor:}
\begin{itemize}
    \item Peter Beerli
    \item Gordon Erlebacher
    \item Alan Lemmon
    \item Anke Meyer-Baese
    \item Tomasz Plewa
    \item Sachin Shanbhag
    \item Kevin Speer
    \item Xiaoqiang Wang
\end{itemize}

\textbf{For cases that involve tenure and promotion to Associate Professor:}
\begin{itemize}
    \item Chen Huang
    \item Bryan Quaife
\end{itemize}

\textit{This email address is being protected from spambots. You need JavaScript enabled to view it.}

\subsection{Research Committee}
\texttt{URL: https://www.sc.fsu.edu/committees/research}

Please see our by-laws (III, C, 2 - Standing Committees) for committee description and information.

\textbf{Current committee members:}
\begin{itemize}
    \item Tomasz Plewa [chair]
    \item Hristo Chipilski
    \item Nicholas Dexter
    \item Olmo Zavala Romero
    \item Xiaoguang Li [ex officio]
\end{itemize}

\textit{This email address is being protected from spambots. You need JavaScript enabled to view it.}

\section{Account Policy}
\texttt{https://www.sc.fsu.edu/accounts?view=article\&id=531:account-policy\&catid=96:tech-docs}

The policy is intended to make clear when DSC users are required to change their password, what sponsors are required to do to keep a sponsored account, and to prevent sponsored accounts from becoming "orphaned" if their sponsor leaves the DSC.

(Note: Only DSC faculty and Staff are allowed to sponsor accounts)

\subsection{Password expiration}
\begin{itemize}
    \item[\textbf{180 days}] without a password change, user gets a warning requiring the change of password.
    \item[\textbf{200 days}] without a password change, user account is locked out and the sponsor is emailed about this action. The account owner must contact a member of the TSG to have account access restored.
    \item[\textbf{350 days}] without a password change, the account owner and sponsor get a message warning that the account data will be deleted in 10 days if the password is not changed by contacting a member of the TSG.
    \item[\textbf{360 days}] without a password change, user account and associated data are deleted from DSC file system.
\end{itemize}

\subsection{Sponsored account verification}
\begin{itemize}
    \item[\textbf{180 days}] without sponsor verification, the sponsor gets a warning requiring the verification of continued sponsorship.
    \item[\textbf{200 days}] without sponsor verification, the sponsored account is locked so that the account owner can no longer log in and the sponsor is informed of this action via email. The sponsor must contact a member of the TSG to have account access restored.
    \item[\textbf{350 days}] without a sponsor verification, the account owner and sponsor get a message warning them that the account and associated data will be deleted from the DSC file systems.
    \item[\textbf{360 days}] without a sponsor verification, the account and associated data is deleted from DSC file systems.
\end{itemize}

\subsection{Sponsoring account removal}
\begin{itemize}
    \item[\textbf{220 days}] without a sponsor password change, the sponsoring and sponsored accounts get a warning.
    \item[\textbf{290 days}] without a sponsor password change, the sponsoring and sponsored accounts are locked.
    \item[\textbf{350 days}] without a sponsor password change, the sponsoring and sponsored accounts get a message warning that the accounts and associated data will be deleted in 10 days.
    \item[\textbf{360 days}] without a sponsor password change, the sponsoring and sponsored accounts and associated data are deleted from the DSC file systems.
\end{itemize}

\end{document}